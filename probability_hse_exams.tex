\documentclass[12pt, a4paper]{article}\usepackage[]{graphicx}\usepackage[]{color}
%% maxwidth is the original width if it is less than linewidth
%% otherwise use linewidth (to make sure the graphics do not exceed the margin)
\makeatletter
\def\maxwidth{ %
  \ifdim\Gin@nat@width>\linewidth
    \linewidth
  \else
    \Gin@nat@width
  \fi
}
\makeatother

\definecolor{fgcolor}{rgb}{0.345, 0.345, 0.345}
\newcommand{\hlnum}[1]{\textcolor[rgb]{0.686,0.059,0.569}{#1}}%
\newcommand{\hlstr}[1]{\textcolor[rgb]{0.192,0.494,0.8}{#1}}%
\newcommand{\hlcom}[1]{\textcolor[rgb]{0.678,0.584,0.686}{\textit{#1}}}%
\newcommand{\hlopt}[1]{\textcolor[rgb]{0,0,0}{#1}}%
\newcommand{\hlstd}[1]{\textcolor[rgb]{0.345,0.345,0.345}{#1}}%
\newcommand{\hlkwa}[1]{\textcolor[rgb]{0.161,0.373,0.58}{\textbf{#1}}}%
\newcommand{\hlkwb}[1]{\textcolor[rgb]{0.69,0.353,0.396}{#1}}%
\newcommand{\hlkwc}[1]{\textcolor[rgb]{0.333,0.667,0.333}{#1}}%
\newcommand{\hlkwd}[1]{\textcolor[rgb]{0.737,0.353,0.396}{\textbf{#1}}}%
\let\hlipl\hlkwb

\usepackage{framed}
\makeatletter
\newenvironment{kframe}{%
 \def\at@end@of@kframe{}%
 \ifinner\ifhmode%
  \def\at@end@of@kframe{\end{minipage}}%
  \begin{minipage}{\columnwidth}%
 \fi\fi%
 \def\FrameCommand##1{\hskip\@totalleftmargin \hskip-\fboxsep
 \colorbox{shadecolor}{##1}\hskip-\fboxsep
     % There is no \\@totalrightmargin, so:
     \hskip-\linewidth \hskip-\@totalleftmargin \hskip\columnwidth}%
 \MakeFramed {\advance\hsize-\width
   \@totalleftmargin\z@ \linewidth\hsize
   \@setminipage}}%
 {\par\unskip\endMakeFramed%
 \at@end@of@kframe}
\makeatother

\definecolor{shadecolor}{rgb}{.97, .97, .97}
\definecolor{messagecolor}{rgb}{0, 0, 0}
\definecolor{warningcolor}{rgb}{1, 0, 1}
\definecolor{errorcolor}{rgb}{1, 0, 0}
\newenvironment{knitrout}{}{} % an empty environment to be redefined in TeX

\usepackage{alltt}

%\usepackage[pdf]{pstricks} % to use QR barcodes
%\usepackage{pst-barcode}
% sudo yum install texlive-auto-pst-pdf
% sudo yum install texlive-pst-barcode texlive-pdfcrop

% RNW RNW RNW RNW RNW RNW RNW RNW RNW RNW RNW RNW RNW RNW RNW RNW RNW RNW RNW RNW RNW RNW RNW RNW RNW RNW RNW RNW




\usepackage{fontspec}
\usepackage{polyglossia}

\setmainlanguage{russian}
\setotherlanguages{english}

% download "Linux Libertine" OTF-fonts:
% http://www.linuxlibertine.org/index.php?id=91&L=1
\setmainfont{Linux Libertine O} % or Helvetica, Arial, Cambria
% why do we need \newfontfamily:
% http://tex.stackexchange.com/questions/91507/
\newfontfamily{\cyrillicfonttt}{Linux Libertine O}
\newfontfamily{\cyrillicfont}{Linux Libertine O}
\newfontfamily{\cyrillicfontsf}{Linux Libertine O}

\usepackage{etoolbox} % to use ifdef, must be after babel
\input{title_bor_utf8} % use local copy

\usepackage{epigraph}

\AddEnumerateCounter{\asbuk}{\russian@alph}{щ} % для списков с русскими буквами
\setlist[enumerate, 2]{label=\asbuk*),ref=\asbuk*}


\unitlength=0.6mm

\title{Подборка экзаменов по теории вероятностей. \\Факультет экономики, НИУ-ВШЭ}
\date{\today}
\author{Коллектив кафедры \\
математической экономики и эконометрики,\\
 фольклор}


%%%%%%%%%%%%%%%%%% вставки
%%%%%%%%%%%%%%%%%%%%%%%%%%%%%%%%%%%%%%% Списки без уродских отступов
\newenvironment{enumerate*}{
\begin{enumerate}
  \setlength{\itemsep}{0pt}
  \setlength{\parskip}{0pt}
  \setlength{\parsep}{0pt}
}{\end{enumerate}}

\newenvironment{itemize*}{
\begin{itemize}
  \setlength{\itemsep}{0pt}
  \setlength{\parskip}{0pt}
  \setlength{\parsep}{0pt}
}{\end{itemize}}

\abovedisplayskip=0mm
\abovedisplayshortskip=0mm
\belowdisplayskip=0mm
\belowdisplayshortskip=0mm
%%%%%%%%%%%%%%%%%%%%%%%%%%%%%%%%%%%%%%%%%%%%%%%%%%%%%%%%%%%%%%%%%%%%%%
\newcommand{\MIN}{\textbf{(MIN)}{}}
\newcommand{\ofbr}[1]{\bigl( \{ #1 \} \bigr)}     % Например, вероятность события. Большие круглые, нормальные фигурные скобки вокруг аргумента
%%%%%%%%%%%%%%%%%
\newenvironment{centered}{%
  \begin{list}{}{%
    \topsep0pt
  }
  \centering
  \item[]
}
{\end{list}}
%%%%%%%%%%%%%%%%%%%%%%%%%%%%%%%%%%%%%%%%%%%%%%%%%%%%%%%%%%%%%%%%%%%%%%%%%




\DeclareMathOperator*\plim{plim}
\newcommand{\cN}{\mathcal{N}}
\IfFileExists{upquote.sty}{\usepackage{upquote}}{}
\begin{document}
\maketitle

\tableofcontents{}


\parindent=0 pt % no indent

\section{Описание}

Свежую версию можно скачать с блога \url{http://pokrovka11.wordpress.com/} или с github репозитория \url{http://bdemeshev.github.io/pr201/}.


Уникальное предложение для студентов факультета экономики ГУ-ВШЭ:


Найдите ошибки в этом документе или пришлите отсутствующие решения в техе и получите дополнительные бонусы! Найденные смысловые ошибки поощряются сильнее, чем просто опечатки. Замеченные ошибки и новые решения оформляйте в виде issues на \url{https://github.com/bdemeshev/probability_hse_exams/issues/}. Перед публикацией issue, пожалуйста, свертесь со свежей версией подборки.

Неполный список благодарностей:

\begin{enumerate}
\item Андрей Зубанов, решения (экзамен 26.03.2012, \ldots)
\item Кирилл Пономарёв, решения (контрольная 1, 2014)
\item Александр Левкун, решения (контрольная 1, 2014)
\item Оля Гнилова, решения (кр 3 2011, 2014, 2015, 2016, \ldots)
\item Настя Жаркова
\item Гарик Варданян
\end{enumerate}


\section{Доброе напутствие пишущим эту подборку :)}

Здесь перечислены стилевые особенности коллекции, а узнать технические подробности по теху можно, например, \href{http://www.ccas.ru/voron/download/voron05latex.pdf}{в учебнике} К.В. Воронцова.

\begin{enumerate}

\item Дробную часть числа отделяй от целой точкой: $3.14$ — хорошо, $3{,}14$ — плохо.
\item Существует длинное тире, —, которое отличается от просто дефиса - и нужно, чтобы разделять части предложения. \href{https://ru.wikihow.com/напечатать-тире}{Инструкция в картинках по набору тире :)}
\item Выключные формулы следует окружать \verb|\[|\ldots\verb|\]|. Никаких \$\$\ldots\$\$!
\item Про остальные окружения: для системы уравнений подойдёт cases, для формул на несколько строк – multline*, для нумерации – enumerate.
\item Русский текст внутри формулы нужно писать в \verb|\text{|\ldots\}.
\item Для многоточий существует команда \verb|\ldots|.
\item В преамбуле определены сокращения! Самые популярные: \verb|\P, \E, \Var, \Cov, \Corr, \cN|.
\item Названия функций тоже идут со слэшем: \verb|\ln, \exp, \cos|\ldots
\item Таблицы нужно оформлять по стандарту booktabs. Самый удобный способ сделать это – зайти на
\href{https://www.tablesgenerator.com}{tablesgenerator} и выбрать там опцию booktabs table style вместо default table style.
\item Уважай букву ё – ставь над ней точки! :)
\end{enumerate}


\section{2004-2005}

\input{chapters/year_2004_2005.tex}

\section{2005-2006}

\input{chapters/year_2005_2006.tex}

\section{2006-2007}

\subsection{Контрольная работа №1, ??.11.2006}

Вывешенное решение может содержать неумышленные опечатки.

Заметил опечатку? Сообщи преподавателю!

\begin{enumerate}
\item  Из семей, имеющих троих разновозрастных детей, случайным
образом выбирается одна семья. Пусть событие А заключается в том,
что в этой семье
старший ребенок — мальчик, В — в семье есть хотя бы одна девочка.
\begin{enumerate}
\item Считая вероятности рождения мальчиков и девочек одинаковыми,
выяснить, являются ли события А и В независимыми.
\item Изменится ли результат, если вероятности рождения мальчиков и
девочек различны.
\end{enumerate}
Решение:
\begin{enumerate}
\item[а)] $\P(A)=0.5$, $\P(B)=1-\P(B^{c})=1-0.5^{3}=\frac{7}{8}$, $\P(A\cap
B)=0.5\cdot (1-0.5^{2})=\frac{3}{8}$, $\P(A\cap B)\neq \P(A)\P(B)$,
события зависимы.
\item[б)] $\P(A)=p$, $\P(B)=1-p^{3}$, $\P(A\cap B)=p(1-p^{2})$,
независимость событий возможна только при $p=0$ или $p=1$
\end{enumerate}
\item Студент решает тест (множественного выбора) проставлением
ответов наугад. В тесте 10 вопросов, на каждый из которых 4
варианта ответов. Зачет ставится в том случае, если правильных
ответов будет не менее 5.
\begin{enumerate}
\item Найти вероятность того, что студент правильно ответит только
на один вопрос
\item Найти наиболее вероятное число правильных ответов
\item Найти математическое ожидание и дисперсию числа правильных
ответов
\item Найти вероятность того, что студент получит зачет
\end{enumerate}
Решение:

Пусть $X$ - число правильных ответов.
\begin{enumerate}
\item[а)] $\P(X=1)=C_{10}^{1}\left(\frac{1}{4}\right)^{1}\left(\frac{3}{4}\right)^{9}$
\item[б)] $k_{\P(X=k)\rightarrow \max}=\lfloor p(n+1)\rfloor=\lfloor
\frac{11}{4}\rfloor=2$ (можно не зная формулы просто выбрать
наибольшую вероятность)
\item[в)] $\E(X)=10\E(X_{i})=\frac{10}{4}$
$\Var(X)=10\Var(X_{i})=10 \cdot \frac{1}{4} \cdot \frac{3}{4}$
\item[г)] $\sum_{i=5}^{10}C_{10}^{i}\left(\frac{1}{4}\right)^{i}\left(\frac{3}{4}\right)^{10-i}$
\end{enumerate}

\item Вероятность изготовления изделия с браком на некотором
предприятии равна 0.04. Перед выпуском изделие подвергается
упрощенной проверке, которая в случае бездефектного изделия
пропускает его с вероятностью 0.96, а в случае изделия с дефектом
- с вероятностью 0.05. Определить:
\begin{enumerate}
\item Какая часть изготовленных изделий выходит с предприятия
\item Какова вероятность того, что изделие, прошедшее упрощенную
проверку, бракованное
\end{enumerate}
Решение:

$A$ - изделие браковано, $B$ - изделие признано хорошим
\begin{enumerate}
\item[а)] $\P(B)=0.96\cdot 0.96+0.04\cdot 0.05$
\item[б)] $\P(A|B)=\frac{0.04\cdot 0.05}{\P(B)}$
\end{enumerate}
\item  Вероятность того, что пассажир, купивший билет, не придет к
отправлению поезда, равна 0.01. Найти вероятность того, что все
400 пассажиров явятся к отправлению поезда (использовать
приближение Пуассона).

Решение:

$\lambda=np=4$

$\P(X=k)=e^{-\lambda}\frac{\lambda^{k}}{k!}$

$\P(X=0)=e^{-4}$

\item Охотник, имеющий 4 патрона, стреляет по дичи до первого
попадания или до израсходования всех патронов. Вероятность
попадания при первом выстреле равна 0.6, при каждом последующем -
уменьшается на 0.1. Найти
\begin{enumerate}
\item Закон распределения числа патронов, израсходованных охотником
\item Математическое ожидание и дисперсию этой случайной величины
\end{enumerate}
Решение:
\begin{enumerate}
\item[а)]
\begin{tabular}{|c|c|c|c|c|}
  \hline
  % after \\: \hline or \cline{col1-col2} \cline{col3-col4} ...
  $x_{i}$ & 1 & 2 & 3 & 4 \\
  \hline
  $\P(X=x_{i})$ & $0.6$& $(1-0.6)\cdot 0.5$ & $(1-0.6)\cdot(1-0.5)\cdot 0.4$ & $1-p_{1}-p_{2}-p_{3}$ \\
  \hline
\end{tabular} \\
\begin{tabular}{|c|c|c|c|c|}
  \hline
  % after \\: \hline or \cline{col1-col2} \cline{col3-col4} ...
  $x_{i}$ & 1 & 2 & 3 & 4 \\
  \hline
  $\P(X=x_{i})$ & $0.6$& $0.2$ & $0.08$ & $0.12$ \\
  \hline
\end{tabular}
\item[б)] $\E(X)=1.7$, $\Var(X)\approx 1.08$
\end{enumerate}
\item Поезда метрополитена идут регулярно с интервалом 2 минуты.
Пассажир приходит на платформу в случайный момент времени. Какова
вероятность того, что ждать пассажиру придется не более полминуты.
Найти математическое ожидание и дисперсию времени ожидания поезда.

Решение:

$\P(X\le 0.5)=\frac{0.5}{2}=0.25$, $\E(X)=\frac{0+2}{2}=1$ (здравый
смысл)

$\Var(X)=\E(X^{2})-(\E(X))^{2}$

$\E(X^{2})=\int_{0}^{2}t^{2}\cdot p(t)dt=\int_{0}^{2}t^{2}\cdot 0.5dt=\frac{4}{3}$

\item Время работы телевизора «Best» до первой поломки является
случайной величиной, распределенной по показательному закону.
Определить вероятность того, что телевизор проработает более 15
лет, если среднее время безотказной работы телевизора фирмы «Best»
составляет 10 лет. Какова вероятность, что телевизор,
проработавший 10 лет, проработает еще не менее 15 лет?

Решение:

$\E(X)=10=\frac{1}{\lambda}$, $\lambda=\frac{1}{10}$, $p(t)=\lambda e^{\lambda t}$ при $t>0$

$\P(X>15)=\int_{15}^{\infty}p(t)dt=...=e^{-\frac{3}{2}}$

$\P(X>25|X>10)=\frac{\P(X>25)}{\P(X>10)}=...=e^{-\frac{3}{2}}$
\end{enumerate}

Дополнительная задача:

Пусть случайные величины $X_{1}$ и $X_{2}$ независимы и равномерно
распределены на отрезках $[-1;1]$ и $[0;1]$, соответственно. Найти
вероятность того, что $\max\{X_{1},X_{2}\}>0.5$, функцию
распределения случайной величины $Y=\max\{X_{1},X_{2}\}$.

Решение:

Функция распределения:

$F_{Y}(t)=\P(Y\le t)=\P(\max\{X_{1},X_{2}\}\le t)=\P(X_{1}\le t\cap X_{2}\le t)=\P(X_{1}\le t)\P(X_{2}\le t)=\frac{t+1}{2}\cdot t$ при $t\in [0;1]$.

При $t>1$ получаем, что $F_{Y}(t)=1$ и при $t<0$ получаем, что $F_{Y}(t)=0$.

$\P(\max\{X_{1},X_{2}\}>0.5)=1-\P(\max\{X_{1},X_{2}\}\le 0.5)=1-F(0.5)=\frac{5}{8}$

\subsection{Контрольная работа №2, 27.01.2007}

\subsubsection*{Часть I.}

Обведите верный ответ:

\begin{enumerate}
\item Сумма двух нормальных независимых случайных величин нормальна.
Да.
\item Нормальная случайная величина может принимать отрицательные
значения. Да.
\item Пуассоновская случайная величина является непрерывной. Нет.
\item Дисперсия суммы зависимых величин всегда не меньше суммы
дисперсий. Нет.
\item Теорема Муавра-Лапласа является частным случаем центральной
предельной. Да.
\item Пусть $X$ - длина наугад выловленного удава в сантиметрах, а
$Y$ - в дециметрах. Коэффициент корреляции между этими
величинами равен $\frac{1}{10}$. Нет.
\item Математическое ожидание выборочного среднего не зависит от
объема выборки, если $X_{i}$ одинаково распределены. Да.
\item Зная закон распределения $X$ и закон распределения $Y$
можно восстановить совместный закон распределения пары $(X,Y)$. Нет.
\item Если  $X$  - непрерывная случайная величина,  $\E\left(X\right)=6$  и
$\Var\left(X\right)=9$ , то  $Y=\frac{X-6}{3} \sim
\cN\left(0;1\right)$.  Нет.
\item Если ты отвечать на первые 10 вопросов этого теста наугад, то
число правильных ответов - случайная величина, имеющая
биномиальное распределение. Да.
\item По-моему, сегодня хорошая погода, и вместо контрольной можно
было бы покататься на лыжах. Да!
\end{enumerate}

$[$правильно=+1 балл; нет ответа=неправильно=0 баллов$]$

Любой ответ на 11 считается правильным.

Тест не является блокирующим.

Обозначения:

$\E(X)$ - математическое ожидание

$\Var(X)$ - дисперсия

\subsubsection*{Часть II.}

Стоимость задач 10 баллов.

\begin{enumerate}
% числа выверены
\item Совместный закон распределения случайных величин  $X$  и  $Y$
задан таблицей:

$\begin{array}{|c|ccc|} \hline {} & {Y=-1} & {Y=0} & {Y=2}
\\  \hline {X=0} & {0.1} & {c} & {0.2}
\\ {X=1} & {0.1} & {0.2} & {0.1} \\  \hline  \end{array}$

Найдите  $c$,  $\P\left(Y>-X\right)$,  $\E\left(X\cdot Y^{2} \right)$,  $\E\left(Y|X>0\right)$

Ответы: $c=0.3$ $[1]$, $\P(Y>-X)=0.5$ $[3]$, $\E(XY^{2})=0.5$ $[3]$,
$\E(Y|X>0)=\frac{0.1}{0.4}=0.25$ $[3]$

% числа выверены
\item Случайный вектор  $\left(\begin{array}{c}
{X_{1} } \\ {X_{2} }
\end{array}\right)$  имеет нормальное распределение с
математическим ожиданием  $\left(\begin{array}{c} {2} \\ {-1}
\end{array}\right)$  и ковариационной матрицей
$\left(\begin{array}{cc} {9} & {-4.5} \\ {-4.5} & {25}
\end{array}\right)$ . Найдите  $\P\left(X_{1} +3X_{2} >20\right)$.

Ответы: $\E(Y)=-1$ $[2]$, $\Var(Y)=207$ $[4]$, $\P(Y>20)=\P(Z>\frac{21}{\sqrt{207}})=\P(Z>1.46)=0.07$ $[4]$

% числа выверены
\item Совместная функция плотности имеет вид
\[
p_{X,Y} \left(x,y\right)=
\begin{cases}
x+y, & \text{ если } x\in \left[0;1\right],\, y\in \left[0;1\right] \\
0, & \text{ иначе}
\end{cases}
\]
Найдите  $\P\left(Y>2X\right)$ ,  $\E\left(X\right)$

Решение:

$\P(Y>2X)=\int_{0}^{1}\int_{0}^{y/2}(x+y)dxdy=\frac{5}{24}$ $[5]$

$\E(X)=\int_{0}^{1}\int_{0}^{1}x(x+y)dxdy=\frac{7}{12}$ $[5]$

(если интеграл выписан верно, но не взят, то $[3]$ вместо $[5]$)

% числа выверены
\item В супермаркете «Покупан» продаются различные вина:

\begin{tabular}{|c|c|c|c|}
  \hline
  % after \\: \hline or \cline{col1-col2} \cline{col3-col4} ...
  Вина & Доля & Средняя цена за бутылку (у.е.) & Стандартное отклонение (у.е.) \\
  \hline
  Элитные & 0.1 & 150 & 24 \\
  Дорогие & 0.3 & 40 & 12 \\
  Дешевые & 0.6 & 10 & 10 \\
  \hline
\end{tabular}

Чтобы оценить среднюю стоимость предлагаемого вина производится
случайная выборка 10 бутылок.
\begin{enumerate}
\item Какое количество элитных, дорогих и дешевых вин должно
присутствовать в выборке, для того, чтобы выборочное среднее
значение цены имело минимальную дисперсию? $[5]$
\item Чему равна минимальная дисперсия? $[5]$
\end{enumerate}
Решение:

Используя метод множителей Лагранжа:

$L=\frac{(0.1\cdot 24)^{2}}{a}+\frac{(0.3\cdot
12)^{2}}{b}+\frac{(0.6\cdot 10)^{2}}{c}+\lambda(10-a-b-c)$

\ldots

$a=2$, $b=3$, $c=5$, можно было использовать готовую формулу
$n_{i}=\frac{w_{i}\sigma_{i}}{\sum w_{j}\sigma_{j}}$

$\Var(\overline{X}^{s})=14.4$

% числа выверены
\item Допустим, что закон распределения $X_{n}$ имеет вид:

\begin{tabular}{|c|c|c|c|}
  \hline
  X & -1 & 0 & 2 \\
  \hline
  Prob & $\theta$ & $2\theta-0.2$ & $1.2-3\theta$ \\
  \hline
\end{tabular}

Имеется выборка: $X_{1}=0$, $X_{2}=2$.
\begin{enumerate}
\item Найдите оценку $\hat{\theta}$ методом максимального правдоподобия
\item Найдите оценку $\hat{\theta}$ методом моментов
\end{enumerate}

Решение:
\begin{enumerate}
\item[а)] $(2\theta-0.2)(1.2-3\theta)\rightarrow\max$,
$\hat{\theta}=0.25$ $[5]$\\
\item[б)] $2.4-7\hat{\theta}=1$, $\hat{\theta}=0.2$ $[5]$
\end{enumerate}

% числа выверены
\item В среднем 30\% покупателей супермаркета делают покупку на сумму
свыше 700 рублей. Какова вероятность того, что из 200 $[$случайно
выбранных$]$ покупателей
более 33\% сделают покупку на сумму свыше 700 рублей?

Решение:

$\P(\overline{X}>0.33)=\P\left(\frac{\bar{X}-0.3}{\sqrt{\frac{0.3\cdot
0.7}{200}}}>\frac{0.33-0.3}{\sqrt{\frac{0.3\cdot
0.7}{200}}}\right)=\P(Z>1.03)=0.15$

Баллы: $[3]$ - $\Var$, $[4]$ - $Z$, $[3]$ - таблица

% числа выверены
\item Пусть $X_{i}$ нормально распределены и
независимы. Имеется выборка
из трех наблюдений: 2, 0, 1.
\begin{enumerate}
\item[a)] Найдите несмещенные оценки для математического ожидания и
дисперсии, $\bar{X}$ и $\hat{\sigma}^{2}$. $[2]+[3]$
\item[б)] Найдите вероятность того, что оценка дисперсии превосходит
истинную дисперсию более чем в 3 раза $[5]$
\end{enumerate}
Решение:

$\bar{X}=1$, $\hat{\sigma}^{2}=1$

$\P(\hat{\sigma}^{2}>3\sigma^{2})=\P\left(2\frac{\hat{\sigma}^{2}}{\sigma^{2}}>6\right)=\P(\chi_{2}^{2}>6)=0.05$

 % числа выверены
\item Известно, что у случайной величины $X$ есть
математическое
ожидание, $\E(X)=0$, и дисперсия.
\begin{enumerate}
\item[а)] Укажите верхнюю границу для $\P(X^{2}>4\Var(X))$? $[5]$
\item[б)] Найдите указанную вероятность, если дополнительно известно, что
$X$ нормально распределена. $[5]$
\end{enumerate}
Решение:
\begin{enumerate}
\item[a)] $\P(X^{2}>4\Var(X))=\P(|X-0|>2\sigma)\le
\frac{Var{X}}{4\Var(X)}=\frac{1}{4}$
\item[б)] $\P(X^{2}>4\Var(X))=\P(|Z|>2)=0.05$
\end{enumerate}

% числа выверены
\item Пусть $X_{i}$ независимы и экспоненциально
распределены, т.е. имеют функцию плотности вида
$p(t)=\frac{1}{\theta}e^{-\frac{1}{\theta}t}$ при $t>0$.
\begin{enumerate}
\item Постройте оценку математического ожидания методом максимального
правдоподобия $[2]$
\item Является ли оценка несмещенной? $[2]$
\item Найдите дисперсию оценки $[2]$
\item С помощью неравенства Крамера-Рао проверьте, является ли
оценка эффективной среди несмещенных оценок? $[2]$
\item Является ли построенная оценка состоятельной? $[2]$
\end{enumerate}
Решение:
\begin{enumerate}
\item[а)] $\bar{X}$
\item[б)] Да;
\item[в)] $\Var(\bar{X})=\frac{\theta^{2}}{n}$;
\item[г)] да: несмещенность и предел дисперсии равный нулю;
\end{enumerate}

 % числа выверены
\item Независимые случайные величины $X_{i}$ распределены
равномерно на отрезке $[0;a]$, известно, что $a>10$. Исследователь
хочет оценить
параметр $\theta=\frac{1}{\P(X_{i}<5)}$.
\begin{enumerate}
\item Используя $\bar{X_{n}}$ постройте несмещенную оценку
$\hat{\theta}$ для $\theta$ $[4]$
\item Найдите дисперсию построенной оценки $[3]$
\item Является ли построенная оценка состоятельной? $[3]$
\end{enumerate}
Решение:
\begin{enumerate}
\item[a)] $\E(\bar{X})=\frac{a}{2}$,
$\theta=\frac{1}{\P(X_{i}<5)}=\frac{1}{5/a}=\frac{1}{5}a$ \\
$\hat{\theta}=\frac{2}{5}\bar{X}$
\item[б)] $\Var(\hat{\theta}_{n})=(\frac{2}{5})^{2}\cdot\frac{a^{2}}{12n}$
\item[в)] $\lim \Var(\hat{\theta}_{n})=0$, оценка несмещенная,
следовательно, состоятельная.
\end{enumerate}
\end{enumerate}

\subsubsection*{Часть III.}

Стоимость задачи 20 баллов.

Требуется решить \textbf{\underbar{одну}} из двух 11-х задач по
выбору!

\begin{enumerate}
\item[11-A.] Каждый день Кощей Бессмертный кладет в сундук случайное количество
копеек (от одной до ста, равновероятно). Сколько в среднем дней нужно Кощею, чтобы набралось не меньше рубля?

Решение:

Обозначим $e_{n}$ - сколько дней осталось в среднем ждать, если
уже набрано $n$ копеек.

Тогда:

$e_{100}=0$

$e_{99}=1$

$e_{98}=\frac{1}{100}e_{99}+\frac{99}{100}e_{100}+1=1+\frac{1}{100}$

$e_{97}=\frac{1}{100}e_{98}+\frac{1}{100}e_{99}+\frac{98}{100}e_{100}+1=(1+\frac{1}{100}))^{2}$

$e_{96}=\frac{1}{100}e_{97}+\frac{1}{100}e_{98}+\frac{1}{100}e_{99}+\frac{97}{100}e_{100}+1=(1+\frac{1}{100})^{3}$

\ldots

По индукции легко доказать, что $e_{n}=(1+\frac{1}{100})^{99-n}$

Таким образом, $e_{0}=(1+\frac{1}{100})^{99}=2.718 \ldots$

\item[11-B.] Каждый день Петя знакомится с новыми девушками. С вероятностью 0.7
ему удается познакомиться с одной девушкой; с вероятностью 0.2 — с
двумя; с вероятностью 0.1 — не удается. Дни, когда Пете не удается
познакомиться ни с одной девушкой, Петя считает неудачными.

Какова вероятность, что до первого неудачного дня Пете удастся
познакомиться $[$ровно$]$ с 30-ю девушками?

Решение:

$p_{0}=0.1$, $p_{1}=0.7\cdot 0.1$;

$p_{n}=\P($в первый день Петя познакомился с одной
девушкой$)p_{n-1}+\P($в первый день Петя познакомился с двумя
девушками$)p_{n-2}$;

Разностное уравнение: $p_{n}=0.7p_{n-1}+0.2p_{n-2}$

\emph{Подсказка}: Думайте!
\end{enumerate}


\subsection{Контрольная работа №3, 21.02.2007}

Нужные и ненужные формулы: \\ \\
$T$ - сумма чего-то там. \\
Если $H_{0}$ верна, то $\E(T)=\frac{n}{2}$ и $\Var(T)=\frac{n}{4}$ \\ \\
$T$ - сумма каких-то рангов. \\
Если $H_{0}$ верна, то $\E(T)=\frac{n(n+1)}{4}$ и
$\Var(T)=\frac{n(n+1)(2n+1)}{24}$. \\ \\
$T$ - сумма каких-то рангов. \\
Если $H_{0}$ верна, то $\E(T)=\frac{n_{1}(n_{1}+n_{2}+1)}{2}$,
$\Var(T)=\frac{n_{1}n_{2}(n_{1}+n_{2}+1)}{12}$. \\ \\
$\cos^{2}(x)+\sin^{2}(x)=1$ \\ \\

\textbf{УДАЧИ!}

\subsubsection*{Часть I.}

Обведите нужный ответ

\begin{enumerate}
\item Если $X\sim \cN(0;12)$, $Y\sim \cN(12,24)$, $\Corr(X,Y)=0$, то
$X+Y\sim \cN(12,36)$.
Да. Нет.

$[$любой ответ считался правильным. на самом деле верный ответ -
нет$]$

\item Если закон распределения $X$ задан табличкой

\begin{tabular}{|c|c|c|}
  \hline
  $x$ & 0 & 1 \\
  \hline
  $\P$ & 0.5 & 0.5 \\
  \hline
\end{tabular}, то $X$ - нормально распределена. Да. Нет.

\item Непараметрические тесты неприменимы, если выборка имеет
$\chi^{2}$ распределение. Да. Нет.
\item P-значение показывает вероятность отвергнуть нулевую
гипотезу, когда она верна. Да. Нет.
\item Если $t$-статистика равна нулю, то P-значение также равно
нулю. Да. Нет.
\item Если гипотеза отвергает при 5\%-ом уровне значимости, то
она будет отвергаться и при 1\%-ом уровне значимости. Да. Нет.
\item При прочих равных 90\% доверительный интервал шире 95\%-го. Да. Нет.
\item Значение функции плотности может превышать единицу. Да. Нет.
\item Для любой случайной величины  $\E(X^{2} )\ge
(\E(X))^{2}$. Да. Нет.
\item Если $\Corr(X,Y)>0$, то $\E(X)\E(Y)<\E(XY)$. Да. Нет.
\item На экзаменационной работе не шутят! Нет, шутят.
\end{enumerate}

$[$правильно=+1 балл; нет ответа=неправильно=0 баллов$]$

Ответ «да» означает истинное утверждение, ответ «нет» - ложное.

Тест не является блокирующим.

$[$Неправильное использование таблиц = штраф 2 балла$]$

$[$Неправильные степени свободы = штраф 2 балла$]$

\subsubsection*{Часть II.}

Стоимость задач 10 баллов.

\begin{enumerate}
 % числа выверены
\item Из урны с 5 белыми и 7 черными шарами случайным образом вынимается
2 шара. Случайная величина $X$ принимает значение (-1), если оба
шара - белые; 0, если шары разного цвета и 1, если оба шара
черные.
\begin{enumerate}
\item Найдите $\P(X=-1)$ $[2]$ , $\E(X)$ $[3]$, $\Var(X)$ $[3]$
\item Постройте функцию распределения величины $X$ $[2$, достаточно аккуратно выписать функцию$]$
\end{enumerate}

 % числа выверены
\item Случайная величина $X$ имеет функцию распределения
\[
F_{X}(t)=
\begin{cases}
  0, & t<0 \\
  ct^{2}, & 0\le t <1 \\
  1, & 1\le t \\
\end{cases}
\]
\begin{enumerate}
\item Найдите $c$ $[1]$, $\P(0.5<X<2)$ $[1]$, 25\%-ый квантиль $[1]$
\item Найдите $\E(X)$ $[2]$, $\Var(X)$ $[2]$, $\Cov(X,-X)$ $[1]$, $\Corr(2X,3X)$ $[1]$
\item Выпишите функцию плотности величины $X$ $[1]$
\end{enumerate}

% числа выверены
\item Доходности акций двух компаний являются случайными величинами $X$
и $Y$ с одинаковым математическим ожиданием и ковариационной
матрицей  $\left(%
\begin{array}{cc}
  4 & -2 \\
  -2 & 9 \\
\end{array}%
\right).$
\begin{enumerate}
\item Найдите $\Corr(X,Y)$  $[1]$,

Ответ: $\Corr=-\frac{1}{3}$
\item $[5]$ В какой пропорции нужно приобрести акции этих двух
компаний, чтобы дисперсия доходности получившегося портфеля была наименьшей?
\item  $[2]$ Можно ли утверждать, что величины $X+Y$ и $7X-2Y$ независимы?
\item $[2]$ Изменится ли ответ на пункт «в», если дополнительно
известно, что величины $X$ и $Y$ в совокупности нормально распределены?
\end{enumerate}
Подсказка: Если $R$ - доходность портфеля, то $R=\alpha
X+(1-\alpha)Y$

Ответ: $\alpha=\frac{11}{17}$

% числа выверены
\item Проверка 40 случайно выбранных лекций показала, что студент
Халявин присутствовал только на двух из них.
\begin{enumerate}
\item{} $[4]$ Найдите 90\%-ый доверительный интервал для вероятности
увидеть Халявина на лекции.
\item{} $[5]$ Укажите минимальный размер выборки, необходимый для того,
чтобы с вероятностью 0.9 выборочная доля посещаемых Халявиным
лекций отличалась от соответствующей вероятности не более, чем на 0.1.
\item{} $[1]$ Какие предпосылки и теоремы использовались при ответах на предыдущие пункты?
\end{enumerate}

% числа выверены
\item Изучается эффективность нового метода обучения. У группы из 40
студентов, обучавшихся по новой методике, средний бал на экзамене
составил 322.12, а выборочное стандартное отклонение 54.53.
Аналогичные показатели для независимой выборки из 60 студентов
того же курса, обучавшихся по старой методике,
приняли значения 304.61 и 62.61 соответственно.
\begin{enumerate}
\item{} $[4]$ Проверьте гипотезу о равенстве дисперсий оценок в двух
группах.
\item{} $[1]$ Какие предпосылки использовались при ответе на «а»?
\item{} $[4]$ Постройте 90\% доверительный интервал для разницы
математических ожиданий оценок в двух группах
\item{} $[1]$ Можно ли считать новую методику более эффективной?
\end{enumerate}

% числа выверены
\item В парке отдыха за час 57 человек посетило аттракцион «Чертово
колесо», 48 - «Призрачные гонки» и 54 - «Американские горки». Можно ли на 5\% уровне значимости считать, что посетители
одинаково любят эти три аттракциона?

% числа выверены
\item Можно ли по имеющейся таблице утверждать о независимости пола и
доминирующей руки на 5\% уровне значимости?

\begin{tabular}{|c|c|c|}
  \hline
  Пол/рука & Правша & Левша \\
  \hline
  Мужчины & 16 & 76 \\
  Женщины & 25 & 81 \\
  \hline
\end{tabular}

% числа выверены
\item Пусть $X_{i}$ нормально распределены, независимы, $\E(X_{i})=0$,
$\Var(X_{i})=\theta$.
\begin{enumerate}
\item{} $[3]$ Постройте оценку $\hat{\theta}$ методом максимального
правдоподобия.
\item{} Проверьте свойства несмещенности, состоятельности,
эффективности у построенной оценки. $[$каждое свойство по $2$, если дано аккуратное определение, то $1]$
\item{} $[1]$ Какая оценка более предпочтительна: построенная или
привычная
$\hat{\sigma}^{2}=\frac{\sum(X_{i}-\bar{X})^{2}}{n-1}$?
\end{enumerate}

% числа выверены
\item Имеются две конкурирующие гипотезы:
\begin{enumerate}
\item[$H_0$:] Случайная величина X распределена равномерно на (0,100)
\item[$H_a$:] Случайная величина X распределена равномерно на (50,150)
\end{enumerate}
Исследователь выбрал следующий критерий: если $X<c$, принимать гипотезу $H_0$, иначе  $H_a$.
\begin{enumerate}
\item Дайте определение ошибок первого и второго рода. $[2+2]$
\item Постройте графики зависимостей ошибок первого и второго рода от $c$. $[3+3]$
\end{enumerate}

%числа выверены
\item Вася измерил длину 10 пойманных им рыб. Часть рыб была поймана на
левом берегу реки, а часть - на правом. Бывалые рыбаки говорят,
что на правом берегу реки рыба крупнее.

\begin{tabular}{|c|c|c|c|c|c|}
  \hline
  Левый берег & 25 & 45 & 37 & 47 & 51   \\
  \hline
  Правый берег & 49 & 28 & 39 & 46 & 57   \\
  \hline
\end{tabular}
\begin{enumerate}
\item{} $[10]$ С помощью теста Манна-Уитни (Mann-Whitney) проверьте
гипотезу о том, что выбор берега реки не влияет на среднюю длину
рыбы против
альтернативной гипотезы, что на правом берегу рыба длиннее.

\emph{Разрешается использование нормальной аппроксимации}
\item{} $[$Не оценивался$]$ Возможно ли в этой задаче использовать
(Wilcoxon Signed Rank Test)?
\end{enumerate}
\end{enumerate}

\subsubsection*{Часть III.}

Стоимость задачи 20 баллов.

Требуется решить \textbf{\underbar{одну}} из двух 11-х задач по
выбору!

\begin{enumerate}
\item[11-A.] Имеются две монетки. Одна правильная, другая - выпадает орлом с
вероятностью $0.45$. Одну из них, неизвестно какую, подкинули $n$
раз и сообщили Вам, сколько раз выпал орел. Ваша задача проверить
гипотезу $H_{0}$: «подбрасывалась правильная монетка» против
$H_{a}$:
«подбрасывалась неправильная монетка». \\
Каким должно быть наименьшее $n$ и критерий выбора гипотезы, чтобы
вероятность ошибок первого рода не превышала 10\%, а вероятность
ошибки второго рода не превышала 15\%?

\item[11-B.] Время горения лампочки – экспоненциальная случайная величина с
математическим ожиданием равным $\theta $. Вася включил
одновременно 20 лампочек. Величина  $Y$ обозначает время самого
первого перегорания.
\begin{enumerate}
\item{} $[8]$ Найдите $\E(Y)$
\item{} $[6]$ Постройте с помощью  $Y$ несмещенную оценку для  $\theta$
\item{} $[6]$ Сравните по эффективности оценку построенную в пункте
«б» и
обычное выборочное среднее
\end{enumerate}
\end{enumerate}


\section{2007-2008}

\input{chapters/year_2007_2008.tex}

\section{2008-2009}

\input{chapters/year_2008_2009.tex}

\section{2009-2010}

\input{chapters/year_2009_2010.tex}

\section{2010-2011}

\input{chapters/year_2010_2011.tex}

\section{2011-2012}

\input{chapters/year_2011_2012.tex}

\section{2012-2013}

\input{chapters/year_2012_2013.tex}

\section{2013-2014}

\input{chapters/year_2013_2014.tex}

\section{2014-2015}

\input{chapters/year_2014_2015.tex}

\section{2015-2016}

\input{chapters/year_2015_2016.tex}

\section{2016-2017}

\input{chapters/year_2016_2017.tex}

\section{2017-2018}

\subsection{Теоретический минимум к кр 1}


\begin{enumerate}
	\item Классическое определение вероятности
	\item Определение условной вероятности
	\item Определение независимости случайных событий
	\item Формула полной вероятности
	\item Формула Байеса
	\item Функция распределения случайной величины. Определение и свойства.
	\item Функция плотности. Определение и свойства.
	\item Математическое ожидание. Определения для дискретного и абсолютно непрерывного случаев. Свойства.
	\item Дисперсия. Определение и свойства.
	\item Законы распределений. Определение, $\E(X)$, $\Var(X)$:
	\begin{enumerate}
	\item Биномиальное распределение
	\item Распределение Пуассона
	\item Геометрическое распределение
	\item Равномерное распределение
	\item Экспоненциальное распределение
	\end{enumerate}
\end{enumerate}


\subsection{Задачный минимум к кр 1}

\begin{enumerate}
\item  Пусть $\P(A) = 0.3, \P(B) = 0.4, \P(A\cap B) = 0.1 $. Найдите
	\begin{enumerate}
		\item  $\P(A|B)$
		\item  $\P(A\cup B)$
		\item  Являются ли события $A$ и $B$ независимыми?
	\end{enumerate}



\item  Пусть $\P(A) = 0.5, \P(B) = 0.5, \P(A\cap B) = 0.25 $. Найдите
\begin{enumerate}
	\item  $\P(A|B)$
	\item  $\P(A\cup B)$
	\item  Являются ли события $A$ и $B$ независимыми?
\end{enumerate}



\item  Карлсон выложил кубиками слово КОМБИНАТОРИКА. Малыш выбирает наугад четыре кубика и выкладывает их в случайном порядке.
Найдите вероятность того, что при этом получится слово КОРТ.


\item  Карлсон выложил кубиками слово КОМБИНАТОРИКА. Малыш выбирает наугад четыре кубика и выкладывает их в случайном порядке.
Найдите вероятность того, что при этом получится слово РОТА.

\item  В первой урне 7 белых и 3 черных шара, во второй урне 8 белых и 4 черных
шара, в третьей урне 2 белых и 13 черных шаров. Из этих урн наугад выбирается одна урна. Какова вероятность того, что шар, взятый наугад из выбранной урны, окажется белым?


\item  В первой урне 7 белых и 3 черных шара, во второй урне 8 белых и 4 черных
шара, в третьей урне 2 белых и 13 черных шаров. Из этих урн наугад выбирается одна урна. Какова вероятность того, что была выбрана первая урна, если шар, взятый наугад из выбранной урны, оказался белым?


\item  В операционном отделе банка работает 80\% опытных сотрудников и 20\%
неопытных. Вероятность совершения ошибки при очередной банковской операции
опытным сотрудником равна 0.01, а неопытным — 0.1. Найдите вероятность совершения ошибки при очередной банковской операции в этом отделе.


\item  В операционном отделе банка работает 80\% опытных сотрудников и 20\%
неопытных. Вероятность совершения ошибки при очередной банковской операции
опытным сотрудником равна 0.01, а неопытным — 0.1. Известно, что при очередной банковской операции была допущена ошибка. Найдите вероятность того, что ошибку допустил неопытный сотрудник.

\item  Пусть случайная величина $X$ имеет таблицу распределения:

\begin{tabular}{ ll l l}
	\toprule
	$X$ & -1  & 0  & 1 \\
	$\P_X$ & 0.25  & c  & 0.25 \\
  \bottomrule
\end{tabular}

Найдите
	\begin{enumerate}
	\item константу $c$
	\item $\P(\{X \geq 0\})$
	\item $\P(\{X < -3\}])$
	\item $\P(\{X \in [-\frac{1}{2}; \frac{1}{2}]\})$
	\item функцию распределения случайной величины $X$
	\item имеет ли случайная величина $X$ плотность распределения?
	\end{enumerate}


\item  Пусть случайная величина $X$ имеет таблицу распределения:

\begin{tabular}{ llll}
\toprule
$X$ & -1  & 0  & 1 \\
$\P_X$ & 0.25  & c  & 0.25 \\
\bottomrule
\end{tabular}

Найдите
\begin{enumerate}
	\item константу $c$
	\item $\E(X)$
	\item $\E(X^2)$
	\item $\Var(X)$
	\item $\E(|X|)$
\end{enumerate}

\item  Пусть случайная величина $X$ имеет таблицу распределения:

\begin{tabular}{ lll l}
\toprule
$X$ & -1  & 0  & 1 \\
$\P_X$ & 0.25  & c  & 0.5 \\
\bottomrule
\end{tabular}

Найдите
	\begin{enumerate}
	\item константу $c$
	\item $\P(\{X \geq 0\})$
	\item $\P(\{X < -3\}])$
	\item $\P(\{X \in [-\frac{1}{2}; \frac{1}{2}]\})$
	\item функцию распределения случайной величины $X$
	\item имеет ли случайная величина $X$ плотность распределения?
\end{enumerate}

\item  Пусть случайная величина $X$ имеет таблицу распределения:

\begin{tabular}{ l l l l}
  \toprule
$X$ & -1  & 0  & 1 \\
$\P_X$ & 0.25  & c  & 0.5 \\
\bottomrule
\end{tabular}

Найдите
\begin{enumerate}
	\item константу $c$
	\item $\E(X)$
	\item $\E(X^2)$
	\item $\Var(X)$
	\item $\E(|X|)$
\end{enumerate}

\item Пусть случайная величина $X$ имеет биномиальное распределение с
параметрами $n = 4$ и $\P = \frac{3}{4}$.
 Найдите
\begin{enumerate}
	\item $\P(\{X = 0\})$
	\item $\P(\{X > 0\})$
	\item $\P(\{X < 0\})$
	\item $\E(X)$
	\item $\Var(X)$
	\item  наиболее вероятное значение, которое принимает случайная величина $X$
\end{enumerate}

\item Пусть случайная величина $X$ имеет биномиальное распределение с
параметрами $n = 5$ и $\P = \frac{2}{5}$.
Найдите
\begin{enumerate}
	\item $\P(\{X = 0\})$
	\item $\P(\{X > 0\})$
	\item $\P(\{X < 0\})$
	\item $\E(X)$
	\item $\Var(X)$
	\item  наиболее вероятное значение, которое принимает случайная величина $X$
\end{enumerate}


\item  Пусть случайная величина X имеет распределение Пуассона с параметром $\lambda = 100$ . Найдите
\begin{enumerate}
	\item $\P(\{X = 0\})$
	\item $\P(\{X > 0\})$
	\item $\P(\{X < 0\})$
	\item $\E(X)$
	\item $\Var(X)$
	\item  наиболее вероятное значение, которое принимает случайная величина $X$
\end{enumerate}


\item  Пусть случайная величина X имеет распределение Пуассона с параметром $\lambda = 101$ . Найдите
\begin{enumerate}
	\item $\P(\{X = 0\})$
	\item $\P(\{X > 0\})$
	\item $\P(\{X < 0\})$
	\item $\E(X)$
	\item $\Var(X)$
	\item  наиболее вероятное значение, которое принимает случайная величина $X$
\end{enumerate}


\item В лифт 10-этажного дома на первом этаже вошли 5 человек. Вычислите
вероятность того, что на 6-м этаже выйдет хотя бы один человек.


\item В лифт 10-этажного дома на первом этаже вошли 5 человек. Вычислите
вероятность того, что на 6-м этаже не выйдет ни один человек.


\item При работе некоторого устройства время от времени возникают сбои.
Количество сбоев за сутки имеет распределение Пуассона. Среднее количество сбоев за сутки равно 3. Найти вероятность того, что в течение суток произойдет хотя бы один сбой.


\item При работе некоторого устройства время от времени возникают сбои.
Количество сбоев за сутки имеет распределение Пуассона. Среднее количество сбоев за сутки равно 3. Найти вероятность того, что за двое суток не произойдет ни одного сбоя.


\item Пусть случайная величина $X$ имеет плотность распределения

\[
f_X(x) =
	\begin{cases}
	c,\text{ при }  x \in [-1; 1] \\
	0,\text{ при } x \notin  [-1; 1] \\
	\end{cases}
\]

Найдите
\begin{enumerate}
	\item константу $c$
	\item $\P(\{X \leq 0\})$
	\item $\P(\{X \in [\frac{1}{2}; \frac{3}{2}]\})$
	\item $\P(\{X \in [2;3]\}$
	\item $F_X(x)$
\end{enumerate}


\item Пусть случайная величина $X$ имеет плотность распределения

\[
f_X(x) =
	\begin{cases}
	c,\text{ при }  x \in [-1; 1] \\
	0,\text{ при } x \notin  [-1; 1] \\
	\end{cases}
\]

Найдите
\begin{enumerate}
	\item константу $c$
	\item $\E(X)$
	\item $\E(X^2)$
	\item $\Var(X)$
	\item $\E(|X|)$
\end{enumerate}


\item Пусть случайная величина $X$ имеет плотность распределения

\[
f_X(x) =
	\begin{cases}
	cx,\text{ при }  x \in [0; 1] \\
	0,\text{ при } x \notin  [0; 1] \\
	\end{cases}
\]

Найдите
\begin{enumerate}
	\item константу $c$
	\item $\P(\{X \leq \frac{1}{2}\})$
	\item $\P(\{X \in [\frac{1}{2}; \frac{3}{2}]\})$
	\item $\P(\{X \in [2;3]\}$
	\item $F_X(x)$
\end{enumerate}


\item Пусть случайная величина $X$ имеет плотность распределения

\[
f_X(x) =
	\begin{cases}
	cx,\text{ при }  x \in [0; 1] \\
	0,\text{ при } x \notin  [0; 1] \\
	\end{cases}
\]

Найдите
\begin{enumerate}
	\item константу $c$
	\item $\E(X)$
	\item $\E(X^2)$
	\item $\Var(X)$
	\item $\E(\sqrt{X})$
\end{enumerate}
\end{enumerate}



\subsubsection*{Ответы}

\begin{enumerate}
	\item
			\begin{enumerate}
				\item 0.25
				\item 0.6
				\item нет
			\end{enumerate}
	\item
			\begin{enumerate}
				\item 0.5
				\item  0.75
				\item нет
			\end{enumerate}
	\item $\frac{4}{10 \cdot 11 \cdot 12 \cdot 13}$
	\item $\frac{4}{10 \cdot 11 \cdot 12 \cdot 13}$
	\item 0.5


	\item 0.42
	\item 0.028
	\item $\frac{5}{7}$
	\item
			\begin{enumerate}
				\item 0.5
				\item 0.75
				\item 0
				\item 0.5
			\end{enumerate}
	\item
			\begin{enumerate}
				\item 0.5
				\item  0
				\item  0.5
				\item  0.5
				\item  0.5
			\end{enumerate}
	\item
			\begin{enumerate}
				\item 0.25
				\item 0.75
				\item 0
				\item 0.5
			\end{enumerate}
	\item
			\begin{enumerate}
				\item 0.25
				\item 0.25
				\item 0.75
				\item 0.5
				\item 0.75
			\end{enumerate}
	\item
			\begin{enumerate}
				\item $\left( \frac{1}{4} \right) ^4$
				\item $1 - \left( \frac{1}{4} \right) ^4$
				\item 0
				\item 3
				\item 0.75
				\item 2, 3
			\end{enumerate}
	\item
			\begin{enumerate}
				\item $\left( \frac{3}{5} \right) ^5$
				\item $1 - \left( \frac{3}{5} \right) ^5$
				\item 0
				\item 2
				\item 1.2
				\item 2
			\end{enumerate}
	\item
			\begin{enumerate}
				\item $e^{-100}$
				\item $1 - e^{-100}$
				\item 0
				\item 100
				\item 100
			\end{enumerate}
	\item
			\begin{enumerate}
				\item $e^{-101}$
				\item $1 - e^{-101}$
				\item 0
				\item 101
				\item 101
			\end{enumerate}
	\item $1 - \frac{8^5}{9^5}$
	\item $\frac{8^5}{9^5}$
	\item $1 - e^{-3}$
	\item $e^{-3}$
	\item
			\begin{enumerate}
				\item 0.5
				\item 0.25
				\item 0.125
				\item 1
			\end{enumerate}
	\item
			\begin{enumerate}
				\item 0.5
				\item 0.5
				\item $\frac{1}{3}$
				\item $\frac{1}{12}$
				\item 1
			\end{enumerate}
	\item
			\begin{enumerate}
				\item 2
				\item 0.25
				\item $\frac{3}{4}$
				\item 1
			\end{enumerate}
	\item
			\begin{enumerate}
				\item 2
				\item 0.5
				\item 0.5
				\item 0
				\item 0.8
			\end{enumerate}
\end{enumerate}




\subsection{Контрольная работа 1, базовый поток, 24.10.2017}

\subsubsection*{Минимум}

\begin{enumerate}
\item Функция распределения случайной величины: определения и свойства.
\item Экспоненциальное распределение: определение, математическое ожидание и дисперсия.
\item В операционном отделе банка работает 80\% опытных сотрудников и 20\% неопытных. Вероятность совершения ошибки при очередной банковской операции опытным сотрудником равна $0.01$, а неопытным — $0.1$. Известно, что при очередной банковской операции была допущена ошибка. Найдите вероятность того, что ошибку допустил неопытный сотрудник.
\item При работе некоторого устройства время от времени возникают сбои. Количество сбоев за сутки имеет распределение Пуассона. Среднее количество сбоев за сутки равно 3. Найдите вероятность того, что за двое суток не произойдет ни одного сбоя.

\end{enumerate}

\subsubsection*{Задачи}

\begin{enumerate}

\item Правильный кубик подбрасывают один раз. Событие $A$ — выпало чётное число, событие $B$ — выпало число кратное трём, событие $C$ — выпало число, большее трёх.

\begin{enumerate}
\item Сформулируйте определение независимости двух событий;
\item Определите, какие из пар событий $A$, $B$ и $C$ будут независимыми.
\end{enumerate}


\item Теоретический минимум (ТМ) состоит из 10 вопросов, задачный (ЗМ) — из 24 задач.
Каждый вариант контрольной содержит два вопроса из ТМ и две задачи из ЗМ.
Чтобы получить за контрольную работу оценку 4 и выше, необходимо и достаточно правильно ответить на каждый вопрос ТМ и задачу ЗМ доставшегося варианта. Студент Вася принципиально выучил только $k$ вопросов ТМ и две трети ЗМ.
\begin{enumerate}
\item Сколько всего можно составить вариантов, отличающихся хотя бы одним заданием в ТМ или ЗМ части? Порядок заданий внутри варианта не важен.
\item Найдите вероятность того, что Вася правильно решит задачи ЗМ;
\item Дополнительно известно, что Васина вероятность правильно ответить на вопросы ТМ, составляет $1/15$. Сколько вопросов ТМ выучил Вася?
\end{enumerate}

\item Производитель молочных продуктов выпустил новый низкокалорийный йогурт Fit и утверждает, что он вкуснее его более калорийного аналога Fat.
Четырем независимым экспертам предлагают выбрать наиболее вкусный йогурт из трёх, предлагая им в одинаковых стаканчиках в случайном порядке два Fat и один Fit.
Предположим, что йогурты одинаково привлекательны.
Величина $\xi$ — число экспертов, отдавших предпочтение Fit.
\begin{enumerate}
\item Какова вероятность, что большинство экспертов выберут Fit?
\item Постройте функцию распределения величины $\xi$;
\item Каково наиболее вероятное число экспертов, отдавших предпочтение йогорту Fit?
\item Вычислите математическое ожидание и дисперсию $\xi$.
\end{enumerate}

\item Дядя Фёдор каждую субботу закупает в магазине продукты по списку, составленному котом Матроскином. Список не изменяется, и в него всегда входит 1 кг сметаны, цена которого является равномерно распределённой величиной $\alpha$, принимающей значения от 250 до 1000 рублей. Стоимость остальных продуктов из списка в тысячах рублей является случайной величиной $\xi$ с функцией распределения

\[
F(x)=\begin{cases}
1-\exp(-x^2 ), \text{ если } x \geq 0 \\
0, \text{ иначе.}\\
\end{cases}
\]

\begin{enumerate}
\item Какую сумму должен выделить кот Матроскин дяде Фёдору, чтобы её достоверно хватало на покупку сметаны?
\item Какую сумму должен выделить кот Матроскин дяде Фёдору, чтобы Дядя Фёдор с вероятностью 0.9 мог оплатить продукты без сметаны?
\item Найдите математическое ожидание стоимости продуктов без сметаны;
\item Найдите математическое ожидание стоимости всего списка.
\item Какова вероятность того, что общие расходы будут в точности равны их математическому ожиданию?
\end{enumerate}

Подсказка: $\int_0^{\infty} \exp(-x^2) \, dx = \sqrt{\pi} / 2$.

\item Эксперт с помощью детектора лжи пытается определить, говорит ли подозреваемый правду. Если подозреваемый говорит правду, то эксперт ошибочно выявляет ложь с вероятностью 0.1. Если подозреваемый обманывает, то эксперт выявляет ложь с вероятностью 0.95.

В деле об одиночном нападении подозревают десять человек, один из которых виновен и будет лгать, остальные невиновны и говорят правду.

\begin{enumerate}
\item Какова вероятность того, что детектор покажет, что конкретный подозреваемый лжёт?
\item Какова вероятность того, что подозреваемый невиновен, если детектор показал, что он лжёт?
\item Какова вероятность того, что эксперт точно выявит преступника?
\item Какова вероятность того, что эксперт ошибочно выявит  преступника, то есть покажет, что лжёт невиновный, а все остальные говорят правду?
\end{enumerate}



\end{enumerate}


\subsection{Контрольная работа 1, базовый поток, 24.10.2017, решения}

\begin{enumerate}
\item
\begin{enumerate}
\item События называются независимыми, если  $ \P(A \cap B) = \P(A) \cdot \P(B)$
\item Запасёмся всеми нужными вероятностями:

$\P(A) = \frac{1}{2}$

$\P(B) = \frac{1}{3}$

$\P(C) = \frac{1}{2}$

$\P(A \cap C) = \frac{1}{3} $ — выпадет чётое число больше трёх

$\P(A \cap B)  = \frac{1}{6}$ — выпадет чётное число, кратное трём

$\P(A \cap C) = \frac{1}{6}$ — выпадет число, большее трёх и кратное трём

Теперь можно проверять независимость:

$\P(A \cap C) \neq \P(A) \cdot \P(C) \Rightarrow$  не являются независимыми

$ \P(A \cap B) = \P(A) \cdot \P(B) \Rightarrow$ являются независимыми

$ \P(B \cap C) = \P(B) \cdot \P(C) \Rightarrow$ являются независимыми

\end{enumerate}
\item
\begin{enumerate}
\item Количество возможных вариантов ТМ: $ C_{10}^2 $,  количество возможных вариантов ЗМ: $ C_{24}^2 $. Количество их возможных сочетаний: $ C_{10}^2 \cdot C_{24}^2$ , где $ C_n^k = \frac{n!}{k!(n-k)!}$.
\item По классическому определению вероятностей, предполагая исходы равновероятными, искомая вероятность равна $ \frac{C_{16}^2}{C_{24}^2} $
\item По тому же принципу:
\[
\frac{C_k^2}{C_{10}^2} = \frac{1}{15} \Rightarrow \frac{\frac{k!}{2!(k-2)!}}{\frac{10!}{2! \cdot 8!}} = \frac{1}{15} \Rightarrow \frac{(k-1)k}{2}\frac{ 2}{9 \cdot 10} = \frac{1}{15}
\]
Получаем квадратное уравнение вида $ k^2 - k - 6 = 0 $ с корнями $-2$ и $3$. Так как $k$ не может быть отрицательным, ответ $3$.
\end{enumerate}
\item
\begin{enumerate}
\item Если эксперт отдаёт предпочтение Fit, то это можно интерпретировать как «успех» в схеме Бернулли. Так как $\xi$ - количество успехов, $ k \in [0;4]$, $p = \frac{1}{3} $, то
\[
\P(\xi = k) = C_n^k(p)^k(1-p)^{n-k}
\]

Большинство означает, что либо три, либо четыре эксперта выбрали Fit.
\[
\P(\xi = 3) = C_4^3\left(\frac{1}{3}\right)^3 \left(\frac{2}{3}\right)^{1} = \frac{8}{81}
\]
\[
\P(\xi = 4) = C_4^4\left(\frac{1}{3}\right)^4 \left(\frac{2}{3}\right)^{0} = \frac{1}{81}
\]
\[
\P( \xi > 2) =  \frac{9}{81}
\]
\item Аналогично:

\[ \P(\xi = 0) = C_4^0\left(\frac{1}{3}\right)^0 \left(\frac{2}{3}\right)^{4} = \frac{16}{81}\]

\[ \P(\xi = 1) = C_4^1\left(\frac{1}{3}\right)^1 \left(\frac{2}{3}\right)^{3} = \frac{32}{81}\]

\[ \P(\xi = 2) = C_4^2\left(\frac{1}{3}\right)^2 \left(\frac{2}{3}\right)^{2} = \frac{24}{81}\]

\begin{figure}[h!]
    \noindent\centering{
    \includegraphics[width=80mm]{images/kr1_2017_3.png}
    }
    \caption{Функция распределения}
    \label{cdf_kr2017}
\end{figure}

\item Все вероятности посчитаны, видим, что наибольшая достигается при $\xi=1$.
\item $\E(X) = np = \frac{4}{3} $, $ \Var(X) = npq = \frac{8}{9}$
\end{enumerate}
\item
\begin{enumerate}
\item Так как указано, что цена сметаны распределена равномерно на отерзке $[250, 1000]$, максимальное значение цены — $1000$, это и есть необходимая сумма.
\item Вспомним, что функция распределения $F(x) = \P(X \leq x)$, нужно найти такой $x$, что $ \P(X \leq x)=0.9$:
\[
0.9 = 1 - \exp({-x^{2}}) \Rightarrow \exp(-x^{2}) = 0.1 \Rightarrow -x^2 = \ln(0.1)  \Rightarrow x=  \sqrt{-\ln(0.1)}
\]
\item Взяв производную от функции распределения списка без сметаны, получим функцию плотности:
\[
f_X(x) =
\begin{cases}
2x\exp(-x^2) & x \ge 0 \\
0 & \text{иначе}
\end{cases}
\]
Найдём математическое ожидание:
\[
\int_{0}^{+\infty}2x^2\exp({-x^2}) dx = -x \exp({-x^2})\big|_0^{+\infty} + \int_{0}^{+\infty}\exp({-x^2}) dx = \frac{\sqrt{\pi}}{2}
\]
\item Математическое ожидание суммы случайных величин равно сумме математических ожиданий случайных влечин, если они существуют. Математическое ожидание от цены сметаны равно: $ \frac{1000 + 250}{2} = 625 $
Математическое ожидание списка без сметаны было найдено в предыдущем пункте, его осталось перевести в рубли. Получаем ответ: $ 625 + \frac{\sqrt{\pi}}{2} \cdot 1000 $.
\item Так как обе величины имеют абсолютно непрерывные распределения, вероятность попасть в конкретную точку равна нулю.
\end{enumerate}
\item
\begin{enumerate}
\item $\P(\text{детектор показл ложь и подозреваемый лжёт}) = 0.9 \cdot 0.1 + 0.1 \cdot 0.95 = 0.185$
\item $\P(\text{невниовен}|\text{детектор показал ложь}) = \frac{0.9\cdot0.1}{0.185} = \frac{90}{185}$
\item $\P(\text{эксперт точно выявит преступника}) = (0.9)^9 \cdot 0.95$
\item $\P(\text{эксперт ошибочно выявит преступника}) = 9 \cdot 0.1 \cdot 0.9^8\cdot 0.05$
\end{enumerate}


%\item
%$\P(Ложь|Лжёт) = 0.95 $

%$\P(Ложь|Не лжёт) = 0.1 $

%$\P({Лжёт}) = \frac{1}{10} $

%\P(\text{Не лжёт}) = \dfrac{9}{10} $

%\begin{enumerate}
%	\item
%	По формуле полной вероятности: \[  \P(Ложь) = 0.95 \cdot 0.1 + 0.1 \cdot 0.9 = 0.185 \]
%	\item
%	По формуле Байеса: \[ \P(\text{Не лжёт}|Ложь) = \dfrac{0.1 \cdot 0.9}{0.185} = 0.486\]
%    \item
%    Событие "эксперт точно выявит преступника" соответствует событию "эксперт выявит, что лгун лжет, и      что остальные говорят правду".  Таким образом:  \[ \P = 0.1 \cdot 0.95 + 0.9 \cdot 0.9  = 0.905\]
%    \item
%    Это означает, что эксперт выберет 8 человек из 9 невиновных и скажет, что они говорят правду
%    (количество вариантов выбрать так людей  $С_9^8$ ), а также выберет 1 виновного и скажет, что он
%    говорит правду (количество вариантов это сделать 1), и выберет одного невиновного и скажет, что он
%    лжет (количество вариантов выбрать так человека $С_9^1$ ). Просуммируем все с учетом вероятностей,
%    указанных в условии: \[ \P = с_9^8 \cdot 0.9 \cdot 0.9 \cdot C_9^1 \cdot 0.9 \cdot 0.1 \cdot 1 \cdot 0.1
%    \cdot 0.05 = 0. 26244\]

%\end{enumerate}
\end{enumerate}



\subsection{Контрольная работа 1, ИП, 24.10.2017}


Ровно 272 года назад императрица Елизавета повелела завезти во дворцы котов для ловли мышей.


\begin{enumerate}

\item В отсутствии кота Леопольда мыши Белый и Серый подкидывают по очереди игральный додекаэдр
%\footnote{Леопольд подсказывает по случаю праздника, что у додекаэдра 12 граней :)}
.
Сыр достаётся тому, кто первым выкинет число 6. Начинает подкидывать Белый.

\begin{enumerate}
  \item Какова вероятность того, что сыр достанется Белому?
  \item Сколько в среднем бросков продолжается игра?
  \item Какова дисперсия числа бросков?
\end{enumerate}

\item Микки Маус, Белый и Серый решили устроить труэль из любви к мышки Мии. Сначала стреляет Микки, затем Белый, затем Серый, затем снова Микки и так до тех пор, пока в живых не останется только один.

Прошлые данные говорят о том, что Микки попадает с вероятностью $1/3$, Белый — с вероятностью $2/3$, а Серый стреляет без промаха.

Найдите оптимальную стратегию каждого мыша.

\item Микки Маус, Белый и Серый пойманый злобным котом Леопольдом до начала труэли. И теперь Леопольд будет играть с ними в странную игру.

В комнате три закрытых внешне неотличимых коробки: с золотом, серебром и платиной. Общаться после начала игры мыши не могут, но могут заранее договориться о стратегии.

Правила игры таковы. Кот Леопольд будет заводить мышей в комнату по очереди. Каждый из мышей может открыть
две коробки по своему выбору. Перед следующим мышом коробки закрываются.

Если Микки откроет коробку с золотом, Белый
— с серебром, а Серый — с платиной, то они выигрывают. Если
хотя бы один из мышей не найдёт свой металл, то Леопольд их съест.
\begin{enumerate}
\item Какова оптимальная стратегия?
\item Какова вероятность выигрыша при использовании оптимальной стратегии?
\end{enumerate}

\item Накануне войны Жестокий Тиран Мышь очень большой страны издал указ. Отныне за каждого новорождённого мыша-мальчика семья получает денежную премию, но если в семье рождается вторая мышка-девочка, то всю семью убивают. Бедные жители страны запуганы и остро нуждаются в деньгах, поэтому в каждой семье мыши будут появляться до тех пор, пока не родится первая мышка-девочка.

\begin{enumerate}
  \item Каким будет среднее число детей в мышиной семье?
  \item Какой будет доля мышей-мальчиков в стране?
  \item Какой будет средняя доля мышей-мальчиков в случайной семье?
  \item Сколько в среднем мышей-мальчиков в случайно выбираемой семье?
\end{enumerate}

\item Вальяжный кот Василий положил на счёт в банке на Гаити один гурд. Сумма на счету растёт непрерывно с постоянной ставкой в течение очень длительного промежутка времени. В случайный момент этого промежутка кот Василий закрывает свой вклад.

Каков закон распределения первой цифры полученной Василием суммы?

\begin{comment}
\item Начинающий трейдер Афанасий совершает не более одной сделки в день.

Если в какой-то день у трейдера Афанасия есть акция, то за этот день он равновероятно продаёт или не продаёт её. Если в какой-то день у трейдера Афанасия нет акций, то он равновероятно покупает или не покупает одну акцию.

Найдите ожидаемую прибыль Афанасия, если известно, что реализовывал он свою стратегию 100 дней, в начале акции стоили по 50 рублей, в конце — по 80 рублей, максимум составил 120 рублей, а минимум — 30.


\item Страшный Мейн-кун разрубает палочку единичной длины на $10$ частей в случайных и независимых местах, равномерно распределённых по всех длине. Затем Страшный Мейн-кун выбирает случайно один из кусочков и возводит его длину в $24$ степень.

Какое в среднем число он получит?
\end{comment}



\subsection{Контрольная работа 1, ИП, 24.10.2017, решения}


\begin{enumerate}

\item[1.]

\begin{enumerate}
	\item Обозначим вероятность того, что сыр достанется Белому за $b$, если игра начинается с его броска. Получаем уравнение
\[
	b = \frac{1}{12} + \frac{11}{12} \frac{11}{12} b
\]

Пояснение: Как Белый может победить в исходной игре? Либо сразу выкинуть 6 с вероятностью $1/12$. Либо передать ход Серому ($11/12$), получить ход снова ($11/12$) и выиграть в продолжении игры. Продолжение игры по сути совпадает с исходной игрой.

\item Игра продолжается до тех пор, пока кто-то не выкинет «6». Для нахождения среднего количества бросков воспользуемся методом первого шага.

Обозначим среднее количество бросков нашей игры за $S$. Когда Белый бросает кубик, с вероятностью $\frac{1}{12}$ игра закончится за один бросок, а с вероятностью $\frac{11}{12}$ игра продолжится и ход перейдёт к Серому. Но та игра, которая начнётся, когда бросать будет Серый, ничем не отличается от предыдущей, поэтому среднее количество бросков в ней будет равно $S$. Однако мы попадём в эту игру, «потратив» один бросок. Таким образом мы получаем:

\[
S = \frac{1}{12} \cdot 1 + \frac{11}{12}(S +1)
\]

Получается, что $S = 12$, значит игра длится в среднем 12 бросков.
\end{enumerate}

\item[3.]

Для того, чтобы выжить, мышам нужно ещё до начала игры договориться о стратегии, которая позволит им с наибольшей вероятностью открыть нужные сундуки. Если хотя бы две мыши выберут одинаковый сундук, то их в любом случае съедят. Поэтому одной из оптимальных стратегий будет ещё до начала игры мышам договориться и назвать левый сундук золотым, сундук посередине серебряным, а правый — платиновым. Каждый мышонок должен открыть тот сундук, в честь которого назван необходимый ему металл. Если внутри он обнаруживает свой металл, то он выбирает этот сундук, если внутри находится не тот металл, мышонок открывает тот сундук, на который указывает лежащий внутри предмет.

Например, первым заходит Микки Маус. Он открывает золотой (левый) ящик. Если внутри лежит золото, то он выходит из комнаты. Если же внутри лежит, например, серебро, то Микки Маус открывает сундук посередине. Путём несложного перебора можно посчитать, что в 4 случаях из 6 мыши смогут найти нужный металл, поэтому вероятность выигрыша при данной стратегии равна $\frac{2}{3}$.

\item[5.]

Функция распределения дохода кота Василия, положившего один гурд на вклад, представляется в виде $m_t = 1\cdot e^{rt}$, где $r$ — процентная ставка, а $t$ — прошедшее время. Момент закрытия вклада Т равномерно распределён на отрезке от 0 до $a$, который очень велик, поэтому сумма, которую получит Василий, представима в виде $Z = e^{Y}$, где $Y \sim v[0; ra]$.

Вероятность того, что первая цифра будет равна 1, равна вероятности того, что доход Василия будет лежать в пределах от 1 до 2 гурдов, плюс вероятность того, что он лежит в пределах от 10 до 20 гурдов и т.д. Таким образом, можно представить эту вероятность, как:
\[
\P(N=1) = \P(e^Y \in [1;2) ) + \P(e^Y \in [10; 20) ) + \ldots
\]

Это выражение можно преобразовать таким образом:
\[
\P(N=1) = \P(Y \in [\ln 1; \ln2) ) + \P(Y \in [\ln 10; \ln 20) ) + \ldots
\]

Так как Y — равномерно распределённая величина, то $\P(Y \in [\ln 1; \ln2) ) = \frac{\ln 2 - \ln 1}{ra}$. Для последующих слагаемых вероятность рассчитывается таким же образом. Воспользовавшись свойством логарифма, можно заметить, что $\frac{\ln 20 - \ln 10}{ra} = \frac{\ln 2}{ra}$. Поэтому вероятность того, что на первом месте суммы вклада стоит единица, равна $n\cdot \frac{\ln 2}{ra}$, где $n$ -- количество слагаемых. Путём аналогичных рассуждений получаем, что вероятность того, что на первом месте стоит двойка, равна $n\cdot \frac{\ln 3- \ln 2}{ra}$. Из-за того, что $a$ велико, можно считать, что число слагаемых одинаково.

Т.к. на первом месте обязательно будет находиться какая-то цифра, то сумма вероятностей будет равна 1. Получаем:
\[
\dfrac{n}{ra}(\ln \frac{2}{1} + \ln \frac{3}{2} + \ldots + \ln \frac{10}{9}) = 1
\]

Таким образом $\frac{n}{ra} = \frac{1}{\ln 10}$. Получается, что вероятность того, что на первом месте стоит единица, равна:
\[
\P (N=1) = \dfrac{\ln 2}{\ln 10}
\]

Закон распределения первой цифры выводится сложением соответствующих вероятностей.

\end{enumerate}
\end{enumerate}



\subsection{Теоретический минимум к кр2}

\begin{enumerate}
\item Сформулируйте определение независимости событий, формулу полной вероятности.
\item Приведите определение условной вероятности случайного события, формулу Байеса.
\item Сформулируйте определение и свойства функции распределения случайной величины.
\item Сформулируйте определение и свойства функции плотности случайной величины.
\item Сформулируйте определение и свойства математического ожидания для абсолютно непрерывной случайной величины.
\item Сформулируйте определение и свойства математического ожидания для дискретной случайной величины.
\item Сформулируйте определение и свойства дисперсии случайной величины.
\item Сформулируйте определения следующих законов распределений: биномиального, Пуассона, шеометрического, равномерного, экспоненциального, нормального. Укажите математическое ожидание и дисперсию.
\item Сформулируйте определение функции совместного распределения двух случайных величин, независимости случайных величин. Укажите, как связаны совместное распределение и частные распределения компонент случайного вектора.
\item	Сформулируйте определение и свойства совместной функции плотности двух случайных величин, сформулируйте определение независимости случайных величин.
\item Сформулируйте определение и свойства ковариации случайных величин.
\item Сформулируйте определение и свойства корреляции случайных величин.
\item Сформулируйте определение и свойства условной функции плотности.
\item Сформулируйте определение  условного математического ожидания $\E(Y|X=x)$ для совместного дискретного и совместного абсолютно непрерывного распределений.
\item Сформулируйте определение математического ожидания и ковариационной матрицы случайного вектора и их свойства.
\item Сформулируйте неравенство Чебышёва и неравенство Маркова.
\item Сформулируйте закон больших чисел в слабой форме.
\item Сформулируйте центральную предельную теорему.
\item Сформулируйте теорему Муавра—Лапласа.
\item Сформулируйте определение сходимости по вероятности для последовательности случайных величин.

\end{enumerate}


\subsection{Задачный минимум кр 2}

\begin{enumerate}

\item Пусть задана таблица совместного распределения случайных величин $X$ и $Y$.

\begin{center}\begin{tabular}{lccc}
\toprule
 $X$ \textbackslash $Y$    & $-1$  & $0$   & $1$   \\ \midrule
$-1$                 & $0.2$ & $0.1$ & $0.2$ \\
 $1$                 & $0.1$ & $0.3$ & $0.1$ \\ \bottomrule
\end{tabular}\end{center}


Найдите
\begin{enumerate}
\item $\P(X = -1)$
\item $\P(Y = -1)$
\item $\P(X = -1 \cap Y = -1 )$
\item Являются ли случайные величины $X$ и $Y$ независимыми?
\item $F_{X,Y}(-1,0)$
\item Таблицу распределения случайной величины $X$
\item Функцию $F_{X}(x)$ распределения случайной величины $X$.
\item Постройте график функции $F_{X}(x)$ распределения случайной величины $X$.
\end{enumerate}

\item Пусть задана таблица совместного распределения случайных величин $X$ и $Y$.


\begin{center}\begin{tabular}{lccc}
\toprule
 $X$ \textbackslash $Y$    & $-1$  & $0$  & $1$   \\ \midrule
$-1$                 & $0.2$ & $0.1$ & $0.2$ \\
 $1$                 & $0.2$ & $0.1$ & $0.2$ \\ \bottomrule
\end{tabular}\end{center}

Найдите
\begin{enumerate}
\item $\P(X = 1)$,
\item $\P(Y = 1)$,
\item $\P(X = 1 \cap Y = 1)$
\item Являются ли случайные величины $X$ и $Y$ независимыми?
\item $F_{X,Y}(1,0)$
\item Таблицу распределения случайной величины $Y$
\item Функцию $F_{Y}(y)$ распределения случайной величины $Y$
\item Постройте график функции $F_{Y}(y)$ распределения случайной величины $Y$.
\end{enumerate}

\item Пусть задана таблица совместного распределения случайных величин $X$ и $Y$.

\begin{center}\begin{tabular}{lccc}
\toprule
 $X$ \textbackslash $Y$    & $-1$  & $0$   & $1$   \\ \midrule
$-1$                 & $0.2$ & $0.1$ & $0.2$ \\
 $1$                 & $0.1$ & $0.3$ & $0.1$ \\ \bottomrule
\end{tabular}\end{center}

Найдите
\begin{enumerate}
    \item $\E(X),$
    \item $\E(X^{2}),$
	\item $\Var(X),$
    \item $\E(Y),$
    \item $\E(Y^{2}),$
    \item $\Var(Y),$
    \item $\E(XY),$
	\item $\Cov(X,Y)$
    \item $\Corr(X,Y)$
    \item Являются ли случайные величины $X$ и $Y$ некоррелированными?
\end{enumerate}

\item Пусть задана таблица совместного распределения случайных величин $X$ и $Y$.

\begin{center}\begin{tabular}{lccc}
\toprule
 $X$ \textbackslash $Y$    & $-1$  &$ 0 $  & $1 $  \\ \midrule
$-1$                 & $0.2$ & $0.1$ & $0.2$ \\
 $1$                 & $0.2$ & $0.1$ & $0.2$ \\ \bottomrule
\end{tabular}\end{center}

Найдите
\begin{enumerate}
    \item $\E(X),$
    \item $\E(X^{2}),$
	\item $\Var(X),$
    \item $\E(Y),$
    \item $\E(Y^{2}),$
    \item $\Var(Y),$
    \item $\E(XY),$
	\item $\Cov(X,Y)$
    \item $\Corr(X,Y)$
    \item Являются ли случайные величины $X$ и $Y$ некоррелированными?
\end{enumerate}

\item
Пусть задана таблица совместного распределения случайных величин $X$ и $Y$.

\begin{center}\begin{tabular}{lccc}
\toprule
 $X$\textbackslash $Y$    & $-1$  & $0$   & $1$   \\ \midrule
$-1$                 & $0.2$ & $0.1$ & $0.2$ \\
 $1$                 & $0.1$ & $0.3$ & $0.1$ \\ \bottomrule
\end{tabular}\end{center}

Найдите
\begin{enumerate}
\item $\P(X = -1 | Y = 0)$
\item $\P(Y = 0 | X = -1)$
\item таблицу условного распределения случайной величины $Y$ при условии $X = -1$
\item условное математическое ожидание случайной величины $Y$ при $X = -1$
\item условную дисперсию случайной величины $Y$
при условии $X = -1$
\end{enumerate}

\item Пусть задана таблица совместного распределения случайных величин $X$ и $Y$.

\begin{center}\begin{tabular}{lccc}
\toprule
 $X$ \textbackslash $Y$    & $-1$  & $0$   & $1$   \\ \midrule
$-1$                 & $0.2$ & $0.1$ & $0.2$ \\
 $1$                 & $0.2$ & $0.1$ & $0.2$ \\ \bottomrule
\end{tabular}\end{center}

Найдите
\begin{enumerate}
\item $\P(X = 1 | Y = 0)$
\item $\P(Y = 0 | X = 1)$
\item таблицу условного распределения случайной величины $Y$ при условии $X = 1$
\item условное математическое ожидание случайной величины $Y$ при $X = 1$
\item условную дисперсию случайной величины $Y$
при условии $X = 1$
\end{enumerate}

\item Пусть $\E(X)=1$, $\E(Y)=2$, $\Var(X) = 3$, $\Var(Y) = 4$, $\Cov(X,Y) = -1$. Найдите
\begin{enumerate}
\item $\E(2X + Y - 4)$
\item $\Var(3Y + 3)$
\item $\Var(X - Y)$
\item $\Var(2X - 3Y +1)$
\item $\Cov(X+ 2Y + 1,3X - Y -1)$
\item $\Corr(X + Y, X - Y)$
\item Ковариационную матрицу случайного вектора $Z = (X\hspace*{0.4cm} Y)$ \end{enumerate}


\item Пусть $\E(X)=-1$, $\E(Y)=2$, $\Var(X) = 1$, $\Var(Y) = 2$, $\Cov(X,Y) = 1$. Найдите
\begin{enumerate}
\item $\E(2X + Y - 4)$
\item $\Var(2Y + 3)$
\item $\Var(X - Y)$
\item $\Var(2X - 3Y +1)$
\item $\Cov(3X+ Y + 1,X - 2Y -1)$
\item $\Corr(X + Y, X - Y)$
\item Ковариационную матрицу случайного вектора $Z = (X\hspace*{0.4cm}Y)$
\end{enumerate}

\item Пусть случайная величина $X$ имеет стандартное нормальное распределение.

Найдите
\begin{enumerate}
\item $\P(0 < X < 1)$
\item $\P(X > 2)$
\item $\P(0 < 1 - 2X \leq 1)$
\end{enumerate}

\item Пусть случайная величина $X$ имеет стандартное нормальное распределение.

Найдите
\begin{enumerate}
\item $\P(-1 < X < 1)$
\item $\P(X < -2)$
\item $\P(-2 < -X + 1 \leq 0)$
\end{enumerate}

\item Пусть случайная величина $X \sim \cN(1,4)$. Найдите $\P(1<X<4)$

\item Пусть случайная величина $X \sim \cN(2,4)$. Найдите $\P(-2<X<4)$

\item Случайные величины $X$ и $Y$ независимы и  имеют нормальное распределение, $\E(X) = 0 $, $\Var(X) = 1$, $\E(Y) = 2$, $\Var(Y) = 6$. Найдите $\P(1 < X + 2Y < 7)$.

\item Случайные величины $X$ и $Y$ независимы и  имеют нормальное распределение, $\E(X) = 0 $, $\Var(X) = 1$, $\E(Y) = 3$, $\Var(Y) = 7$. Найдите $\P(1 < 3X + Y < 7)$.

\item Игральная кость подбрасывается $420$ раз. При помощи центральной предельной теоремы приближенно найти вероятность того, что суммарное число очков будет находиться в пределах от $1400$ до $1505$?

\item При выстреле по мишени стрелок попадает в десятку с вероятностью $0.5$, в девятку – $0.3$, в восьмерку – $0.1$, в семерку – $0.05$, в шестерку – $0.05$.
Стрелок сделал $100$ выстрелов. При помощи центральной предельной теоремы приближенно найти вероятность того, что он набрал не менее 900 очков?

\item Предположим, что на станцию скорой помощи поступают вызовы, число которых распределено по закону Пуассона с параметром $\lambda = 73$, и в разные сутки их количество не зависит друг от друга. При помощи центральной предельной теоремы приближенно найти вероятность того, что в течение года (365 дней) общее число вызовов будет в пределах от $26500$ до $26800$.

\item Число посетителей магазина (в день) имеет распределение Пуассона с математическим ожиданием $289$. При помощи центральной предельной теоремы приближенно найти вероятность того, что за $100$ рабочих дней суммарное число посетителей составит от $28550$ до $29250$ человек.

\item Пусть плотность распределения случайного вектора $(X,Y)$ имеет вид
\begin{center} $f_{X,Y}(x,y) = \begin{cases} x+y, & \text{при } (x,y) \in [0;1] \times [0;1] \\ 0 , & \text{при } (x,y) \not\in [0;1] \times [0;1] \end{cases}$  \end{center}

Найдите
\begin{enumerate}
\item $\P(X \leq \frac{1}{2} \cap Y \leq \frac{1}{2})$,
\item $\P(X\leq Y)$,
\item $f_{X}(x)$,
\item $f_{Y}(y)$,
\item Являются ли случайные величины $X$ и $Y$ независимыми?
\end{enumerate}

\item Пусть плотность распределения случайного вектора $(X,Y)$ имеет вид
\begin{center} $f_{X,Y}(x,y) = \begin{cases} 4xy, & \text{при } (x,y) \in [0;1] \times [0;1] \\ 0 , & \text{при } (x,y) \not\in [0;1] \times [0;1] \end{cases}$  \end{center}

Найдите
\begin{enumerate}
\item $\P(X \leq \frac{1}{2} \cap Y \leq \frac{1}{2})$,
\item $\P(X\leq Y)$,
\item $f_{X}(x)$,
\item $f_{Y}(y)$,
\item Являются ли случайные величины $X$ и $Y$ независимыми?
\end{enumerate}

\item Пусть плотность распределения случайного вектора $(X,Y)$ имеет вид
\begin{center} $f_{X,Y}(x,y) = \begin{cases} x+y, & \text{при } (x,y) \in [0;1] \times [0;1] \\ 0 , & \text{при } (x,y) \not\in [0;1] \times [0;1] \end{cases}$  \end{center}

Найдите
\begin{enumerate}
\item $\E(X)$,
\item $\E(Y)$,
\item $\E(XY)$,
\item $\Cov(X,Y)$,
\item $\Corr(X,Y)$.
\end{enumerate}

\item Пусть плотность распределения случайного вектора $(X,Y)$ имеет вид
\begin{center} $f_{X,Y}(x,y) = \begin{cases} 4xy, & \text{при } (x,y) \in [0;1] \times [0;1] \\ 0 , & \text{при } (x,y) \not\in [0;1] \times [0;1] \end{cases}$  \end{center}

Найдите
\begin{enumerate}
\item $\E(X)$,
\item $\E(Y)$,
\item $\E(XY)$,
\item $\Cov(X,Y)$,
\item $\Corr(X,Y)$.
\end{enumerate}

\item Пусть плотность распределения случайного вектора $(X,Y)$ имеет вид
\begin{center} $f_{X,Y}(x,y) = \begin{cases} x+y, & \text{при } (x,y) \in [0;1] \times [0;1] \\ 0 , & \text{при } (x,y) \not\in [0;1] \times [0;1] \end{cases}$  \end{center}

Найдите
\begin{enumerate}
\item $f_{Y}(y)$,
\item $f_{X|Y}\left(x|\frac{1}{2}\right)$
\item $\E\left(X|Y = \frac{1}{2}\right)$
\item $\Var\left(X|Y = \frac{1}{2}\right)$
\end{enumerate}

\item Пусть плотность распределения случайного вектора $(X,Y)$ имеет вид
\begin{center} $f_{X,Y}(x,y) = \begin{cases} 4xy, & \text{при } (x,y) \in [0;1] \times [0;1] \\ 0 , & \text{при } (x,y) \not\in [0;1] \times [0;1] \end{cases}$  \end{center}

Найдите
\begin{enumerate}
\item $f_{Y}(y)$,
\item $f_{X|Y}\left(x|\frac{1}{2}\right)$
\item $\E\left(X|Y = \frac{1}{2}\right)$
\item $\Var\left(X|Y = \frac{1}{2}\right)$
\end{enumerate}

\end{enumerate}

\subsection*{Ответы}

\begin{enumerate}

\item
\begin{enumerate}
\item   $0.5 $
\item   $0.3$
\item   $0.2$
\item   нет
\item   $0.3$
\item
\begin{tabular}{lrr}
\toprule
$X$ & $-1$  & $1$   \\ \midrule
$\P(\cdot)$ & $0.5$ & $0.5$ \\ \bottomrule
\end{tabular}
\item  $F_{X}(x) = \begin{cases}
0, & \text{при } x < -1 \\
0.5 , & \text{при } x \in [-1;1) \\
1, & \text{при }  x \geq 1
\end{cases}$
\end{enumerate}
\item
\begin{enumerate}
\item   $0.5$
\item   $0.4$
\item   $0.2$
\item   да
\item   $0.6$
\item
\begin{tabular}{lrrr}
\toprule
$Y$ & $-1$  & $0$   & $1$   \\ \midrule
$\P(\cdot)$ & $0.4$ & $0.2$ & $0.4$ \\ \bottomrule
\end{tabular}
\item   $F_{Y}(y) = \begin{cases}
0, & \text{при } y < -1 \\
0.4 , & \text{при } y \in [-1;0) \\
0.6, & \text{при }  y \in [0;1)\\
1, & \text{при } y \geq 1
\end{cases}$
\end{enumerate}

\item
\begin{enumerate}
\item   $0$
\item   $1$
\item  $1$
\item   $0$
\item   $0.6$
\item   $0.6$
\item   $0$
\item   $0$
\item   $0$
\item   да, являются некоррелированными, но нельзя утверждать, что являются независимыми
\end{enumerate}

\item
\begin{enumerate}
\item   $0$
\item   $1$
\item   $1$
\item   $0$
\item   $0.8$
\item   $0.8$
\item   $0$
\item   $0$
\item   $0$
\item   да, являются некоррелированными, но нельзя утверждать, что являются независимыми
\end{enumerate}

\item
\begin{enumerate}
\item   $0.25$
\item   $0.2$
\item   \begin{tabular}{lrrr}
\toprule
$Y$ | $\{X = -1\}$ & $-1$  & $0$   & $1$   \\ \midrule
$\P(\cdot)$              & $0.4$ & $0.2$ & $0.4$ \\ \bottomrule
\end{tabular}
\item   $0$
\item   $0.8 $
\end{enumerate}
\item
\begin{enumerate}
\item   $0.5$
\item   $0.2$
\item   \begin{tabular}{lrrr}
\toprule
$Y$ | $\{X = 1\}$ & $-1$  & $0$   & $1$   \\ \midrule
$\P(\cdot)$             & $0.4$ & $0.2$ & $0.4$ \\ \bottomrule
\end{tabular}
\item   $0$
\item   $0.8$
\end{enumerate}

\item
\begin{enumerate}
\item   $0 $
\item   $36$
\item  $9 $
\item   $60 $
\item  $-4$
\item   $\frac{-1}{3\sqrt{5}}$
\item  $\begin{pmatrix}
 3 & -1 \\
-1 & 4
\end{pmatrix}$
\end{enumerate}

\item
\begin{enumerate}
\item $-4$
\item $8 $
\item $1 $
\item $10 $
\item $-6$
\item$ \frac{-1}{\sqrt{5}}$

\item $\begin{pmatrix}
 1 & 1 \\
 1 & 2
\end{pmatrix}$
\end{enumerate}
\item
\begin{enumerate}
\item $0.3413$
\item $0.0228$
\item $0.1915$
\end{enumerate}

\item
\begin{enumerate}
\item $0.6826$
\item $0.0228  $
\item $0.1574  $
\end{enumerate}

\item $0.4332 $
\item $0.8185  $
\item $0.4514 $
\item $0.5328  $
\item $\approx 0.8185 $
\item $\approx 0.9115$
\item $\approx 0.6422 $
\item $\approx 0.9606$

\item
\begin{enumerate}
\item $0.125   $
\item $0.5 $
\item $f_{X}(x) = \begin{cases} x+\frac{1}{2}, & \text{при } x \in [0;1] \\ 0 , & \text{при } x \not\in [0;1] \end{cases}$
\item $f_{Y}(y) = \begin{cases} y+\frac{1}{2}, & \text{при } y \in [0;1] \\ 0 , & \text{при } y \not\in [0;1] \end{cases}$
\item нет
\end{enumerate}

\item
\begin{enumerate}
\item $\frac{1}{16}$

\item $\frac{1}{2}$

\item$f_{X}(x) = \begin{cases} 2x, & \text{при } x \in [0;1] \\ 0 , & \text{при } x \not\in [0;1] \end{cases}$

\item$f_{Y}(y) = \begin{cases} 2y, & \text{при } y \in [0;1] \\ 0 , & \text{при } y \not\in [0;1] \end{cases}$

\item да
\end{enumerate}

\item
\begin{enumerate}
\item $\frac{7}{12}$

\item $\frac{7}{12}$

\item $\frac{1}{3}$

\item $-\frac{1}{144}$

\item $-\frac{1}{11}$
\end{enumerate}

\item
\begin{enumerate}
\item $\frac{2}{3}$

\item $\frac{2}{3}$

\item $\frac{4}{9}$

\item 0

\item 0
\end{enumerate}

\item
\begin{enumerate}
\item $f_{Y}(y) = \begin{cases} y+\frac{1}{2}, & \text{при } y \in [0;1] \\ 0 , & \text{при } y \not\in [0;1] \end{cases}$

\item $f_{X|Y}(x|\frac{1}{2}) = \begin{cases} x+\frac{1}{2}, & \text{при } x \in [0;1] \\ 0 , & \text{при } x \not\in [0;1] \end{cases}$

\item $\frac{7}{12}$

\item $\frac{11}{144}$
\end{enumerate}

\item
\begin{enumerate}
\item $f_{Y}(y) = \begin{cases} 2y, & \text{при } y \in [0;1] \\ 0 , & \text{при } y \not\in [0;1] \end{cases}$

\item $f_{X|Y}(x|\frac{1}{2}) = \begin{cases} 2x, & \text{при } x \in [0;1] \\ 0 , & \text{при } x \not\in [0;1] \end{cases}$

\item $\frac{2}{3}$

\item $\frac{1}{18}$
\end{enumerate}
\end{enumerate}

\subsection{Контрольная работа 2, базовый поток, 09.12.2017}



\subsubsection{Минимум}
% 2 + 4 + 14 + 16

\begin{enumerate}
\item Приведите определение условной вероятности случайного события, формулу Байеса.
\item Сформулируйте определение и свойства функции плотности случайной величины. 
\item Сформулируйте определение  условного математического ожидания $\E(Y|X=x)$ для совместного дискретного и совместного абсолютно непрерывного распределений.
\item Сформулируйте неравенство Чебышёва и неравенство Маркова.

\item Задана таблица совместного распределения случайных величин $X$ и $Y$.
\begin{center}
\begin{tabular}{lccc}
\toprule
                       & $Y=-1$  & $Y=0$   & $Y=1$   \\ 
 \midrule
$X=0$                 & $0.2$ & $0.1$ & $0.3$ \\
 $X=1$                 & $0.2$ & $0.1$ & $0.1$ \\ 
 \bottomrule
\end{tabular}
\end{center}


\begin{enumerate}
    \item Найдите $F_{X,Y}(0, 0)$;
    \item Найдите $\E(X)$, $\E(X^2)$, $\E(Y)$, $\E(Y^2)$;
    \item Найдите $\Var(X)$, $\Var(Y)$;
    \item Найдите $\Cov(X, Y)$, $\Corr(X, Y)$
\end{enumerate}    
\item Плотность распределения случайного вектора $(X,Y)$ имеет вид
\[
f_{X,Y}(x,y) = 
\begin{cases} 
\frac{4x+10y}{7}, & \text{при } (x,y) \in [0;1] \times [0;1] \\ 
0 , & \text{при } (x,y) \not\in [0;1] \times [0;1] \\
\end{cases}
\]

\begin{enumerate}
\item Найдите $\P(X \leq Y)$;
\item Найдите функцию плотности $f_X(x)$;
\item Найдите $\E(X)$, $\E(Y)$ и $\Cov(X, Y)$;
\item Являются ли случайные величины $X$ и $Y$ независимыми?
\end{enumerate} 


\end{enumerate}

\subsubsection{Задачи}

\begin{enumerate}[resume]

\item Статистика авиакомпании «А» за много лет свидетельствует о том, что 10\% людей, купивших билет на самолет, не являются на рейс. Авиакомпания продала 330 билетов на 300 мест.
\begin{enumerate}
\item Какова вероятность, что всем явившимся на рейс пассажирам хватит места?
\item Укажите наибольшее число билетов, которое можно продавать на 300 мест, чтобы случаи переполнения случались не чаще, чем на одном из десяти рейсов.
\end{enumerate}

\item Сегодня акция компании «Ух» стоит 1 рубль. Каждый день акция может с вероятностью 0.7 вырасти на 1\%, с вероятностью 0.2999 упасть на 1\% и с вероятностью 0.0001 обесцениться (упасть на 100\%).
\begin{enumerate}
\item Считая изменение цены акции независимыми, найдите математическое ожидание её стоимости через 20 торговых дней.
\item Найдите предел по вероятности среднего изменения цены акции в процентах на бесконечном промежутке времени (Ответ обоснуйте).
\item Найдите математическое ожидание цены акции на бесконечном промежутке времени.
\item Инвестор вложил все свои средства в акции компании «Ух». Найдите вероятность его разорения на бесконечном промежутке времени.
\end{enumerate}


\end{enumerate}




\subsection{Контрольная работа 2, базовый поток, решения}

\begin{enumerate}
\item[7.]
\begin{enumerate}
\item Всем хватит места, если число явившихся на рейс пассажиров ($X$) не превысит $300$,
то есть нужно найти $\P(X \leq 300)$. Найдём матожидание и дисперсию
случайной величины $X$:
\begin{align*}
\E(X) &= np = 330 \cdot 0.9 = 297 \\
\Var(X) &= np(1-p) = 330 \cdot 0.9 \cdot 0.1 = 29.7
\end{align*}
Теперь посчитаем нужную вероятность:
\[
\P(X \leq 300) = \P \left(\frac{X - 297}{\sqrt{29.7}} \leq \frac{300 - 297}{\sqrt{29.7}} \right) = \P(\cN(0,1) \leq 0.55) \approx 0.709
\]
\item Вероятность переполнения не должна превышать $0.1$:
\begin{align*}
&\P(X > 300) < 0.1 \\
&\P\left(\frac{X - 0.9 \cdot n}{\sqrt{0.9 \cdot 0.1 \cdot n}} > \frac{300 - 0.9 \cdot n}{\sqrt{0.9 \cdot 0.1 \cdot n}} \right) < 0.1 \\
&\frac{300 - 0.9 \cdot n}{\sqrt{0.9 \cdot 0.1 \cdot n}}  > 1.28 \\
&300 - 0.9n > 1.28 \cdot 0.3 \sqrt{n} \\
&n < 325.6
\end{align*}
\end{enumerate}
\item[8.]
\begin{enumerate}
\item Выпишем случайную величину $X_i$ — цену акции после $i$-ого дня:
\[
X_i =
\begin{cases}
1.01, & p = 0.7 \\
0.99, & p = 0.2999 \\
0, & p = 0.0001
\end{cases}
\]
Нужно посчитать ожидание цены акциии после 20 дней:
\[
\E(X_1 \cdot \ldots \cdot X_{20}) \stackrel{\text{незав-ть}}{=} \E(X_1) \cdot \ldots \cdot \E(X_{20}) = 1.004^{20} \approx 1.083
\]
\item По ЗБЧ:
\[
\plim_{n\to\infty} \frac{1}{n} \sum_{i=1}^n X_i = \E(X_i) = 1.004
\]
\item Аналогично пункту (а):
\[
\E(X_1 \cdot \ldots \cdot X_{n}) = (\E(X_1))^n = 1.004^n
\]
И понятно, что $1.004^n \to_{n\to\infty} +\infty$.
\item
\begin{multline*}
\P(\text{разорения}) = 1 - \P(X_1 > 0, \ldots, X_n >0) = 1 - \prod_{i=1}^n \P(X_i > 0) \\
= 1 - (1 - 0.0001)^n \to_{n\to\infty} 1
\end{multline*}
\end{enumerate}
\end{enumerate}







\subsection{Теоретический минимум к кр3}

\begin{enumerate}
  \item Дайте определение нормально распределённой случайной величины. Укажите диапазон возможных значений, функцию плотности, ожидание, дисперсию. Нарисуйте функцию плотности.
  \item Дайте определение хи-квадрат распределения. Укажите диапазон возможных значений, выражение через нормальные распределения, математическое ожидание. Нарисуйте функцию плотности при разных степенях свободы.
  \item Дайте определение распределения Стьюдента. Укажите диапазон возможных значений, выражение через нормальные распределения. Нарисуйте функцию плотности распределения Стьюдента при разных степенях свободы на фоне нормальной стандартной функции плотности.
  \item Дайте определение распределения Фишера. Укажите диапазон возможных значений, выражение через нормальные распределенеия. Нарисуйте возможную функцию плотности.
\end{enumerate}

Для следующего блока вопросов предполагается, что
имеется случайная выборка $X_1$, $X_2$, \ldots, $X_n$ из распределения
с функцией плотности $f(x, \theta)$, зависящей от от параметра $\theta$. Дайте определение каждого понятия из списка или сформулируйте соответствующую теорему:

\begin{enumerate}[resume]
  \item Выборочное среднее и выборочная дисперсия;
  \item Формула несмещённой оценки дисперсии;
  \item Выборочный начальный момент порядка $k$;
  \item Выборочный центральный момент порядка $k$;
  \item Выборочная функция распределения;
  \item Несмещённая оценка $\hat \theta$ параметра $\theta$;
  \item Состоятельная последовательность оценок $\hat \theta_n$;
  \item Эффективность оценки $\hat \theta$ среди множества оценок $\hat \Theta$;
  \item Неравенство Крамера–Рао для несмещённых оценок;
  \item Функция правдоподобия и логарифмическая функция правдоподобия;
  \item Информация Фишера о параметре $\theta$, содержащаяся в одном наблюдении;
  \item Оценка метода моментов параметра $\theta$ при использовании первого момента, если $\E(X_i)=g(\theta)$ и существует обратная функция $g^{-1}$;
  \item Оценка метода максимального правдоподобия параметра $\theta$;
\end{enumerate}

Для следующего блока вопросов предполагается, что величины $X_1$, $X_2$, \ldots, $X_n$ независимы и нормальны $\cN(\mu;\sigma^2)$.

\begin{enumerate}[resume]
  \item Укажите закон распределения выборочного среднего, величины $\frac{\bar X - \mu}{\sigma/\sqrt{n}}$, величины $\frac{\bar X - \mu}{\hat\sigma/\sqrt{n}}$, величины $\frac{\hat\sigma^2(n-1)}{\sigma^2}$;
  \item Укажите формулу доверительного интервала с уровнем доверия $(1-\alpha)$ для $\mu$ при известной дисперсии, для $\mu$ при неизвестной дисперсии, для $\sigma^2$;
\end{enumerate}

\subsection{Задачный минимум к кр3}

\begin{enumerate}

\item Рост в сантиметрах (случайная величина $X$) и вес в килограммах (случайная величина $Y$) взрослого мужчины является нормальным случайным вектором $Z = (X, Y)$ с математическим ожиданием $\E(Z) = (175, 74)$ и ковариационной матрицей

\[
\Var(Z) =
\begin{pmatrix}
 49 & 28 \\
28 & 36
\end{pmatrix}
\]

Лишний вес характеризуется случайной величиной $U = X - Y$. Считается, что человек страдает избыточным весом, если $U < 90$.

\begin{enumerate}
\item Определите вероятность того, что рост мужчины отклоняется от среднего более, чем на $10$ см.
\item Укажите распределение случайной величины $U$. Выпишите её плотность распределения.
\item Найдите вероятность того, что случайно выбранный мужчина страдает избыточным весом.
\end{enumerate}

\item Рост в сантиметрах, случайная величина $X$, и вес в килограммах, случайная величина $Y$, взрослого мужчины является нормальным случайным вектором $Z = (X, Y)$ с математическим ожиданием $\E(Z) = (175, 74)$ и ковариационной матрицей

\[
\Var(Z) =
\begin{pmatrix}
 49 & 28 \\
28 & 36
\end{pmatrix}
\]

\begin{enumerate}
\item Найдите средний вес мужчины при условии, что его рост составляет $170$ см.
\item Выпишите условную плотность распределения веса мужчины при условии, что его рост составляет $170$ см.
\item Найдите условную вероятность того, что человек будет иметь вес, больший $90$ кг, при условии, что его рост составляет $170$ см.
\end{enumerate}

\item Для реализации случайной выборки $x=(1,0,-1,1)$ найдите:

\begin{enumerate}
\item выборочное среднее,
\item неисправленную выборочную дисперсию,
\item исправленную выборочную дисперсию,
\item выборочный второй начальный момент,
\item выборочный третий центральный момент,
\end{enumerate}

\item Для реализации случайной выборки $x=(1,0,-1,1)$ найдите:

\begin{enumerate}
\item вариационный ряд,
\item первый член вариационного ряда,
\item последний член вариационного ряда,
\item график выборочной функции распределения.
\end{enumerate}

\item Пусть $X=(X_1, \ldots,X_n)$ — случайная выборка из дискретного распределения, заданного с помощью таблицы

\begin{center}
\begin{tabular}{cccc}
\toprule
 $x$ & $-3$  &$ 0 $  & $2 $  \\
 \midrule
 $\P(X_i = x)$ & $2/3 - \theta$ & $1/3$ & $\theta$ \\
 \bottomrule
\end{tabular}
\end{center}

Рассмотрите оценку $\hat{\theta} = \dfrac{\bar{X}+2}{5}$.

\begin{enumerate}
    \item Найдите $\E[\hat{\theta}]$.
    \item Является ли оценка $\hat{\theta}$ несмещенной оценкой неизвестного параметра $\theta$?
\end{enumerate}

\item Пусть $X=(X_1, \ldots ,X_n)$ — случайная выборка из распределения с плотностью распределения

\[
f(x,\theta) = \begin{cases}
\dfrac{6x(\theta - x)}{\theta^3} & \text{при } x \in [0;\theta], \\
0 & \text{при } x \not\in [0;\theta],
\end{cases}
\]


где $\theta > 0$ — неизвестный параметр распределения и $\hat{\theta} = \bar{X}$.

\begin{enumerate}
\item Является ли оценка $\hat{\theta} = \bar{X}$ несмещенной оценкой неизвестного параметра $\theta$?
\item Подберите константу $c$ так, чтобы оценка $\tilde{\theta} = c\bar{X}$ оказалась несмещенной оценкой неизвестного параметра $\theta$.
\end{enumerate}

\item Пусть $X = (X_1,X_2,X_3)$ — случайная выборка из распределения Бернулли с неизвестным параметром $p \in (0,1)$. Какие из следующих ниже оценкой являются несмещенными? Среди перечисленных ниже оценок найдите наиболее эффективную оценку:

\begin{itemize}
  \item $\hat{p}_1 = \dfrac{X_1+X_3}{2}$,
  \item $\hat{p}_2 = \frac{1}{4}X_1+\frac{1}{2}X_2+\frac{1}{4}X_3$,
  \item $\hat{p}_3 = \frac{1}{3}X_1+\frac{1}{3}X_2+\frac{1}{3}X_3$.
\end{itemize}

\item Пусть $X=(X_1, \ldots,X_n)$ — случайная выборка из распределения с плотностью

\[
f(x,\theta) =
\begin{cases}
\frac{1}{\theta} \ e^{-\frac{x}{\theta}} & \text{при } x \geq 0, \\
0 & \text{при } x < 0,
\end{cases}
\]
где $\theta > 0$ — неизвестный параметр.
Является ли оценка  $\hat{\theta}_n = \dfrac{X_1+...+X_n}{n+1}$ состоятельной?

\item Пусть $X=(X_1, \ldots ,X_n)$ — случайная выборка из распределения с плотностью распределения

\[
f(x,\theta) = \begin{cases}
\dfrac{6x(\theta-x)}{\theta^3} & \text{при } x \in [0;\theta], \\
0 & \text{при } x \not\in [0;\theta], \end{cases}
\]


где $\theta > 0$ — неизвестный параметр распределения. Является ли оценка \ $\hat{\theta}_n = \frac{2n+1}{n}\bar{X}_n$ состоятельной оценкой неизвестного параметра $\theta$?

\item Пусть $X=(X_1, \ldots ,X_n)$ — случайная выборка из распределения с плотностью распределения

\[
f(x,\theta) =
\begin{cases}
\dfrac{6x(\theta-x)}{\theta^3} & \text{при } x \in [0;\theta], \\
0 & \text{при } x \not\in [0;\theta],
\end{cases}
\]


где $\theta > 0$ — неизвестный параметр распределения. Используя центральный момент 2-го порядка, при помощи метода моментов найдите оценку для неизвестного параметра $\theta$.

\item Пусть $X=(X_1, \ldots,X_n)$ — случайная выборка. Случайные величины $X_1, \ldots, X_n$ имеют дискретное распределение, которое задано при помощи таблицы

\begin{center}
\begin{tabular}{cccc}
\toprule
 $x$ & $-3$  &$ 0 $  & $2 $  \\
 \midrule
 $\P(X_i = x)$ & $2/3 - \theta$ & $1/3$ & $\theta$ \\
 \bottomrule
\end{tabular}
\end{center}

Используя второй начальный момент, при помощи метода моментов найдите оценку неизвестного параметра $\theta$. Для реализации случайной выборки $x=(0,0,-3,0,2)$ найдите числовое значение найденной оценки параметра $\theta$.

\item Пусть $X=(X_1, \ldots,X_n)$ — случайная выборка из распределения с плотностью распределения

\[
f(x,\theta) =
\begin{cases}
\frac{2x}{\theta} \ e^{-\frac{x^2}{\theta}} & \text{при } x>0, \\
0 & \text{при } x \leq 0,
\end{cases}
\]

где $\theta > 0$. При помощи метода максимального правдоподобия найдите оценку неизвестного параметра $\theta$.

\item Пусть $X=(X_1, \ldots, X_n)$ – случайная выборка из распределения Бернулли с параметром $\P \in (0;1)$. При помощи метода максимального правдоподобия найдите оценку неизвестного параметра $\P$.

\item Пусть $X=(X_1, \ldots, X_n)$ — случайная выборка из распределения с плотностью

\[
f(x,\theta) =
\begin{cases}
\frac{1}{\theta} \ e^{-\frac{x}{\theta}} & \text{при } x \geq 0, \\
0 & \text{при } x < 0, \end{cases}
\]

где $\theta > 0$ — неизвестный параметр. Является ли оценка  $\hat{\theta} = \bar{X}$ эффективной?

\item Стоимость выборочного исследования генеральной совокупности, состоящей из трех страт, определяется по формуле $TC = c_1n_1 + c_2n_2 + c_3n_3$, где $c_i$ — цена одного наблюдения в $i$-ой страте, a $n_i$ — число наблюдений, которые приходятся на $i$-ую страту. Найдите $n_1$, $n_2$ и $n_3$, при которых дисперсия стратифицированного среднего достигает наименьшего значения, если бюджет исследования 8000 и имеется следующая информация:

\begin{center}
\begin{tabular}{cccc}
\toprule
 Страта & $1$ & $2$ & $3$  \\
 \midrule
 Среднее значение & $30$ & $40$ & $50$ \\
 Стандартная ошибка  & $5$ & $10$ & $20$ \\
 Вес & $25\%$ & $25\%$ & $50\%$ \\
 Цена наблюдения & $1$ & $5$ & $10$ \\
 \bottomrule
\end{tabular}
\end{center}

\end{enumerate}

Ответы:

\begin{enumerate}
\item
\begin{enumerate}
\item $\approx 0.15$
\item $U \sim \cN(101,29)$, $f(u) = \frac{1}{\sqrt{2\pi\cdot 29}}e^{-\frac{1}{2}\frac{(u-101)^2}{29}}$
\item $\approx 0.02$
\end{enumerate}
\item
\begin{enumerate}
\item $71.14$
\item $f(y|x=170) = \frac{1}{\sqrt{2\pi\cdot20}}e^{-\frac{1}{2}\frac{(y-71.14)^2}{20}}$
\item $\approx 0$
\end{enumerate}
\item
\begin{enumerate}
\item $0.25$
\item $0.6875$
\item $0.91(6)$
\item $0.75$
\item $-0.28125$
\end{enumerate}
\item
\begin{enumerate}
\item $-1, 0, 1, 1$
\item $-1$
\item $1$
\item $f(x) = \begin{cases}
0, & x < -1 \\
0.25, & -1 \leq x < 0 \\
0.5, & 0 \leq x < 1 \\
1, & x \geq 1
\end{cases}$
\end{enumerate}
\item
\begin{enumerate}
\item $\theta$
\item да
\end{enumerate}

\item
\begin{enumerate}
\item нет, оценка смещена
\item $c = 2$
\end{enumerate}
\item
\begin{enumerate}
\item все оценки несмещенные
\item $\hat{p}_3$ наиболее эффективная
\end{enumerate}
\item да
\item да
\item $\hat{\theta}_{MM} = \sqrt{\frac{\sum_{i=1}^n(X_i-\overline{X})^2\cdot20}{n}}$

\item $\hat{\theta}_{MM} = \frac{1}{5}\left(6 - \frac{1}{n}\sum_{i=1}^n X_i^2 \right)$, $\hat{\theta}_{MM} = 0.68$
\item $\hat{\theta}_{ML} = \frac{\sum_{i=1}^n x_i^2}{n}$
\item $\hat{p}_{ML} = \frac{\sum_{i=1}^n x_i}{n}$
\item да
\item $n_1 \approx 260$, $n_2 \approx 232$, $n_3 \approx 658$

\end{enumerate}




\end{document}
