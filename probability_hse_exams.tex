\documentclass[12pt, a4paper]{article}\usepackage[]{graphicx}\usepackage[]{color}
%% maxwidth is the original width if it is less than linewidth
%% otherwise use linewidth (to make sure the graphics do not exceed the margin)
\makeatletter
\def\maxwidth{ %
  \ifdim\Gin@nat@width>\linewidth
    \linewidth
  \else
    \Gin@nat@width
  \fi
}
\makeatother

\definecolor{fgcolor}{rgb}{0.345, 0.345, 0.345}
\newcommand{\hlnum}[1]{\textcolor[rgb]{0.686,0.059,0.569}{#1}}%
\newcommand{\hlstr}[1]{\textcolor[rgb]{0.192,0.494,0.8}{#1}}%
\newcommand{\hlcom}[1]{\textcolor[rgb]{0.678,0.584,0.686}{\textit{#1}}}%
\newcommand{\hlopt}[1]{\textcolor[rgb]{0,0,0}{#1}}%
\newcommand{\hlstd}[1]{\textcolor[rgb]{0.345,0.345,0.345}{#1}}%
\newcommand{\hlkwa}[1]{\textcolor[rgb]{0.161,0.373,0.58}{\textbf{#1}}}%
\newcommand{\hlkwb}[1]{\textcolor[rgb]{0.69,0.353,0.396}{#1}}%
\newcommand{\hlkwc}[1]{\textcolor[rgb]{0.333,0.667,0.333}{#1}}%
\newcommand{\hlkwd}[1]{\textcolor[rgb]{0.737,0.353,0.396}{\textbf{#1}}}%
\let\hlipl\hlkwb

\usepackage{framed}
\makeatletter
\newenvironment{kframe}{%
 \def\at@end@of@kframe{}%
 \ifinner\ifhmode%
  \def\at@end@of@kframe{\end{minipage}}%
  \begin{minipage}{\columnwidth}%
 \fi\fi%
 \def\FrameCommand##1{\hskip\@totalleftmargin \hskip-\fboxsep
 \colorbox{shadecolor}{##1}\hskip-\fboxsep
     % There is no \\@totalrightmargin, so:
     \hskip-\linewidth \hskip-\@totalleftmargin \hskip\columnwidth}%
 \MakeFramed {\advance\hsize-\width
   \@totalleftmargin\z@ \linewidth\hsize
   \@setminipage}}%
 {\par\unskip\endMakeFramed%
 \at@end@of@kframe}
\makeatother

\definecolor{shadecolor}{rgb}{.97, .97, .97}
\definecolor{messagecolor}{rgb}{0, 0, 0}
\definecolor{warningcolor}{rgb}{1, 0, 1}
\definecolor{errorcolor}{rgb}{1, 0, 0}
\newenvironment{knitrout}{}{} % an empty environment to be redefined in TeX

\usepackage{alltt}

%\usepackage[pdf]{pstricks} % to use QR barcodes
%\usepackage{pst-barcode}
% sudo yum install texlive-auto-pst-pdf
% sudo yum install texlive-pst-barcode texlive-pdfcrop

% RNW RNW RNW RNW RNW RNW RNW RNW RNW RNW RNW RNW RNW RNW RNW RNW RNW RNW RNW RNW RNW RNW RNW RNW RNW RNW RNW RNW




\usepackage{fontspec}
\usepackage{polyglossia}

\setmainlanguage{russian}
\setotherlanguages{english}

% download "Linux Libertine" OTF-fonts:
% http://www.linuxlibertine.org/index.php?id=91&L=1
\setmainfont{Linux Libertine O} % or Helvetica, Arial, Cambria
% why do we need \newfontfamily:
% http://tex.stackexchange.com/questions/91507/
\newfontfamily{\cyrillicfonttt}{Linux Libertine O}
\newfontfamily{\cyrillicfont}{Linux Libertine O}
\newfontfamily{\cyrillicfontsf}{Linux Libertine O}

\usepackage{etoolbox} % to use ifdef, must be after babel
\input{title_bor_utf8} % use local copy

\usepackage{epigraph}

\AddEnumerateCounter{\asbuk}{\russian@alph}{щ} % для списков с русскими буквами
\setlist[enumerate, 2]{label=\asbuk*),ref=\asbuk*}


\unitlength=0.6mm

\title{Подборка экзаменов по теории вероятностей. \\Факультет экономики, НИУ-ВШЭ}
\date{\today}
\author{Коллектив кафедры \\
математической экономики и эконометрики,\\
 фольклор}


%%%%%%%%%%%%%%%%%% вставки
%%%%%%%%%%%%%%%%%%%%%%%%%%%%%%%%%%%%%%% Списки без уродских отступов
\newenvironment{enumerate*}{
\begin{enumerate}
  \setlength{\itemsep}{0pt}
  \setlength{\parskip}{0pt}
  \setlength{\parsep}{0pt}
}{\end{enumerate}}

\newenvironment{itemize*}{
\begin{itemize}
  \setlength{\itemsep}{0pt}
  \setlength{\parskip}{0pt}
  \setlength{\parsep}{0pt}
}{\end{itemize}}

\abovedisplayskip=0mm
\abovedisplayshortskip=0mm
\belowdisplayskip=0mm
\belowdisplayshortskip=0mm
%%%%%%%%%%%%%%%%%%%%%%%%%%%%%%%%%%%%%%%%%%%%%%%%%%%%%%%%%%%%%%%%%%%%%%
\newcommand{\MIN}{\textbf{(MIN)}{}}
\newcommand{\ofbr}[1]{\bigl( \{ #1 \} \bigr)}     % Например, вероятность события. Большие круглые, нормальные фигурные скобки вокруг аргумента
%%%%%%%%%%%%%%%%%
\newenvironment{centered}{%
  \begin{list}{}{%
    \topsep0pt
  }
  \centering
  \item[]
}
{\end{list}}
%%%%%%%%%%%%%%%%%%%%%%%%%%%%%%%%%%%%%%%%%%%%%%%%%%%%%%%%%%%%%%%%%%%%%%%%%




\DeclareMathOperator*\plim{plim}
\newcommand{\cN}{\mathcal{N}}
\IfFileExists{upquote.sty}{\usepackage{upquote}}{}
\begin{document}
\maketitle

\tableofcontents{}


\parindent=0 pt % no indent

\section{Описание}

Свежую версию можно скачать с блога \url{http://pokrovka11.wordpress.com/} или с github репозитория \url{http://bdemeshev.github.io/pr201/}.


Уникальное предложение для студентов факультета экономики ГУ-ВШЭ:


Найдите ошибки в этом документе или пришлите отсутствующие решения в техе и получите дополнительные бонусы! Найденные смысловые ошибки поощряются сильнее, чем просто опечатки. Замеченные ошибки и новые решения оформляйте в виде issues на \url{https://github.com/bdemeshev/probability_hse_exams/issues/}. Перед публикацией issue, пожалуйста, свертесь со свежей версией подборки.

Неполный список благодарностей:

\begin{enumerate}
\item Андрей Зубанов, решения (экзамен 26.03.2012, \ldots)
\item Кирилл Пономарёв, решения (контрольная 1, 2014)
\item Александр Левкун, решения (контрольная 1, 2014)
\item Оля Гнилова, решения (кр 3 2011, 2014, 2015, 2016, \ldots)
\item Настя Жаркова
\item Гарик Варданян
\end{enumerate}


\section{Доброе напутствие пишущим эту подборку :)}

Здесь перечислены стилевые особенности коллекции, а узнать технические подробности по теху можно, например, \href{http://www.ccas.ru/voron/download/voron05latex.pdf}{в учебнике} К.В. Воронцова.

\begin{enumerate}

\item Дробную часть числа отделяй от целой точкой: $3.14$ — хорошо, $3{,}14$ — плохо.
\item Существует длинное тире, —, которое отличается от просто дефиса - и нужно, чтобы разделять части предложения. \href{https://ru.wikihow.com/напечатать-тире}{Инструкция в картинках по набору тире :)}
\item Выключные формулы следует окружать \verb|\[|\ldots\verb|\]|. Никаких \$\$\ldots\$\$!
\item Про остальные окружения: для системы уравнений подойдёт cases, для формул на несколько строк – multline*, для нумерации – enumerate.
\item Русский текст внутри формулы нужно писать в \verb|\text{|\ldots\}.
\item Для многоточий существует команда \verb|\ldots|.
\item В преамбуле определены сокращения! Самые популярные: \verb|\P, \E, \Var, \Cov, \Corr, \cN|.
\item Названия функций тоже идут со слэшем: \verb|\ln, \exp, \cos|\ldots
\item Таблицы нужно оформлять по стандарту booktabs. Самый удобный способ сделать это – зайти на
\href{https://www.tablesgenerator.com}{tablesgenerator} и выбрать там опцию booktabs table style вместо default table style.
\item Уважай букву ё – ставь над ней точки! :)
\end{enumerate}


\section{2004-2005}

\input{chapters/year_2004_2005.tex}

\section{2005-2006}

\input{chapters/year_2005_2006.tex}

\section{2006-2007}

\subsection{Контрольная работа №1, ??.11.2006}

Вывешенное решение может содержать неумышленные опечатки.

Заметил опечатку? Сообщи преподавателю!

\begin{enumerate}
\item  Из семей, имеющих троих разновозрастных детей, случайным
образом выбирается одна семья. Пусть событие А заключается в том,
что в этой семье
старший ребенок — мальчик, В — в семье есть хотя бы одна девочка.
\begin{enumerate}
\item Считая вероятности рождения мальчиков и девочек одинаковыми,
выяснить, являются ли события А и В независимыми.
\item Изменится ли результат, если вероятности рождения мальчиков и
девочек различны.
\end{enumerate}
Решение:
\begin{enumerate}
\item[а)] $\P(A)=0.5$, $\P(B)=1-\P(B^{c})=1-0.5^{3}=\frac{7}{8}$, $\P(A\cap
B)=0.5\cdot (1-0.5^{2})=\frac{3}{8}$, $\P(A\cap B)\neq \P(A)\P(B)$,
события зависимы.
\item[б)] $\P(A)=p$, $\P(B)=1-p^{3}$, $\P(A\cap B)=p(1-p^{2})$,
независимость событий возможна только при $p=0$ или $p=1$
\end{enumerate}
\item Студент решает тест (множественного выбора) проставлением
ответов наугад. В тесте 10 вопросов, на каждый из которых 4
варианта ответов. Зачет ставится в том случае, если правильных
ответов будет не менее 5.
\begin{enumerate}
\item Найти вероятность того, что студент правильно ответит только
на один вопрос
\item Найти наиболее вероятное число правильных ответов
\item Найти математическое ожидание и дисперсию числа правильных
ответов
\item Найти вероятность того, что студент получит зачет
\end{enumerate}
Решение:

Пусть $X$ - число правильных ответов.
\begin{enumerate}
\item[а)] $\P(X=1)=C_{10}^{1}\left(\frac{1}{4}\right)^{1}\left(\frac{3}{4}\right)^{9}$
\item[б)] $k_{\P(X=k)\rightarrow \max}=\lfloor p(n+1)\rfloor=\lfloor
\frac{11}{4}\rfloor=2$ (можно не зная формулы просто выбрать
наибольшую вероятность)
\item[в)] $\E(X)=10\E(X_{i})=\frac{10}{4}$
$\Var(X)=10\Var(X_{i})=10 \cdot \frac{1}{4} \cdot \frac{3}{4}$
\item[г)] $\sum_{i=5}^{10}C_{10}^{i}\left(\frac{1}{4}\right)^{i}\left(\frac{3}{4}\right)^{10-i}$
\end{enumerate}

\item Вероятность изготовления изделия с браком на некотором
предприятии равна 0.04. Перед выпуском изделие подвергается
упрощенной проверке, которая в случае бездефектного изделия
пропускает его с вероятностью 0.96, а в случае изделия с дефектом
- с вероятностью 0.05. Определить:
\begin{enumerate}
\item Какая часть изготовленных изделий выходит с предприятия
\item Какова вероятность того, что изделие, прошедшее упрощенную
проверку, бракованное
\end{enumerate}
Решение:

$A$ - изделие браковано, $B$ - изделие признано хорошим
\begin{enumerate}
\item[а)] $\P(B)=0.96\cdot 0.96+0.04\cdot 0.05$
\item[б)] $\P(A|B)=\frac{0.04\cdot 0.05}{\P(B)}$
\end{enumerate}
\item  Вероятность того, что пассажир, купивший билет, не придет к
отправлению поезда, равна 0.01. Найти вероятность того, что все
400 пассажиров явятся к отправлению поезда (использовать
приближение Пуассона).

Решение:

$\lambda=np=4$

$\P(X=k)=e^{-\lambda}\frac{\lambda^{k}}{k!}$

$\P(X=0)=e^{-4}$

\item Охотник, имеющий 4 патрона, стреляет по дичи до первого
попадания или до израсходования всех патронов. Вероятность
попадания при первом выстреле равна 0.6, при каждом последующем -
уменьшается на 0.1. Найти
\begin{enumerate}
\item Закон распределения числа патронов, израсходованных охотником
\item Математическое ожидание и дисперсию этой случайной величины
\end{enumerate}
Решение:
\begin{enumerate}
\item[а)]
\begin{tabular}{|c|c|c|c|c|}
  \hline
  % after \\: \hline or \cline{col1-col2} \cline{col3-col4} ...
  $x_{i}$ & 1 & 2 & 3 & 4 \\
  \hline
  $\P(X=x_{i})$ & $0.6$& $(1-0.6)\cdot 0.5$ & $(1-0.6)\cdot(1-0.5)\cdot 0.4$ & $1-p_{1}-p_{2}-p_{3}$ \\
  \hline
\end{tabular} \\
\begin{tabular}{|c|c|c|c|c|}
  \hline
  % after \\: \hline or \cline{col1-col2} \cline{col3-col4} ...
  $x_{i}$ & 1 & 2 & 3 & 4 \\
  \hline
  $\P(X=x_{i})$ & $0.6$& $0.2$ & $0.08$ & $0.12$ \\
  \hline
\end{tabular}
\item[б)] $\E(X)=1.7$, $\Var(X)\approx 1.08$
\end{enumerate}
\item Поезда метрополитена идут регулярно с интервалом 2 минуты.
Пассажир приходит на платформу в случайный момент времени. Какова
вероятность того, что ждать пассажиру придется не более полминуты.
Найти математическое ожидание и дисперсию времени ожидания поезда.

Решение:

$\P(X\le 0.5)=\frac{0.5}{2}=0.25$, $\E(X)=\frac{0+2}{2}=1$ (здравый
смысл)

$\Var(X)=\E(X^{2})-(\E(X))^{2}$

$\E(X^{2})=\int_{0}^{2}t^{2}\cdot p(t)dt=\int_{0}^{2}t^{2}\cdot 0.5dt=\frac{4}{3}$

\item Время работы телевизора «Best» до первой поломки является
случайной величиной, распределенной по показательному закону.
Определить вероятность того, что телевизор проработает более 15
лет, если среднее время безотказной работы телевизора фирмы «Best»
составляет 10 лет. Какова вероятность, что телевизор,
проработавший 10 лет, проработает еще не менее 15 лет?

Решение:

$\E(X)=10=\frac{1}{\lambda}$, $\lambda=\frac{1}{10}$, $p(t)=\lambda e^{\lambda t}$ при $t>0$

$\P(X>15)=\int_{15}^{\infty}p(t)dt=...=e^{-\frac{3}{2}}$

$\P(X>25|X>10)=\frac{\P(X>25)}{\P(X>10)}=...=e^{-\frac{3}{2}}$
\end{enumerate}

Дополнительная задача:

Пусть случайные величины $X_{1}$ и $X_{2}$ независимы и равномерно
распределены на отрезках $[-1;1]$ и $[0;1]$, соответственно. Найти
вероятность того, что $\max\{X_{1},X_{2}\}>0.5$, функцию
распределения случайной величины $Y=\max\{X_{1},X_{2}\}$.

Решение:

Функция распределения:

$F_{Y}(t)=\P(Y\le t)=\P(\max\{X_{1},X_{2}\}\le t)=\P(X_{1}\le t\cap X_{2}\le t)=\P(X_{1}\le t)\P(X_{2}\le t)=\frac{t+1}{2}\cdot t$ при $t\in [0;1]$.

При $t>1$ получаем, что $F_{Y}(t)=1$ и при $t<0$ получаем, что $F_{Y}(t)=0$.

$\P(\max\{X_{1},X_{2}\}>0.5)=1-\P(\max\{X_{1},X_{2}\}\le 0.5)=1-F(0.5)=\frac{5}{8}$

\subsection{Контрольная работа №2, 27.01.2007}

\subsubsection*{Часть I.}

Обведите верный ответ:

\begin{enumerate}
\item Сумма двух нормальных независимых случайных величин нормальна.
Да.
\item Нормальная случайная величина может принимать отрицательные
значения. Да.
\item Пуассоновская случайная величина является непрерывной. Нет.
\item Дисперсия суммы зависимых величин всегда не меньше суммы
дисперсий. Нет.
\item Теорема Муавра-Лапласа является частным случаем центральной
предельной. Да.
\item Пусть $X$ - длина наугад выловленного удава в сантиметрах, а
$Y$ - в дециметрах. Коэффициент корреляции между этими
величинами равен $\frac{1}{10}$. Нет.
\item Математическое ожидание выборочного среднего не зависит от
объема выборки, если $X_{i}$ одинаково распределены. Да.
\item Зная закон распределения $X$ и закон распределения $Y$
можно восстановить совместный закон распределения пары $(X,Y)$. Нет.
\item Если  $X$  - непрерывная случайная величина,  $\E\left(X\right)=6$  и
$\Var\left(X\right)=9$ , то  $Y=\frac{X-6}{3} \sim
\cN\left(0;1\right)$.  Нет.
\item Если ты отвечать на первые 10 вопросов этого теста наугад, то
число правильных ответов - случайная величина, имеющая
биномиальное распределение. Да.
\item По-моему, сегодня хорошая погода, и вместо контрольной можно
было бы покататься на лыжах. Да!
\end{enumerate}

$[$правильно=+1 балл; нет ответа=неправильно=0 баллов$]$

Любой ответ на 11 считается правильным.

Тест не является блокирующим.

Обозначения:

$\E(X)$ - математическое ожидание

$\Var(X)$ - дисперсия

\subsubsection*{Часть II.}

Стоимость задач 10 баллов.

\begin{enumerate}
% числа выверены
\item Совместный закон распределения случайных величин  $X$  и  $Y$
задан таблицей:

$\begin{array}{|c|ccc|} \hline {} & {Y=-1} & {Y=0} & {Y=2}
\\  \hline {X=0} & {0.1} & {c} & {0.2}
\\ {X=1} & {0.1} & {0.2} & {0.1} \\  \hline  \end{array}$

Найдите  $c$,  $\P\left(Y>-X\right)$,  $\E\left(X\cdot Y^{2} \right)$,  $\E\left(Y|X>0\right)$

Ответы: $c=0.3$ $[1]$, $\P(Y>-X)=0.5$ $[3]$, $\E(XY^{2})=0.5$ $[3]$,
$\E(Y|X>0)=\frac{0.1}{0.4}=0.25$ $[3]$

% числа выверены
\item Случайный вектор  $\left(\begin{array}{c}
{X_{1} } \\ {X_{2} }
\end{array}\right)$  имеет нормальное распределение с
математическим ожиданием  $\left(\begin{array}{c} {2} \\ {-1}
\end{array}\right)$  и ковариационной матрицей
$\left(\begin{array}{cc} {9} & {-4.5} \\ {-4.5} & {25}
\end{array}\right)$ . Найдите  $\P\left(X_{1} +3X_{2} >20\right)$.

Ответы: $\E(Y)=-1$ $[2]$, $\Var(Y)=207$ $[4]$, $\P(Y>20)=\P(Z>\frac{21}{\sqrt{207}})=\P(Z>1.46)=0.07$ $[4]$

% числа выверены
\item Совместная функция плотности имеет вид
\[
p_{X,Y} \left(x,y\right)=
\begin{cases}
x+y, & \text{ если } x\in \left[0;1\right],\, y\in \left[0;1\right] \\
0, & \text{ иначе}
\end{cases}
\]
Найдите  $\P\left(Y>2X\right)$ ,  $\E\left(X\right)$

Решение:

$\P(Y>2X)=\int_{0}^{1}\int_{0}^{y/2}(x+y)dxdy=\frac{5}{24}$ $[5]$

$\E(X)=\int_{0}^{1}\int_{0}^{1}x(x+y)dxdy=\frac{7}{12}$ $[5]$

(если интеграл выписан верно, но не взят, то $[3]$ вместо $[5]$)

% числа выверены
\item В супермаркете «Покупан» продаются различные вина:

\begin{tabular}{|c|c|c|c|}
  \hline
  % after \\: \hline or \cline{col1-col2} \cline{col3-col4} ...
  Вина & Доля & Средняя цена за бутылку (у.е.) & Стандартное отклонение (у.е.) \\
  \hline
  Элитные & 0.1 & 150 & 24 \\
  Дорогие & 0.3 & 40 & 12 \\
  Дешевые & 0.6 & 10 & 10 \\
  \hline
\end{tabular}

Чтобы оценить среднюю стоимость предлагаемого вина производится
случайная выборка 10 бутылок.
\begin{enumerate}
\item Какое количество элитных, дорогих и дешевых вин должно
присутствовать в выборке, для того, чтобы выборочное среднее
значение цены имело минимальную дисперсию? $[5]$
\item Чему равна минимальная дисперсия? $[5]$
\end{enumerate}
Решение:

Используя метод множителей Лагранжа:

$L=\frac{(0.1\cdot 24)^{2}}{a}+\frac{(0.3\cdot
12)^{2}}{b}+\frac{(0.6\cdot 10)^{2}}{c}+\lambda(10-a-b-c)$

\ldots

$a=2$, $b=3$, $c=5$, можно было использовать готовую формулу
$n_{i}=\frac{w_{i}\sigma_{i}}{\sum w_{j}\sigma_{j}}$

$\Var(\overline{X}^{s})=14.4$

% числа выверены
\item Допустим, что закон распределения $X_{n}$ имеет вид:

\begin{tabular}{|c|c|c|c|}
  \hline
  X & -1 & 0 & 2 \\
  \hline
  Prob & $\theta$ & $2\theta-0.2$ & $1.2-3\theta$ \\
  \hline
\end{tabular}

Имеется выборка: $X_{1}=0$, $X_{2}=2$.
\begin{enumerate}
\item Найдите оценку $\hat{\theta}$ методом максимального правдоподобия
\item Найдите оценку $\hat{\theta}$ методом моментов
\end{enumerate}

Решение:
\begin{enumerate}
\item[а)] $(2\theta-0.2)(1.2-3\theta)\rightarrow\max$,
$\hat{\theta}=0.25$ $[5]$\\
\item[б)] $2.4-7\hat{\theta}=1$, $\hat{\theta}=0.2$ $[5]$
\end{enumerate}

% числа выверены
\item В среднем 30\% покупателей супермаркета делают покупку на сумму
свыше 700 рублей. Какова вероятность того, что из 200 $[$случайно
выбранных$]$ покупателей
более 33\% сделают покупку на сумму свыше 700 рублей?

Решение:

$\P(\overline{X}>0.33)=\P\left(\frac{\bar{X}-0.3}{\sqrt{\frac{0.3\cdot
0.7}{200}}}>\frac{0.33-0.3}{\sqrt{\frac{0.3\cdot
0.7}{200}}}\right)=\P(Z>1.03)=0.15$

Баллы: $[3]$ - $\Var$, $[4]$ - $Z$, $[3]$ - таблица

% числа выверены
\item Пусть $X_{i}$ нормально распределены и
независимы. Имеется выборка
из трех наблюдений: 2, 0, 1.
\begin{enumerate}
\item[a)] Найдите несмещенные оценки для математического ожидания и
дисперсии, $\bar{X}$ и $\hat{\sigma}^{2}$. $[2]+[3]$
\item[б)] Найдите вероятность того, что оценка дисперсии превосходит
истинную дисперсию более чем в 3 раза $[5]$
\end{enumerate}
Решение:

$\bar{X}=1$, $\hat{\sigma}^{2}=1$

$\P(\hat{\sigma}^{2}>3\sigma^{2})=\P\left(2\frac{\hat{\sigma}^{2}}{\sigma^{2}}>6\right)=\P(\chi_{2}^{2}>6)=0.05$

 % числа выверены
\item Известно, что у случайной величины $X$ есть
математическое
ожидание, $\E(X)=0$, и дисперсия.
\begin{enumerate}
\item[а)] Укажите верхнюю границу для $\P(X^{2}>4\Var(X))$? $[5]$
\item[б)] Найдите указанную вероятность, если дополнительно известно, что
$X$ нормально распределена. $[5]$
\end{enumerate}
Решение:
\begin{enumerate}
\item[a)] $\P(X^{2}>4\Var(X))=\P(|X-0|>2\sigma)\le
\frac{Var{X}}{4\Var(X)}=\frac{1}{4}$
\item[б)] $\P(X^{2}>4\Var(X))=\P(|Z|>2)=0.05$
\end{enumerate}

% числа выверены
\item Пусть $X_{i}$ независимы и экспоненциально
распределены, т.е. имеют функцию плотности вида
$p(t)=\frac{1}{\theta}e^{-\frac{1}{\theta}t}$ при $t>0$.
\begin{enumerate}
\item Постройте оценку математического ожидания методом максимального
правдоподобия $[2]$
\item Является ли оценка несмещенной? $[2]$
\item Найдите дисперсию оценки $[2]$
\item С помощью неравенства Крамера-Рао проверьте, является ли
оценка эффективной среди несмещенных оценок? $[2]$
\item Является ли построенная оценка состоятельной? $[2]$
\end{enumerate}
Решение:
\begin{enumerate}
\item[а)] $\bar{X}$
\item[б)] Да;
\item[в)] $\Var(\bar{X})=\frac{\theta^{2}}{n}$;
\item[г)] да: несмещенность и предел дисперсии равный нулю;
\end{enumerate}

 % числа выверены
\item Независимые случайные величины $X_{i}$ распределены
равномерно на отрезке $[0;a]$, известно, что $a>10$. Исследователь
хочет оценить
параметр $\theta=\frac{1}{\P(X_{i}<5)}$.
\begin{enumerate}
\item Используя $\bar{X_{n}}$ постройте несмещенную оценку
$\hat{\theta}$ для $\theta$ $[4]$
\item Найдите дисперсию построенной оценки $[3]$
\item Является ли построенная оценка состоятельной? $[3]$
\end{enumerate}
Решение:
\begin{enumerate}
\item[a)] $\E(\bar{X})=\frac{a}{2}$,
$\theta=\frac{1}{\P(X_{i}<5)}=\frac{1}{5/a}=\frac{1}{5}a$ \\
$\hat{\theta}=\frac{2}{5}\bar{X}$
\item[б)] $\Var(\hat{\theta}_{n})=(\frac{2}{5})^{2}\cdot\frac{a^{2}}{12n}$
\item[в)] $\lim \Var(\hat{\theta}_{n})=0$, оценка несмещенная,
следовательно, состоятельная.
\end{enumerate}
\end{enumerate}

\subsubsection*{Часть III.}

Стоимость задачи 20 баллов.

Требуется решить \textbf{\underbar{одну}} из двух 11-х задач по
выбору!

\begin{enumerate}
\item[11-A.] Каждый день Кощей Бессмертный кладет в сундук случайное количество
копеек (от одной до ста, равновероятно). Сколько в среднем дней нужно Кощею, чтобы набралось не меньше рубля?

Решение:

Обозначим $e_{n}$ - сколько дней осталось в среднем ждать, если
уже набрано $n$ копеек.

Тогда:

$e_{100}=0$

$e_{99}=1$

$e_{98}=\frac{1}{100}e_{99}+\frac{99}{100}e_{100}+1=1+\frac{1}{100}$

$e_{97}=\frac{1}{100}e_{98}+\frac{1}{100}e_{99}+\frac{98}{100}e_{100}+1=(1+\frac{1}{100}))^{2}$

$e_{96}=\frac{1}{100}e_{97}+\frac{1}{100}e_{98}+\frac{1}{100}e_{99}+\frac{97}{100}e_{100}+1=(1+\frac{1}{100})^{3}$

\ldots

По индукции легко доказать, что $e_{n}=(1+\frac{1}{100})^{99-n}$

Таким образом, $e_{0}=(1+\frac{1}{100})^{99}=2.718 \ldots$

\item[11-B.] Каждый день Петя знакомится с новыми девушками. С вероятностью 0.7
ему удается познакомиться с одной девушкой; с вероятностью 0.2 — с
двумя; с вероятностью 0.1 — не удается. Дни, когда Пете не удается
познакомиться ни с одной девушкой, Петя считает неудачными.

Какова вероятность, что до первого неудачного дня Пете удастся
познакомиться $[$ровно$]$ с 30-ю девушками?

Решение:

$p_{0}=0.1$, $p_{1}=0.7\cdot 0.1$;

$p_{n}=\P($в первый день Петя познакомился с одной
девушкой$)p_{n-1}+\P($в первый день Петя познакомился с двумя
девушками$)p_{n-2}$;

Разностное уравнение: $p_{n}=0.7p_{n-1}+0.2p_{n-2}$

\emph{Подсказка}: Думайте!
\end{enumerate}


\subsection{Контрольная работа №3, 21.02.2007}

Нужные и ненужные формулы: \\ \\
$T$ - сумма чего-то там. \\
Если $H_{0}$ верна, то $\E(T)=\frac{n}{2}$ и $\Var(T)=\frac{n}{4}$ \\ \\
$T$ - сумма каких-то рангов. \\
Если $H_{0}$ верна, то $\E(T)=\frac{n(n+1)}{4}$ и
$\Var(T)=\frac{n(n+1)(2n+1)}{24}$. \\ \\
$T$ - сумма каких-то рангов. \\
Если $H_{0}$ верна, то $\E(T)=\frac{n_{1}(n_{1}+n_{2}+1)}{2}$,
$\Var(T)=\frac{n_{1}n_{2}(n_{1}+n_{2}+1)}{12}$. \\ \\
$\cos^{2}(x)+\sin^{2}(x)=1$ \\ \\

\textbf{УДАЧИ!}

\subsubsection*{Часть I.}

Обведите нужный ответ

\begin{enumerate}
\item Если $X\sim \cN(0;12)$, $Y\sim \cN(12,24)$, $\Corr(X,Y)=0$, то
$X+Y\sim \cN(12,36)$.
Да. Нет.

$[$любой ответ считался правильным. на самом деле верный ответ -
нет$]$

\item Если закон распределения $X$ задан табличкой

\begin{tabular}{|c|c|c|}
  \hline
  $x$ & 0 & 1 \\
  \hline
  $\P$ & 0.5 & 0.5 \\
  \hline
\end{tabular}, то $X$ - нормально распределена. Да. Нет.

\item Непараметрические тесты неприменимы, если выборка имеет
$\chi^{2}$ распределение. Да. Нет.
\item P-значение показывает вероятность отвергнуть нулевую
гипотезу, когда она верна. Да. Нет.
\item Если $t$-статистика равна нулю, то P-значение также равно
нулю. Да. Нет.
\item Если гипотеза отвергает при 5\%-ом уровне значимости, то
она будет отвергаться и при 1\%-ом уровне значимости. Да. Нет.
\item При прочих равных 90\% доверительный интервал шире 95\%-го. Да. Нет.
\item Значение функции плотности может превышать единицу. Да. Нет.
\item Для любой случайной величины  $\E(X^{2} )\ge
(\E(X))^{2}$. Да. Нет.
\item Если $\Corr(X,Y)>0$, то $\E(X)\E(Y)<\E(XY)$. Да. Нет.
\item На экзаменационной работе не шутят! Нет, шутят.
\end{enumerate}

$[$правильно=+1 балл; нет ответа=неправильно=0 баллов$]$

Ответ «да» означает истинное утверждение, ответ «нет» - ложное.

Тест не является блокирующим.

$[$Неправильное использование таблиц = штраф 2 балла$]$

$[$Неправильные степени свободы = штраф 2 балла$]$

\subsubsection*{Часть II.}

Стоимость задач 10 баллов.

\begin{enumerate}
 % числа выверены
\item Из урны с 5 белыми и 7 черными шарами случайным образом вынимается
2 шара. Случайная величина $X$ принимает значение (-1), если оба
шара - белые; 0, если шары разного цвета и 1, если оба шара
черные.
\begin{enumerate}
\item Найдите $\P(X=-1)$ $[2]$ , $\E(X)$ $[3]$, $\Var(X)$ $[3]$
\item Постройте функцию распределения величины $X$ $[2$, достаточно аккуратно выписать функцию$]$
\end{enumerate}

 % числа выверены
\item Случайная величина $X$ имеет функцию распределения
\[
F_{X}(t)=
\begin{cases}
  0, & t<0 \\
  ct^{2}, & 0\le t <1 \\
  1, & 1\le t \\
\end{cases}
\]
\begin{enumerate}
\item Найдите $c$ $[1]$, $\P(0.5<X<2)$ $[1]$, 25\%-ый квантиль $[1]$
\item Найдите $\E(X)$ $[2]$, $\Var(X)$ $[2]$, $\Cov(X,-X)$ $[1]$, $\Corr(2X,3X)$ $[1]$
\item Выпишите функцию плотности величины $X$ $[1]$
\end{enumerate}

% числа выверены
\item Доходности акций двух компаний являются случайными величинами $X$
и $Y$ с одинаковым математическим ожиданием и ковариационной
матрицей  $\left(%
\begin{array}{cc}
  4 & -2 \\
  -2 & 9 \\
\end{array}%
\right).$
\begin{enumerate}
\item Найдите $\Corr(X,Y)$  $[1]$,

Ответ: $\Corr=-\frac{1}{3}$
\item $[5]$ В какой пропорции нужно приобрести акции этих двух
компаний, чтобы дисперсия доходности получившегося портфеля была наименьшей?
\item  $[2]$ Можно ли утверждать, что величины $X+Y$ и $7X-2Y$ независимы?
\item $[2]$ Изменится ли ответ на пункт «в», если дополнительно
известно, что величины $X$ и $Y$ в совокупности нормально распределены?
\end{enumerate}
Подсказка: Если $R$ - доходность портфеля, то $R=\alpha
X+(1-\alpha)Y$

Ответ: $\alpha=\frac{11}{17}$

% числа выверены
\item Проверка 40 случайно выбранных лекций показала, что студент
Халявин присутствовал только на двух из них.
\begin{enumerate}
\item{} $[4]$ Найдите 90\%-ый доверительный интервал для вероятности
увидеть Халявина на лекции.
\item{} $[5]$ Укажите минимальный размер выборки, необходимый для того,
чтобы с вероятностью 0.9 выборочная доля посещаемых Халявиным
лекций отличалась от соответствующей вероятности не более, чем на 0.1.
\item{} $[1]$ Какие предпосылки и теоремы использовались при ответах на предыдущие пункты?
\end{enumerate}

% числа выверены
\item Изучается эффективность нового метода обучения. У группы из 40
студентов, обучавшихся по новой методике, средний бал на экзамене
составил 322.12, а выборочное стандартное отклонение 54.53.
Аналогичные показатели для независимой выборки из 60 студентов
того же курса, обучавшихся по старой методике,
приняли значения 304.61 и 62.61 соответственно.
\begin{enumerate}
\item{} $[4]$ Проверьте гипотезу о равенстве дисперсий оценок в двух
группах.
\item{} $[1]$ Какие предпосылки использовались при ответе на «а»?
\item{} $[4]$ Постройте 90\% доверительный интервал для разницы
математических ожиданий оценок в двух группах
\item{} $[1]$ Можно ли считать новую методику более эффективной?
\end{enumerate}

% числа выверены
\item В парке отдыха за час 57 человек посетило аттракцион «Чертово
колесо», 48 - «Призрачные гонки» и 54 - «Американские горки». Можно ли на 5\% уровне значимости считать, что посетители
одинаково любят эти три аттракциона?

% числа выверены
\item Можно ли по имеющейся таблице утверждать о независимости пола и
доминирующей руки на 5\% уровне значимости?

\begin{tabular}{|c|c|c|}
  \hline
  Пол/рука & Правша & Левша \\
  \hline
  Мужчины & 16 & 76 \\
  Женщины & 25 & 81 \\
  \hline
\end{tabular}

% числа выверены
\item Пусть $X_{i}$ нормально распределены, независимы, $\E(X_{i})=0$,
$\Var(X_{i})=\theta$.
\begin{enumerate}
\item{} $[3]$ Постройте оценку $\hat{\theta}$ методом максимального
правдоподобия.
\item{} Проверьте свойства несмещенности, состоятельности,
эффективности у построенной оценки. $[$каждое свойство по $2$, если дано аккуратное определение, то $1]$
\item{} $[1]$ Какая оценка более предпочтительна: построенная или
привычная
$\hat{\sigma}^{2}=\frac{\sum(X_{i}-\bar{X})^{2}}{n-1}$?
\end{enumerate}

% числа выверены
\item Имеются две конкурирующие гипотезы:
\begin{enumerate}
\item[$H_0$:] Случайная величина X распределена равномерно на (0,100)
\item[$H_a$:] Случайная величина X распределена равномерно на (50,150)
\end{enumerate}
Исследователь выбрал следующий критерий: если $X<c$, принимать гипотезу $H_0$, иначе  $H_a$.
\begin{enumerate}
\item Дайте определение ошибок первого и второго рода. $[2+2]$
\item Постройте графики зависимостей ошибок первого и второго рода от $c$. $[3+3]$
\end{enumerate}

%числа выверены
\item Вася измерил длину 10 пойманных им рыб. Часть рыб была поймана на
левом берегу реки, а часть - на правом. Бывалые рыбаки говорят,
что на правом берегу реки рыба крупнее.

\begin{tabular}{|c|c|c|c|c|c|}
  \hline
  Левый берег & 25 & 45 & 37 & 47 & 51   \\
  \hline
  Правый берег & 49 & 28 & 39 & 46 & 57   \\
  \hline
\end{tabular}
\begin{enumerate}
\item{} $[10]$ С помощью теста Манна-Уитни (Mann-Whitney) проверьте
гипотезу о том, что выбор берега реки не влияет на среднюю длину
рыбы против
альтернативной гипотезы, что на правом берегу рыба длиннее.

\emph{Разрешается использование нормальной аппроксимации}
\item{} $[$Не оценивался$]$ Возможно ли в этой задаче использовать
(Wilcoxon Signed Rank Test)?
\end{enumerate}
\end{enumerate}

\subsubsection*{Часть III.}

Стоимость задачи 20 баллов.

Требуется решить \textbf{\underbar{одну}} из двух 11-х задач по
выбору!

\begin{enumerate}
\item[11-A.] Имеются две монетки. Одна правильная, другая - выпадает орлом с
вероятностью $0.45$. Одну из них, неизвестно какую, подкинули $n$
раз и сообщили Вам, сколько раз выпал орел. Ваша задача проверить
гипотезу $H_{0}$: «подбрасывалась правильная монетка» против
$H_{a}$:
«подбрасывалась неправильная монетка». \\
Каким должно быть наименьшее $n$ и критерий выбора гипотезы, чтобы
вероятность ошибок первого рода не превышала 10\%, а вероятность
ошибки второго рода не превышала 15\%?

\item[11-B.] Время горения лампочки – экспоненциальная случайная величина с
математическим ожиданием равным $\theta $. Вася включил
одновременно 20 лампочек. Величина  $Y$ обозначает время самого
первого перегорания.
\begin{enumerate}
\item{} $[8]$ Найдите $\E(Y)$
\item{} $[6]$ Постройте с помощью  $Y$ несмещенную оценку для  $\theta$
\item{} $[6]$ Сравните по эффективности оценку построенную в пункте
«б» и
обычное выборочное среднее
\end{enumerate}
\end{enumerate}


\section{2007-2008}

\input{chapters/year_2007_2008.tex}

\section{2008-2009}

\input{chapters/year_2008_2009.tex}

\section{2009-2010}

\input{chapters/year_2009_2010.tex}

\section{2010-2011}

\input{chapters/year_2010_2011.tex}

\section{2011-2012}

\input{chapters/year_2011_2012.tex}

\section{2012-2013}

\subsection{Контрольная работа №1, 14.11.2012}


%Теория вероятностей и математическая статистика. Контрольная работа 1, 14.11.12.

%\vspace{20pt}
%Обозначения: $\E(X)$ — математическое ожидание, $\Var(X)$ — дисперсия

\vspace{20pt}
14 ноября 1936 года в СССР была создана Гидрометеорологическая служба.
\vspace{20pt}

\begin{enumerate}

\item Погода завтра может быть ясной с вероятностью $0.3$ и пасмурной с вероятностью $0.7$. Вне зависимости от того, какая будет погода, Маша даёт верный прогноз с вероятностью $0.8$. Вовочка, не разбираясь в погоде, делает свой прогноз по принципу: с вероятностью $0.9$ копирует Машин прогноз, и с вероятностью $0.1$ меняет его на противоположный.
\begin{enumerate}
\item Какова вероятность того, что Маша спрогнозирует ясный день?
\item Какова вероятность того, что Машин и Вовочкин прогнозы совпадут?
\item Какова вероятность того, что день будет ясный, если Маша спрогнозировала ясный?
\item Какова вероятность того, что день будет ясный, если Вовочка спрогнозировал ясный?
\end{enumerate}

\item Машин результат за контрольную, $M$, равномерно распределен на отрезке $[0;1]$. Вовочка ничего не знает, поэтому списывает у Маши, да ещё может наделать ошибок при списывании. Поэтому Вовочкин результат, $V$, распределен равномерно от нуля до Машиного результата.
\begin{enumerate}
\item Найдите $\P(M>2V)$, $\P(M>V+0.1)$
\item Зачёт получают те, чей результат больше $0.4$. Какова вероятность того, что Вовочка получит зачёт? Какова вероятность того, что Вовочка получит зачёт, если Маша получила зачёт?
\end{enumerate}
Подсказка: попробуйте нарисовать нужные события в осях $(V,M)$

Это была задачка-неберучка!

\item Функция плотности случайной величины $X$ имеет вид $f(x)=\left\{\begin{array}{l}
\frac{3}{7}x^2,\, x\in[1;2] \\
0,\, x\notin [1,2]
\end{array}\right.$
\begin{enumerate}
\item Не производя вычислений найдите $\int_{-\infty}^{+\infty}f(x)\,dx$
\item Найдите $\E(X)$, $\E(X^2)$ и дисперсию $\Var(X)$
\item Найдите $\P(X>1.5)$
\item Найдите функцию распределения $F(x)$ и постройте её график
\end{enumerate}

\item Совместное распределение случайных величин $X$ и $Y$ задано таблицей

\begin{tabular}{@{}cccc@{}}
\toprule
      & $X=-2$ & $X=0$ & $X=2$ \\ \midrule
$Y=1$ & $0.2$  & $0.3$ & $0.1$ \\
$Y=2$ & $0.1$  & $0.2$ & $a$   \\ \bottomrule
\end{tabular}

\begin{enumerate}
\item Определите неизвестную вероятность $a$.
\item Найдите вероятности $\P(X>-1)$, $\P(X>Y)$
\item Найдите математические ожидания $\E(X)$, $\E(X^2)$
\item Найдите корреляцию $\Corr(X,Y)$
\end{enumerate}

\item Винни Пух собрался полакомиться медом, но ему необходимо принять решение, к каким пчелам отправиться за медом. Неправильные пчелы кусают каждого, кто лезет к ним на дерево с вероятностью 0.9, но их всего 10 штук. Правильные пчелы кусаются с вероятностью 0.1, но их 100 штук.
\begin{enumerate}
\item  Определите математическое ожидание и дисперсию числа укусов Винни Пуха для каждого случая
\item Определите наиболее вероятное число укусов и его вероятность для каждого случая
\item К каким пчелам следует отправиться Винни Пуху, если он не может выдержать больше двух укусов?
\end{enumerate}
\end{enumerate}



\subsection{Контрольная работа №1, 14.11.2012, решения}

\begin{enumerate}
\item
\begin{enumerate}
\item $\P(A)=0.8\cdot 0.3+0.7\cdot 0.2=0.38$
\item $\P(B)=0.9$
\item $\P(C|A)=\frac{0.3\cdot 0.8}{0.38}=0.632$
\item $\P(C|D)=\frac{0.3\cdot (0.9\cdot 0.8+0.1\cdot 0.2)}{0.9\cdot 0.38+0.1\cdot (1-0.38)}=0.55$
\end{enumerate}
\item Это была задачка-неберучка!
\item
\begin{enumerate}
\item $1$
\item $\E(X)=45/28\approx 1.61$, $\E(X^2)=93/35\approx 2.66$, $\Var(X)=291/3920\approx 0.07$
\item $37/56\approx 0.66$
\item $F(x)=\begin{cases} 0,\, x<1 \\
\frac{x^3-1}{7},\, x\in [1;2] \\
1,\, x>1 \end{cases}$
\end{enumerate}
\item
\begin{enumerate}
\item $a=0.1$
\item $\P(X>-1)=0.7$, $\P(X>Y)=0.1$
\item $\E(X)=-0.2$, $\E(X^2)=2$
\item $\Corr(X,Y)=0.117$
\end{enumerate}
\item
\begin{enumerate}
\item Правильные: $\E(X)=10$, $\Var(X)=9$, Неправильные: $\E(Y)=9$, $\Var(Y)=0.9$
\item Наиболее вероятное число укусов равно математическому ожиданию
\item Лучше идти к неправильным пчёлам, так как $\P(X\leq 2)<\P(Y\leq 2)$.
\end{enumerate}
\end{enumerate}


\subsection{Контрольная работа №2, 26.12.2012}

Тест:

\begin{enumerate}
\item Зная распределение компонент случайного вектора всегда можно восстановить их совместное распределение. Да. Нет.
\item Пусть $X$ — длина наугад выловленного удава в сантиметрах, а $Y$ — в дециметрах. Коэффициент корреляции между этими величинами равен 0.1. Да. Нет.
\item Для любой случайной величины $X$ справедливо неравенство
\[ \P(|X-\E(X)|>2\sqrt{\Var(X)})\leq 1/4 \]
Да. Нет.
\item Сумма независимых нормальных случайных величин нормальна. Да. Нет.
\item  Сумма $n$ независимых равномерно распределенных на интервале $(0,1)$ случайных величин асимптотически нормальна. Да. Нет.
\item Квадрат стандартной нормальной случайной величины имеет хи-квадрат распределение. Да. Нет.
\item Если ковариация компонент случайного двумерного нормального вектора равна нулю, то они независимы. Да. Нет.
\item Условная дисперсия всегда больше безусловной. Да. Нет.
\item Элементы выборки без возвращения из конечной генеральной совокупности независимы. Да. Нет.
\item Математическое ожидание выборочного среднего одинаково распределенных случайных величин не зависит от объема выборки. Да. Нет.
\item Конец света по техническим причинам переносится на показ работ по теории вероятности. Да. Может быть.
\end{enumerate}

Ответы: Нет, Нет, Да, Да, Да, Да, Да, Нет, Нет (зависимы), Да (не зависит), Любой верный

Задачи:
\begin{enumerate}
\item Купчиха Сосипатра Титовна очень любит чаёвничать. Её чаепитие продолжается случайное время $S$, имеющее равномерное распределение от 0 до 3 часов.  Встретив Сосипатру Титовну в пассаже на Петровке, её подруга Олимпиада Карповна узнала, сколько длилось вчерашнее чаепитие Сосипатры Титовны. Решив, что такая продолжительность чаепития является максимально возможной, Олимпиада Карповна устраивает чаепитие, продолжающееся случайное время $T$, имеющее равномерное распределение от 0 до $S$ часов.
\begin{enumerate}
\item Найдите совместную функцию плотности величин $S$ и $T$
\item Найдите вероятность $\P(S>T)$
\item Найдите $\E(T^2)$
\end{enumerate}

\item  Для случайно выбранного домохозяйства случайные величины $X$ и $Y$ принимают значения, равные доле расходов на продукты питания и алкоголь плюс табак соответственно. Случайный вектор $(X,Y)^T$  хорошо описывается двумерным нормальным законом распределения с математическим ожиданием $(0.45, 0.16)^T$ и ковариационной матрицей
\[
C=0.144\cdot
\left(\begin{array}{cc}
1 & -0.9 \\
-0.9 & 1
\end{array}\right)
\]
Найдите:
\begin{enumerate}
\item Вероятность того, что домохозяйство тратит более половины своих доходов на питание.
\item Вероятность того, что домохозяйство тратит более половины своих доходов на алкогольную и табачную продукцию и продукты питания.
\item Ожидаемую долю расходов на алкоголь и табак для домохозяйства, которое тратит на питание четверть своих доходов.
\item Вероятность того, что домохозяйство из предыдущего пункта тратит более трети своих доходов на алкогольную и табачную продукцию.

% из-за коррелированности этот пункт не решается :(
\begin{comment}
\item  Характеристикой отклонения от типичного потребления выступает величина
\[
U=\sqrt{\frac{(X-\E(X))^2}{\Var(X)}+\frac{(Y-\E(Y))^2}{\Var(Y)}}
\]
Найдите $\P(U>3)$.
\end{comment}
\item Для доли расходов на питание вычислите центральный момент 2013-го порядка.

\end{enumerate}




\item Вычислите (или оцените) вероятность того, что по результатам 4000 бросаний симметричной монеты,  частота выпадения герба будет отличаться от 0.5 не более, чем на $0.01$. Решите задачу с помощью неравенства Чебышёва и с помощью ЦПТ.





\item Компания кабельного телевидения НВТ, Новая Вершина Телевидения, анализирует возможность присоединения к своей сети пригородов N-ска. Опросы показали, что в среднем каждые 3 из 10 семей жителей пригородов хотели бы стать абонентами сети. Стоимость работ, необходимых для организации сети в любом пригороде оценивается величиной 2\,080\,000 у.е. При подключении каждого пригорода НВТ надеется получить 1\,000\,000 у.е. в год от рекламодателей. Планируемая чистая прибыль от оплаты за кабельное телевидение одной семьей в год равна 120 у.е.

Каким должно быть минимальное количество семей в пригороде для того, чтобы с вероятностью 0.99 расходы на организацию сети в этом пригороде окупились за год?


\item Оценки за контрольную работу по теории вероятностей 6 случайно выбранных студентов оказались равны:

8 5 6 7 3 9.

\begin{enumerate}
\item Выпишите вариационный ряд
\item Постройте график выборочной функции распределения
\item Вычислите значение выборочного среднего и выборочной дисперсии.
\end{enumerate}
\end{enumerate}

\subsection{Контрольная работа №2, 26.12.2012, решения}
\begin{enumerate}
\item  $f(s,t)=f(s)\cdot f(t|s)=\frac{1}{3s}$ при $0\leq t\leq s\leq 3$. Бонус тем, кто прочитал условие, $\P(S>T)=1$.
\[ \E(T^2)=\int_0^3\int_0^s \frac{t^2}{3s}\,dt\,ds=1 \]
\item
\begin{enumerate}
\item $\P(X>0.5)=\P(Z>0.1317616)\approx 0.4475864$, $\sigma_X\approx 0.3794733$
\item $\P(X+Y>0.5)=
\P(Z>-0.6481812)\approx 0.7415661$,
$\sigma_{X+Y}=0.1697056$, $\E(X+Y)=0.61$

\item $X=0.25$ при нормировке даёт $\tilde{X}=-0.5270463$. Получаем: $\E(\tilde{Y}\mid \tilde{X}=-0.5270463)=0.4743416$,  $\Var(\tilde{Y}\mid \tilde{X}=-0.5270463)=0.19$.

Значит $\E(Y \mid \tilde{X}=-0.5270463)=0.34$,  $\Var(Y\mid \tilde{X}=-0.5270463)=0.02736$.

\item $\P(Y>1/3\mid \tilde{X}=-0.5270463)=\P(Z>-0.0403042)=0.5160747$
\item Ноль
\end{enumerate}
\item $\E(\hat{p})=0.5$, $\Var(\hat{p})=0.25/n=1/16000$. По Чебышёву:
\[
\P(|\hat{p}-0.5|\leq 0.01)\geq 1-\frac{\Var(\hat{p})}{0.01^2}=\ldots=0.375
\]
Используя нормальную аппроксимацию:

\[
\P(|\hat{p}-0.5|\leq 0.01)=\P(|Z|\leq 1.2649111)\approx 0.7940968
\]
\item Обозначим $N$ — количество подключенных абонентов, тогда $N\sim Bin(n,0.3)$. При больших $n$ биномиальное распределение можно заменить на нормальное, $N\sim \cN(0.3n,0.21n)$.

\[ \P(120N>1\,080\,000)=\P(N>9000)=\P\left(Z>\frac{9000-0.3n}{\sqrt{0.21n}}\right)=0.99 \]

Из таблицы находим, что

\[ \frac{9000-0.3n}{\sqrt{0.21n}}=-2.3263479\]



Решаем квадратное уравнение, находим корни, один — отрицательный, другой, $n\approx 30622$.
\item



Вариационный ряд: 3, 5, 6, 7, 8, 9. $\bar{X}\approx 6.3333333$,
$\frac{\sum (X_i-\bar{X})^2}{n-1}\approx 4.6666667$,
$\frac{\sum (X_i-\bar{X})^2}{n}\approx 3.8888889$

\begin{minipage}{0.6\textwidth}
\begin{center}
\includegraphics[scale=0.5]{auto_figures_tikz/2012_2013_fig_02_empirical_dist.pdf}
\end{center}
\end{minipage}


\end{enumerate}

\subsection{Демо-версия зачёта}

\begin{enumerate}
\item Двое подельников, Маша и Саша, украли десять миллионов евро. Через некоторое время Саша был найден убитым, а Маша была арестована. Из свидетельских показаний ясно следует, что Маша и Саша ругались по поводу делёжки. Защита и обвинение выясняют, убила ли Маша Сашу. Из статистических данных известно, что:
\begin{enumerate}
\item[A] 20\% подельников-мужчин ругаются по поводу делёжки
\item[B] 20\% оставшихся в живых подельников-мужчин ругаются по поводу делёжки
\item[C] 5\% мужчин убивают
\item[D] 3\% мужчин убивают их подельники
\item[E] 90\% убитых мужчин-подельников ругались по поводу делёжки
\end{enumerate}
Располагая этой информацией,
\begin{enumerate}
\item Найдите вероятность того, что Маша убила Сашу, если известно, что Маша и Саша ругались по поводу делёжки.
\item Найдите вероятность того, что Маша убила Сашу, если известно, что Маша и Саша ругались по поводу делёжки, и Саша был найден убитым.
\end{enumerate}


\item Маша подкидывает 300 игральных кубиков. Те, что выпали не на шестёрку, она перекидывает один раз. Обозначим буквой $N$ количество шестёрок на всех кубиках после возможных перекидываний.
\begin{enumerate}
\item Найдите $\E(N)$, $\Var(N)$
\item Какова примерно вероятность того, величина $N$ лежит от 50 до 70?
\item Укажите любой интервал, в который величина $N$ попадает с вероятностью 0.9
\end{enumerate}

\item На лукоморье набегают волны. Кот Учёный заметил, что размер каждой волны, $X_i$, — случайная величина, имеющая равномерное распределение от 0 до 1, а размеры волн независимы. Кот Учёный считает волну \textit{большой}, если она больше предыдущей и следующей. Случайная величина $R_i$ равна 1, если $i$-ая волна была \textit{большой}, и 0 иначе.
\begin{enumerate}
\item Найдите $\P(R_i=1)$, $\E(X_i)$
\item Найдите $\E(X_i \mid R_i=1)$
\item Найдите $\Cov(R_1,R_2)$, $\Cov(R_1,R_3)$
\end{enumerate}
\item Ермолай Лопахин решил приступить к вырубке вишневого сада. Однако выяснилось, что растут в нём не только вишни, но и яблони. Причём, по словам Любови Андреевны Раневской, среднее количество деревьев (а они периодически погибают от холода или жары, либо из семян вырастают новые) в саду распределено в соответствии с нормальным законом ($X$ — число яблонь, $Y$ — число вишен) со следующими параметрами:
\begin{equation*}
\begin{pmatrix}	X \\ 	Y 	\end{pmatrix}
\sim \mN
\left(
\begin{pmatrix}
25 \\ 125
\end{pmatrix}
;
\begin{pmatrix}
    5 & 4 \\
    4 & 10
    \end{pmatrix}
\right)
\end{equation*}

\begin{enumerate}
\item Найдите вероятность того, что Ермолаю Лопахину придется вырубить более 150~деревьев.
\item Каково ожидаемое число подлежащих вырубке вишен, если известно, что предприимчивый и последовательный Лопахин, не затронув ни одного вишнёвого дерева, начал очистку сада с яблонь и все 35~яблонь уже вырубил? Какова при этом вероятность того, что Лопахину придется вырубить более 100 вишен?
\end{enumerate}

\item Вопрос из интервью в Морган-Стэнли. Есть две независимых равномерных на отрезке $[0;1]$ случайных величины, $X$ и $Y$. Как их нужно преобразовать, чтобы корреляция между ними оказалась равна $\rho$?
\end{enumerate}

\subsection{Зачёт, 15.01.2013}


\begin{enumerate}
\item Самолёт упал либо в горах, либо на равнине. Вероятность того, что самолёт упал в горах, равна 0.75. Для поиска пропавшего самолёта выделено 3 вертолёта. Каждый вертолёт можно использовать только в одном месте. Как распределить имеющиеся вертолёты, если вероятность обнаружения пропавшего самолёта отдельно взятым вертолётом равна $0.6$?

\item Совместная функция плотности величин $X$ и $Y$ имеет вид
\[
f(x,y)=\frac{1}{x}e^{-x},\: \text{при}\: 0<y<x
\]

\begin{enumerate}
\item Найдите $\P\left( \frac{Y}{X}<0.7 \right)$
\item Найдите $\E(X)$
\item Являются ли $X$ и $Y$ независимыми?
\item Как распределена величина $Z=Y/X$?
\end{enumerate}

\item Величины $X_1$, \ldots, $X_n$ независимы и имеют биномиальное распределение, $X_i \sim Bin(10,p)$. Используя неравенство Чебышёва найдите наименьшее число $t$, чтобы выполнялось условие
\[
\P( | \bar{X}-\E(\bar{X}) | \geq t )\leq 0.01
\]



\item Допустим, что срок службы пылесоса имеет экспоненциальное распределение. В среднем один
пылесос бесперебойно работает 10 лет. Завод предоставляет гарантию 7 лет на свои изделия.
Предположим для простоты, что все потребители соблюдают условия гарантии.
\begin{enumerate}
\item Какой процент потребителей в среднем обращается за гарантийным ремонтом?
\item Какова вероятность того, что из 1000 потребителей за гарантийным ремонтом обратится
более 55\% покупателей?
\end{enumerate}
Подсказка: $\ln 2\approx 0.7$

\item Вася попадает мячом в корзину с вероятностью $0.2$, Петя — с вероятностью $0.25$. Каждый из них сделал по 100 бросков мяча.
\begin{enumerate}
\item Какова вероятность того, что Петя попал на 10 раз больше Васи?
\item Какое минимальное количество бросков мяча нужно сделать каждому, чтобы вероятность, того, что Петя попал на 10 раз больше Васи достигла 0.99?
\end{enumerate}
\end{enumerate}

\subsection{КоКо, компьютерная контрольная №3, 13.03.13}

Продолжительность 1 час 10 минут, разрешено пользоваться конспектами, книжками, заготовками программ, нельзя общаться, использовать Интернет. Текст работы в группах с R:

\begin{enumerate}
\item Величины $X$ и $Y$ независимы. Величина $X$ распределена нормально, $X\sim \cN(4.2,6.7)$, величина $Y$ распределена экспоненциально, $Y\sim \exp(\lambda=2.7)$. Используя симуляционный подход примерно посчитайте

\begin{enumerate}
\item $\P(X+Y>5.6)$
\item $\E(X/(X+3Y))$
\item $\Var(XY)$
\item $\Cov(XY,X/Y)$
\end{enumerate}

\item Загрузите данные по стоимости квартир в Москве, \href{http://goo.gl/zL5JQ}{goo.gl/zL5JQ}, в табличку с именем \verb|h|. Обозначим буквой \verb|a| ответ на первый вопрос первой задачи. Отберите индивидуальную выборку лично для себя, выполнив команды:
\begin{knitrout}
\definecolor{shadecolor}{rgb}{0.969, 0.969, 0.969}\color{fgcolor}\begin{kframe}
\begin{alltt}
\hlkwd{set.seed}\hlstd{(}\hlkwd{round}\hlstd{(}\hlnum{100} \hlopt{*} \hlstd{a))} \hlcom{# здесь "a" — это ответ на первый пункт первой задачи}
\hlstd{h} \hlkwb{<-} \hlstd{h[}\hlkwd{sample}\hlstd{(}\hlnum{1}\hlopt{:}\hlkwd{nrow}\hlstd{(h),} \hlnum{1000}\hlstd{), ]}
\end{alltt}
\end{kframe}
\end{knitrout}

Постройте 90\%-ый доверительный интервал для:
\begin{enumerate}
\item Доли кирпичных домов, \verb|brick==1|
\item Доли кирпичных домов, \verb|brick==1|, среди домов находящихся близко от метро,  \verb|walk==1|
\item Разницы доли кирпичных домов среди домов расположенных близко и далеко от метро
\end{enumerate}


\item Сгенерируйте искусственные данные, выполнив команды:
\begin{knitrout}
\definecolor{shadecolor}{rgb}{0.969, 0.969, 0.969}\color{fgcolor}\begin{kframe}
\begin{alltt}
\hlkwd{set.seed}\hlstd{(}\hlkwd{round}\hlstd{(}\hlnum{100} \hlopt{*} \hlstd{a)} \hlopt{+} \hlnum{42}\hlstd{)} \hlcom{# здесь "a" — это ответ на первый пункт первой задачи}
\hlstd{x} \hlkwb{<-} \hlkwd{rexp}\hlstd{(}\hlnum{200}\hlstd{,} \hlkwc{rate} \hlstd{=} \hlnum{2}\hlstd{)}
\end{alltt}
\end{kframe}
\end{knitrout}

Величины $X_i$ независимы и имеют функцию плотности $f(x)=e^{b-xe^b}$ при $x>0$.
\begin{enumerate}
\item Оцените неизвестный параметр $b$
\item Оцените дисперсию полученной оценки
\item Постройте 90\%-ый доверительный интервал для $b$
\item Используя результат предыдущего пункта, на 10\%-ом уровне значимости проверьте гипотезу $H_0$: $b=0.7$ против альтернативной гипотезы $H_a$: $b\neq 0.7$.
\end{enumerate}

\end{enumerate}


\subsection{Экзамен, 26.03.2013}


\begin{enumerate}
%в комментариях предполагаемые ответы


\item Вероятность выигрыша по лотерейному билету равна $0.05$. Вероятность того, что из трёх купленных билетов ровно два окажутся выигрышными примерно равна

\otvet{0.002}{0.0025}{0.007}{0.1}{0.3}

\item Закон распределения случайной величины задан табличкой

\begin{tabular}{@{}cccc@{}}
\toprule
$x$         & $-1$  & $0$   & $1$ \\ \midrule
$\P(X=x)$ & $0.4$ & $0.2$ & ?   \\ \bottomrule
\end{tabular}

Дисперсия величины $X$, $\Var(X)$, равняется

\otvet{0}{0.02}{0.3}{0.8}{2}

%3
\item Если $f(x)$ — функция плотности, то $\int_{-\infty}^{x}f(u)\,du$ равен

\otvet{0}{1}{$\E(X)$}{$\Var(X)$}{$F(x)$}

%4
\item События $A$ и $B$ называются независимыми, если

\lotvet{$\P(A\cup B)=\P(A)+\P(B)$}
{$\P(A)\cdot\P(B)=\P(A\cap B)$}
{$\P(A\cup B)=\P(A)+\P(B)-\P(A\cap B)$}
{$\P(A\cap B)=0$}
{нет верного} \\ \\

%5
\item Правильную монетку подбрасывают два раза. Рассмотрим два события: $A$ — при втором броске выпала «решка», $B$ — «орёл» выпал хотя бы один раз. Найдите $\P(A|B)$

\otvet{0}{1/3}{1/2}{2/3}{1}

%6
\item Есть пять случайных величин: $X\sim \chi^2_{10}$, $Y\sim F_{5,10}$, $T\sim t_{10}$, $Z\sim \cN(0,1)$, $W\sim \cN(10,1)$. Какие из величин распределены симметрично относительно 0?

\otvet {X, Y, Z}{Z, W}{Z, T}{Z}{X, Y}

%7
\item Известно, что $\E(X)=3$, $\Var(X)=1$, $\E(Y)=4$, $\Var(Y)=9$, $\E(XY)=13$, найдите $\Cov(X,Y)$

\otvet{0}{-3}{18}{3}{1}

%8
\item Известно, что $\E(X)=3$, $\Var(X)=1$, $\E(Y)=4$, $\Var(Y)=9$, $\E(XY)=6$, найдите $\Var(2X+Y)$

\otvet{13}{7}{1}{17}{нет верного ответа}

%9
\item Если $F(x)$ — это функция распределения, то $\lim_{x\to -\infty}F(x)$ равен

\otvet{0}{0.5}{1}{$\E(X)$}{$+\infty$}

%10
\item Если $X\sim \cN(-4;1)$, то $\P(3X+571>0)$ примерно равна

\otvet{0}{0.5}{1}{$+\infty$}{нет верного ответа}

%11
\item Про закон распределения величины $X$ ничего не известно. Укажите самую точную оценку сверху для вероятности $\P(|X-\E(X)|>3\sqrt{\Var(X)})$

\otvet{$0.(3)$}{$0.6(3)$}{$0.(1)$}{$1$}{нет верного ответа}

%12
\item Функция распределения, $F(x)=\P(X\leq x)$ может не являться

\otvet{непрерывной}{непрерывной справа}{монотонно неубывающей}{ограниченной}{неотрицательной}

%13
\item Ковариационной может быть матрица:

\otvet {$\left(\begin{array}{cc}
-1 & 1\\
1 & 2
\end{array}\right)$}{$\left(\begin{array}{cc}
1 & 0.5\\
1 & 2
\end{array}\right)$}{$\left(\begin{array}{cc}
1 & -1\\
-1 & 2
\end{array}\right)$}{$\left(\begin{array}{cc}
-1 & 1\\
1 & -2
\end{array}\right)$}{$\left(\begin{array}{cc}
1 & 1\\
-0.7 & 2
\end{array}\right)$}\\

%14
\item Если $X$ и $Y$ независимые случайные величины, то неверным может быть утверждение

\lotvet{$\E(X+Y)=\E(X)+\E(Y)$}
{$\E(X/Y)=\E(X)/\E(Y)$}
{$\E(XY)=\E(X)\cdot\E(Y)$}
{$\Var(X+Y)=\Var(X)+\Var(Y)$}
{$\Cov(X,Y)=0$} \\ \\

%15
\item Известно, что  $\Cov(X,Y)=0$, $\Var(X)=10$, $\Var(Y)=10$. \\
Неверным может быть утверждение

\lotvet{$\Corr(X,Y)=0$}
{$\Corr(X+a,Y+b)=0$}
{$\E(X\cdot Y)=\E(X)\cdot \E(Y)$}
{$\Var(X+Y)=\Var(X)+\Var(Y)$}
{$X$ и $Y$ независимы} \\ \\

%16
\item $Z_1,Z_2,\ldots,Z_n\sim\cN(0,1)$. Тогда величина $\frac{Z_1\sqrt{n-2}}{\sqrt{\sum_{i=3}^n Z_i^2}}$ имеет распределение

\otvet {$\cN(0,1)$}{$t_n$}{$F_{1,n-2}$}{$\chi^2_n$}{$t_{n-2}$}

%17
\item Количество страниц в книгах авторов $X$ и $Y$ распределено нормально с дисперсиями $\sigma^2_X$ и $\sigma^2_Y$ соответственно. Для тестирования гипотезы о равенстве дисперсий было выбрано $n$ книг автора $X$ и $m$ книг — автора $Y$. Какое распределение имеет статистика, используемая в данном случае?

\otvet {$\chi^2_{\min(m,n)}$}{$\chi^2_{\max(m,n)}$}{$F_{m,n}$}{$F_{m-1,n-1}$}{$F_{m+1,n+1}$}

%18
\item Если величина $X$ имеет $\chi^2_k$-распределение, величина $Y$ — $\chi^2_n$-распределение, и они независимы, то дробь $nX/(kY)$ имеет распределение

\otvet{$F_{k,n}$}{$F_{n,k}$}{$F_{k-1,n-1}$}{$\chi^2_{n-k}$}{$F_{n-1,k-1}$}

%19
\item \emph{Смещённой} оценкой математического ожидания по выборке независимых,
одинаково распределенных случайных величин $X_1$, $X_2$, $X_3$ является оценка

\lotvet{$(X_1+X_2)/2$}{$(X_1+X_2+X_3)/3$}{$0.7X_1+0.2X_2+0.1X_3$}{$0.3X_1+0.3X_2+0.3X_3$}{$X_1+X_2-X_3$} \\ \\

%20
\item Если величины $X$ и $Y$ независимы и равномерно распределены на $[0;1]$, а $F(x,y)$ — их совместная функция  распределения, то $F(0.5,3)$ равно

\otvet{0}{0.5}{1}{1.5}{не существует}

%21
\item Если $X_i$ независимы и имеют нормальное распределение $\cN(-1;2013)$, то $\sqrt{n}(1+\bar{X})/\hat{\sigma}$ имеет распределение

\otvet{$\cN(0;1)$}{$t_{n-1}$}{$\chi^2_{n-1}$}{$\cN(0;2013/n)$}{$t_n$}

%22
\item Последовательность оценок $\hat{\theta}_1$, $\hat{\theta}_2$, \ldots называется состоятельной, если

\lotvet{$\E(\hat{\theta}_n)=\theta$}{$\Var(\hat{\theta}_n)\to 0$}{$\P(|\hat{\theta}_n - \theta |>t)\to 0$ для всех $t>0$}{$\E(\hat{\theta}_n)\to \theta$}
{$\Var(\hat{\theta}_n)\geq \Var(\hat{\theta}_{n+1})$} \\ \\

%23
\item Величины $X_1$, \ldots, $X_5$ равномерны на отрезке $[0;a]$. Известно, что $\sum_{i=1}^5 x_i=25$. При использовании первого момента оценка методом моментов неизвестного $a$ равна

\otvet{1}{5}{10}{20}{нет верного ответа}

%24
\item При построении доверительного интервала для отношения дисперсий по двум независимым нормальным выборкам из $n$ наблюдений каждая, используется статистика, имеющая распределение

\otvet{$F_{n-1,n-1}$}{$t_{n-1}$}{$\chi^2_{n-1}$}{$\chi^2_{n}$}{$t_n$}

%25
\item Ботаники\footnote{В отличие от ботаников, зоологи точно знают, сколько иголок у ёжиков!} строят доверительный интервал для математического ожидания числа иголок у ёжа. Количество иголок на одном еже предполагается нормально распределенным. Среднее число иголок у пойманных 100 ёжиков равно 1500, выборочная дисперсия — 400. На 5\% уровне значимости какой примерно доверительный интервал должны построить ботаники?


\lotvet {$[1499;1501]$}{$[1498;1502]$}{$[1497;1503]$}{$[1496;1504]$}{нет верного ответа}\\ \\

%26
\item Функция правдоподобия, построенная по случайной выборке $X_1$, \ldots, $X_n$ из распределения с функцией плотности $f(x)=(\theta+1)x^{\theta}$ при $x\in [0;1]$ имеет вид

\otvet{$(\theta+1)x^{n\theta}$}{$\sum (\theta+1)x_i^{\theta}$}
{$(\theta+1)^{\sum x_i}$}{$(\sum x_i)^{\theta}$}{$(\theta+1)^n\prod x_i^{\theta}$}

%27
\item Если $X_i$ независимы, $\E(X_i)=\mu$ и $\Var(X_i)=\sigma^2$, то математическое ожидание величины $Y=\sum_{i=1}^{n}(X_i-\bar{X})^2$ равно

\otvet{$\hat{\sigma}^2$}{$(n-1)\sigma^2$}{$\mu$}{$\sigma^2$}{$\sigma^{2}/n$}

%28
\item Если $P$-значение меньше уровня значимости $\alpha$, то гипотеза $H_0$: $\sigma=\sigma_0$

\lotvet{отвергается}{не отвергается}{отвергается только если $H_a$: $\sigma \neq \sigma_0$}{отвергается только если $H_a$: $\sigma<\sigma_0$}{недостаточно информации} \\

\item Если $H_0$ верна, то $P$-значение имеет распределение

\otvet{$U[0;1]$}{$\cN(0,1)$}{$t_n$}{$t_{n-1}$}{$\chi^2_{n-1}$}


\end{enumerate}




\textbf{Экзамен по теории вероятностей! Суперигра!}

\vspace{20pt}
\begin{center}
\textbf{DON'T PANIC}
\end{center}
\vspace{20pt}


\begin{enumerate}
\item В группе учится 30 студентов, 20 девушек и 10 юношей. Они входят в аудиторию в случайном порядке. Рассчитайте вероятности событий:
\begin{enumerate}
\item Маша Петрова\footnote{Маша Петрова — единственная и неподражаемая!} войдёт девятой по счёту
\item Девятый вошедший окажется девушкой
\item Перед Машей Петровой войдут ровно 5 юношей
\item Перед Машей Петровой войдут ровно 5 юношей, если известно, что Маша Петрова вошла девятой
\item Маша Петрова войдёт девятой по счёту, если известно, что перед ней вошло ровно 5 юношей
\end{enumerate}

\item В поселке 2500 жителей. Каждый из них в среднем 6 раз в месяц ездит в город, выбирая день поездки независимо от других людей. Поезд ходит в город один раз в сутки.
\begin{enumerate}
\item Какой наименьшей вместимостью должен обладать поезд, чтобы он переполнялся в среднем не чаще 1 раза в 100 дней?
\item Сколько в среднем человек будет ехать в таком поезде, если предположить, что при переполнении часть людей полностью откажется от поездки?
\end{enumerate}

Источник: экзамен РЭШ

\item Случайные величины $X_1$, $X_2$, \ldots, $X_n$ независимы и имеют пуассоновское распределение с неизвестным параметром $\lambda$.
\begin{enumerate}
\item С помощью метода максимального правдоподобия постройте оценки для $\lambda$ и для $\exp(\lambda)$
\item Предположим, что исследователь не знает, чему равны $X_i$. Ему известно лишь, равно ли каждое из $X_i$ нулю или нет. С помощью метода максимального правдоподобия постройте оценки для $\lambda$ и для $\exp(\lambda)$. Всегда ли существуют предложенные оценки?
\end{enumerate}


\item В таблице представлены данные по количеству пассажиров «Титаника», поделенные на группы по классу каюты:

% latex table generated in R 3.4.0 by xtable 1.8-2 package
% Tue Nov 14 09:38:58 2017
\begin{table}[ht]
\centering
\begin{tabular}{rrrr}
  \hline
 & 1 класс & 2 класс & 3 класс \\ 
  \hline
Погиб & 122 & 167 & 528 \\ 
  Выжил & 203 & 118 & 178 \\ 
   \hline
\end{tabular}
\end{table}


Проверьте гипотезу о независимости шансов выжить от класса каюты.

\item Перед Вами две внешне неотличимых монетки. Одна из них выпадает «орлом» вверх с вероятностью $0.7$, другая — с вероятностью $0.3$. Вы имеете право на 4 подбрасывания. Перед каждым подбрасыванием Вы можете выбирать подбрасываемую монетку. За каждого выпавшего «орла» вы получаете 1 рубль.
\begin{enumerate}
\item Какова оптимальная стратегия?
\item Каков ожидаемый выигрыш при использовании оптимальной стратегии?
\end{enumerate}

\end{enumerate}

\section{2013-2014}

\input{chapters/year_2013_2014.tex}

\section{2014-2015}

\input{chapters/year_2014_2015.tex}

\section{2015-2016}


\subsection{Контрольная номер 1, базовый поток, 26.10.2015}

\begin{enumerate}
\item
Подбрасываются две симметричные монеты. Событие А — на первой монете выпал
герб, событие В — на второй монете выпал герб, событие С — монеты выпали
разными сторонами.
\begin{enumerate}
    \item[$\alpha$)] Будут ли эти события попарно независимы?
    \item[$\beta$)]  Сформулируйте определение независимости в совокупности для трех событий. Являются ли события $A$, $B$, $C$ независимыми в совокупности?
\end{enumerate}

\item
Имеются два игральных кубика:
\begin{itemize}
    \item красный со смещенным центром тяжести, так что вероятность выпадения «6»
    равняется 1/3, а оставшиеся грани имеют равные шансы на появление
    \item честный белый кубик
\end{itemize}

\begin{enumerate}
    \item[$\alpha$)] Петя случайным образом выбирает кубик и подбрасывает его. Найдите
    вероятность того, что выпадет «6».
    \item[$\beta$)]   Петя случайным образом выбирает кубик и подбрасывает его. Какова
    вероятность того, что Петя взял красный кубик, если известно, что выпала
    шестерка?
\end{enumerate}

\item
Все те же кубики. Петя играет с Васей в следующую игру: Петя выбирает кубик и
подбрасывает его. Вася подбрасывает оставшийся кубик. Выигрывает тот, у кого
выпало большее число. Если выпадает равное число очков, выигрывает тот, у кого
белый кубик.

Пусть случайная величина $\xi$ — число очков, выпавших на красном кубике, случайная величина $\eta$ — число очков,
выпавших на белом кубике, а величина $\zeta$ — максимальное число очков.

\begin{enumerate}
    \item[$\alpha$)] Задайте в виде таблицы совместное распределение величин $\xi$ и $\eta$. Отметьте (* или кружочком) все те пары значений, когда выигрывает красный кубик.
    \item[$\beta$)] Какой кубик нужно выбрать Пете, чтобы его шансы выиграть были выше?
    \item[$\gamma$)] Сформулируйте определение функции распределения и постройте функцию
    распределения величины $\zeta$.
    \item[$\delta$)] Вычислите математическое ожидание величины $\zeta$.
\end{enumerate}

\item
Проводится исследование с целью определения процента мужчин, которые любят
петь в душе. Поскольку некоторые мужчины стесняются прямо отвечать на этот
вопрос, предлагается перед ответом на вопрос: «поете ли Вы, когда принимаете
душ?» подбросить правильный кубик, и выбрать ответ «ДА», если выпала
шестерка, ответ «НЕТ», если выпала единица, и честный ответ («ДА» или «НЕТ»),
если выпала любая другая цифра.

Предположим, что по результатам исследования
вероятность ответа «ДА» составляет $2/3$. Каков истинный процент «певцов»?

\item
Ваш полный тезка страдает дисграфией. При подписывании контрольной работы
по теории вероятностей в своих имени и фамилии в именительном падеже Ваш
тезка с вероятностью 0.1 вместо нужной буквы пишет любую другую (независимо
от предыдущих ошибок).

\begin{enumerate}
    \item[$\alpha$)] Найдите вероятность того, что он напишет свою фамилию правильно.
    \item[$\beta$)] Найдите вероятность того, что он сделает ровно 2 ошибки в своем имени.
    \item[$\gamma)$] Вычислите наиболее вероятное число допущенных тезкой ошибок.
    \item[$\delta$)] Найдите вероятность того, что при подписывании работы Ваш тезка допустит хотя бы одну ошибку.
\end{enumerate}

\item
Время (в часах), за которое студенты выполняют экзаменационное задание
является случайной величиной с функцией плотности

\[
f(y) =
    \begin{cases}
        cy^2 + y , & \mbox{if } 0 \le y \le 1 \\
        0, & \mbox{else}
    \end{cases}
\]

\begin{enumerate}
    \item[$\alpha$)] Найдите константу $c$.
    \item[$\beta$)]  Найдите функцию распределения и постройте её.
    \item[$\gamma)$] Вычислите вероятность того, что случайно выбранный студент закончит работу менее чем за полчаса.
    \item[$\delta$)] Найдите медиану распределения.
    \item[$\epsilon$)] Определите вероятность того, что студент, которому требуется по меньшей мере 15 минут для выполнения задания, справится с ним более, чем за 30 минут.
\end{enumerate}

\item
Вам известно, что на большом листе бумаги $1.5$ м $\times$ $1$ м нарисован слон. Вам завязали
глаза и выдали кисточку хвоста для слона. Вам нужно прилепить эту кисточку к
листу (рисунок Вы не видели). Вы подходите к листу и произвольно приклеиваете
кисточку
\begin{enumerate}
    \item[$\alpha$)] ) Какова вероятность того, что кисточка окажется на слоне, если площадь рисунка составляет $1$ м$^2$?
    \item[$\beta$)]  Запишите вид функции совместной плотности для координат кисточки.
    \item[$\gamma)$] Запишите вид частных функций плотности для каждой из координат кисточки.
    \item[$\delta$)] Являются ли координаты кисточки независимыми случайными величинами?
    \item[$\epsilon$)] Запишите вид функции плотности суммы координат кисточки.
\end{enumerate}
\textit{Подсказка: слон не должен заслонить равномерного распределения.}

\begin{center}
\includegraphics[scale=1.5]{images/slon.jpg}
\end{center}

\item
Укажите названия букв греческого алфавита и запишите соответствующие заглавные буквы:
\[\alpha, \zeta, \eta, \theta\].

\end{enumerate}

\subsection{Контрольная номер 1, базовый поток, 26.10.2015, решения}

\begin{enumerate}
\item
\begin{enumerate}
\item[$\alpha$)] Найдём вероятности каждого события: $\P(A) = 1/2$, $\P(B) = 1/2$, $\P(C) = 1/2$.

Проверим попарную независимость:
\begin{itemize}
\item $\P(A \cap B) = 1/4$, $\P(A) \cdot \P(B) = 1/2 \cdot 1/2 = 1/4$
\item $\P(A \cap C) = 1/4$, $\P(A) \cdot \P(C) = 1/2 \cdot 1/2 = 1/4$
\item $\P(B \cap C) = 1/4$, $\P(B) \cdot \P(C) = 1/2 \cdot 1/2 = 1/4$
\end{itemize}
Значит, события попарно независимы.
\item[$\beta$)] События $A_1, A_2, A_3$ называются независимыми в совокупности, если $\P(A_1 \cap A_2 \cap A_3) = \P(A_1) \cdot \P(A_2) \cdot \P(A_3)$.

В нашем случае: $\P(A \cap B \cap C) = 0$, $ \P(A) \cdot \P(B) \cdot \P(C) =  \frac{1}{2} \cdot \frac{1}{2} \cdot \frac{1}{2} $, следовательно, события
не являются независимыми в совокупности.
\end{enumerate}
\item
\begin{enumerate}
\item[$\alpha$)] Воспользуемся формулой полной вероятности:
\begin{multline*}
\P(\text{выпала «6»}) = \P(\text{выпала «6»} \mid \text{взят белый кубик}) \cdot \P(\text{взят белый кубик}) + \\
+ \P(\text{выпала «6»} \mid \text{взят красный кубик}) \cdot \P(\text{взят красный кубик}) = \\
= \frac{1}{6} \cdot \frac{1}{2} + \frac{1}{3} \cdot \frac{1}{2} = \frac{1}{4}
\end{multline*}
\item[$\beta$)] Воспользуемся формулой условной вероятности и результатом предыдущего пункта:
\begin{multline*}
\P(\text{взят красный кубик} \mid \text{выпала «6»}) = \frac{\P(\text{взят красный кубик} \cap \text{выпала «6»})}{\P(\text{выпала «6»})} =  \\
= \frac{\frac{1}{2}\cdot \frac{1}{3}}{\frac{1}{4}} = \frac{2}{3}
\end{multline*}
\end{enumerate}
\item
\begin{enumerate}
\item[$\alpha$)] Совместное распределение имеет вид:
\begin{center}
\begin{tabular}{@{}lllllll@{}}
\toprule
$\eta$ $\backslash$ $\xi$ & $1$                            & $2$                            & $3$                            & $4$                            & $5$                            & $6$                            \\ \midrule
$1$           & $\frac{2}{15}\cdot\frac{1}{6}$ & $\frac{2}{15}\cdot\frac{1}{6}\mbox{*}$  & $\frac{2}{15}\cdot\frac{1}{6}\mbox{*}$   & $\frac{2}{15}\cdot\frac{1}{6} \mbox{*}$   & $\frac{2}{15}\cdot\frac{1}{6} \mbox{*}$   & $\frac{1}{3}\cdot\frac{1}{6} \mbox{*}$   \\
$2$           & $\frac{2}{15}\cdot\frac{1}{6}$ & $\frac{2}{15}\cdot\frac{1}{6}$ & $\frac{2}{15}\cdot\frac{1}{6}\mbox{*}$   & $\frac{2}{15}\cdot\frac{1}{6}\mbox{*}$   & $\frac{2}{15}\cdot\frac{1}{6}\mbox{*}$   & $\frac{1}{3}\cdot\frac{1}{6} \mbox{*}$   \\
$3$           & $\frac{2}{15}\cdot\frac{1}{6}$ & $\frac{2}{15}\cdot\frac{1}{6}$ & $\frac{2}{15}\cdot\frac{1}{6}$ & $\frac{2}{15}\cdot\frac{1}{6} \mbox{*}$   & $\frac{2}{15}\cdot\frac{1}{6} \mbox{*}$   & $\frac{1}{3}\cdot\frac{1}{6} \mbox{*}$   \\
$4$           & $\frac{2}{15}\cdot\frac{1}{6}$ & $\frac{2}{15}\cdot\frac{1}{6}$ & $\frac{2}{15}\cdot\frac{1}{6}$ & $\frac{2}{15}\cdot\frac{1}{6}$ & $\frac{2}{15}\cdot\frac{1}{6} \mbox{*}$ & $\frac{1}{3}\cdot\frac{1}{6} \mbox{*}$   \\
$5$           & $\frac{2}{15}\cdot\frac{1}{6}$ & $\frac{2}{15}\cdot\frac{1}{6}$ & $\frac{2}{15}\cdot\frac{1}{6}$ & $\frac{2}{15}\cdot\frac{1}{6}$ & $\frac{2}{15}\cdot\frac{1}{6}$ & $\frac{1}{3}\cdot\frac{1}{6} \mbox{*}$   \\
$6$           & $\frac{2}{15}\cdot\frac{1}{6}$ & $\frac{2}{15}\cdot\frac{1}{6}$ & $\frac{2}{15}\cdot\frac{1}{6}$ & $\frac{2}{15}\cdot\frac{1}{6}$ & $\frac{2}{15}\cdot\frac{1}{6}$ & $\frac{1}{3}\cdot\frac{1}{6}$ \\ \bottomrule
\end{tabular}
\end{center}
\item[$\beta$)] $\P(\text{выиграет белый кубик}) = (6 + 5 + 4 + 3 + 2) \cdot \frac{2}{15}\cdot\frac{1}{6} + 1 \cdot \frac{1}{3}\cdot\frac{1}{6} = \frac{1}{2}$.

Значит, Пете безразлично, какой кубик брать.
\item[$\gamma)$] $F_{\zeta}(x) = \P(\zeta \leq x)$

Выпишем таблицу распределения случайной величины $\zeta$:

\begin{tabular}{@{}lcccccc@{}}
\toprule
$\zeta$     & $1$                              & $2$                                      & $3$                                      & $4$                                      & $5$                                      & $6$                                                                              \\ \midrule
$\P(\cdot)$ & $\frac{2}{15} \cdot \frac{1}{6}$ & $\frac{2}{15} \cdot \frac{1}{6} \cdot 3$ & $\frac{2}{15} \cdot \frac{1}{6} \cdot 5$ & $\frac{2}{15} \cdot \frac{1}{6} \cdot 7$ & $\frac{2}{15} \cdot \frac{1}{6} \cdot 9$ & $\frac{1}{3} \cdot \frac{1}{6} \cdot 6 + \frac{2}{15} \cdot \frac{1}{6} \cdot 5$ \\ \bottomrule
\end{tabular}

Тогда функция распределения имеет вид:
\[
F_{\zeta}(x) =
\begin{cases}
0 & x \leq 1 \\
\frac{1}{45} & 1 < x \leq 2 \\
\frac{4}{45} & 2 < x \leq 3 \\
\frac{9}{45} & 3 < x \leq 4 \\
\frac{16}{45} & 4 < x \leq 5 \\
\frac{25}{45} & 5 < x \leq 6 \\
1 & x > 6
\end{cases}
\]
\item[$\delta$)] $\E(\zeta) = \frac{2}{15} \cdot \frac{1}{6} \cdot 1 + \frac{2}{15} \cdot \frac{1}{6} \cdot 3 \cdot 2 + \frac{2}{15} \cdot \frac{1}{6} \cdot 5 \cdot 3 + \frac{2}{15} \cdot \frac{1}{6} \cdot 7 \cdot 4 + \frac{2}{15} \cdot \frac{1}{6} \cdot 9 \cdot 5 + \frac{1}{3} \cdot \frac{1}{6} \cdot 6 + \frac{2}{15} \cdot \frac{1}{6} \cdot 6 = \frac{43}{9} \approx 4.8 $
\end{enumerate}
\item Пусть $x$ — вероятность того, что мужчина честно любит петь в душе.

Распишем по формуле полной вероятности вероятность получить ответ «да»:
\begin{multline*}
P(\text{ответ «Да»}) = 1 \cdot \P(\text{выпала «6»}) + x \cdot(\P(\text{выпала «2»}) + \P(\text{выпала «3»}) +  \\
+ \P(\text{выпала «4»}) + \P(\text{выпала «5»})) = 1 \cdot \frac{1}{6} + x \cdot \frac{4}{6} \Rightarrow x = \frac{3}{4}
\end{multline*}
Тогда истинный процент «певцов» составляет $75 \%$
\item Предположим, что ваше имя — Студент (7 букв), а фамилия — Идеальный (9 букв).
\begin{enumerate}
\item[$\alpha$)] $\P(\text{напишет фаимлию правильно}) = (0.9)^9$
\item[$\beta$)] $\P(\text{ровно 2 ошибки в имени}) = C_{7}^2 \cdot 0.1^2 \cdot 0.9^5$
\item[$\gamma$)] Наиболее вероятное число ошибок — 1
\item[$\delta$)] $\P(\text{допустит хотя бы одну ошибку}) = 1 - \P(\text{не допустит ни одной ошибки}) = 1 - (0.9)^{16}$
\end{enumerate}
\item
\begin{enumerate}
\item[$\alpha$)] Из условия $\int_{0}^{1} (cy^2 + y) dy = 1$ получаем, что $c=3/2$.
\item[$\beta$)]
$F_{Y} (y) =
\begin{cases}
1 & y > 1 \\
\frac{y^3 + y^2}{2} & 0 < y \leq 1 \\
0 & y < 0
\end{cases} $
\item[$\gamma$)] $\P(Y < 0.5) = \int_{0}^{0.5} \left(\frac{3}{2} y^2 + y   \right) dy = \frac{3}{16}$
\item[$\delta$)] $F_{Y} (y) = 0.5 \Rightarrow y \approx 0.75 $
\item[$\epsilon$)] $\P(Y > 0.5 \mid Y \geq 0.25) = \frac{\P(Y > 0.5)}{\P(Y \geq 0.25)} = \frac{1 - \frac{3}{16}}{\int_{0.25}^{1} \left(\frac{3}{2} y^2 + y   \right) dy} = \frac{104}{123}$
\end{enumerate}
\item
\begin{enumerate}
\item[$\alpha$)] $\P(\text{кисточка окажется на слоне}) = \frac{1}{1.5} = \frac{2}{3}$
\item[$\beta$)] $f_{\xi, \eta}(x, y) = \frac{1}{1.5}$
\item[$\gamma$)] $f_{\xi} (x) = \int_{0}^{1} \frac{1}{1.5} dy = 1.5$

$f_{\eta}(y) = \int_{0}^{1.5} \frac{1}{1.5} dx = 1$
\item[$\delta$)] Да, поскольку $ f_{\xi} (x) \cdot f_{\eta}(y) = f_{\xi, \eta}(x, y)$
\item[$\epsilon$)] $f_{\xi+\eta} (t) = \int_{-\infty}^{+\infty} f_{\xi}(u) f_{\eta}(t-u) du $
\end{enumerate}
\end{enumerate}


\subsection{Праздник номер 1, исследователи, индивидуальный тур}

\begin{enumerate}
\item Для разминки вспомним греческий алфавит!

\begin{enumerate}
\item По-гречески — Σωκρατης, а по-русски — \underline{\hspace{2cm}}
\item Изобразите прописные и строчные буквы: эта \underline{\hspace{2cm}}, дзета \underline{\hspace{2cm}}, вега \underline{\hspace{2cm}}, шо \underline{\hspace{2cm}}. Если такой буквы в греческом нет, то поставьте прочерк.
\item Назовите буквы: τ \underline{\hspace{2cm}}, θ \underline{\hspace{2cm}}, ξ \underline{\hspace{2cm}}.
%\item Если пересчитать все буквы в греческом алфавите, то их окажется ровно \underline{\hspace{2cm}} %24
\end{enumerate}

\item Подбрасываются 2 симметричные монеты. Событие $A$ — на первой монете выпал герб, событие $B$ — на второй монете выпал герб, событие $C$ — монеты выпали разными сторонами.
\begin{enumerate}
\item Будут ли эти события попарно независимы?
\item Сформулируйте определение независимости в совокупности для трех событий
\item Являются ли события $A$, $B$, $C$ независимыми в совокупности?
\end{enumerate}


\item Имеются два игральных кубика: \textbf{красный} со смещенным центром тяжести, так что вероятность выпадения «6» равняется 1/3, а оставшиеся грани имеют равные шансы на появление и
правильный \textbf{белый} кубик.  Петя случайным образом выбирает кубик и подбрасывает его.
\begin{enumerate}
\item Вероятность того, что выпадет «6», равна \underline{\hspace{2cm}}
\item Вероятность того, что Петя взял красный кубик, если известно, что выпала шестерка, равна \underline{\hspace{2cm}}
\item Если бы в эксперименте Петя подбрасывал  бы кубик не один раз, а 60 раз, то безусловное математическое ожидание количества выпавших шестёрок равнялось бы \underline{\hspace{2cm}}
\end{enumerate}


\begin{comment}
\item Неразменный пятак всегда выпадает «орлом». У Александра Привалова в кармане один неразменный пятак и два обычных, равновероятно выпадающих «орлом» и «решкой». Привалов достаёт одну из монет наугад не глядя.
\begin{enumerate}
\item Вероятность того, что он достанет неразменный пятак равна \underline{\hspace{2cm}} % 1/3
\item Не глядя на монету, Привалов подкидывает её. Вероятность того, что она выпадет  «орлом», равна \underline{\hspace{2cm}} % 2/3
%\item Если бы эту случайную монету подкинуть не один раз, а 10, то математическое ожидание числа «орлов» равнялось бы \underline{\hspace{2cm}} % 20/3
\item Наконец Привалов глядит на упавшую монету и видит, что выпал «орёл». Вероятность того, что монета — неразменный пятак, равна \underline{\hspace{2cm}}
\end{enumerate}
\end{comment}

\item Винни-Пуху снится сон, будто он спустился в погреб, а там бесконечное количество горшков. Каждый из них независимо от других может оказаться либо пустым с вероятностью $0.8$, либо с мёдом с вероятностью $0.2$. Винни-Пух начинает перебирать горшки по очереди в поисках полного. Хотя у него в голове и опилки, Винни-Пух два раза в один и тот же горшок заглядывать не будет.
\begin{enumerate}
\item Вероятность того, что все горшки окажутся пустыми равна \underline{\hspace{2cm}}
\item Вероятность того, что полный горшок будет найден ровно с шестой попытки, равна \underline{\hspace{2cm}}
\item Вероятность того, что полный горшок будет найден на шестой попытке или ранее, равна \underline{\hspace{2cm}}
%\item Математическое ожидание числа перебранных горшков равняется \underline{\hspace{2cm}} % 5
\end{enumerate}

\item На самом деле у Винни-Пуха в погребе стоит 10 горшков. Каждый из них независимо от других может оказаться либо пустым с вероятностью $0.8$, либо с мёдом с вероятностью $0.2$.
\begin{enumerate}
\item Все десять горшков окажутся пустыми с вероятностью \underline{\hspace{2cm}}
\item Ровно $7$ горшков из десяти окажутся пустыми с вероятностью \underline{\hspace{2cm}}
\item Математическое ожидание числа горшков с мёдом равно \underline{\hspace{2cm}}
\end{enumerate}


\begin{comment}
\item Внутри треугольника с вершинами $(0,0)$, $(2,5)$ и $(8,0)$ случайно равномерно по площади выбирается точка. Пусть $X$ и $Y$ — абсцисса и ордината этой случаной точки.
\begin{enumerate}
\item Вероятность того, что $X>5$ равна \underline{\hspace{2cm}}.
\item Вероятность того, что $X>5$ и одновременно $Y<3$ равна \underline{\hspace{2cm}}.
\item Вероятность того, что $X>5$ если известно, что $Y<3$ равна \underline{\hspace{2cm}}.
\item События $X>5$ и $Y<3$ являются \underline{\hspace{1cm}}висимыми.
\item Функция плотности величины $X$ равна \underline{\hspace{2cm}}
\end{enumerate}
\end{comment}

\item В галактике Флатландии все объекты двумерные. На планету Тау-Слона (окружность) в случайных точках независимо друг от друга садятся три корабля. Любые два корабля могут поддерживать прямую связь между собой, если центральный угол между ними меньше прямого.

\begin{enumerate}
\item Вероятность того, что первый и второй корабли могут поддерживать прямую связь равна \underline{\hspace{2cm}}
\item Вероятность того, что все корабли смогут поддерживать прямую связь друг с другом равна \underline{\hspace{2cm}}
\item Вероятность того, что все корабли смогут поддерживать прямую связь друг с другом, если первый и второй корабль могут поддерживать прямую связь, равна \underline{\hspace{2cm}}
\end{enumerate}
Подсказка: во Флатландии хватит рисунка на плоскости, ведь координату третьего корабля можно принять за\ldots



\item Время (в часах), за которое студенты выполняют экзаменационное задание является случайной величиной $X$ с функцией плотности
\[
f(x)=\begin{cases}
3x^2, \, \text{ если } x \in [0;1] \\
0, \, \text{ иначе }
\end{cases}
\]

\begin{enumerate}
\item Функция распределения случайной величины $X$ равна \underline{\hspace{2cm}}
\item Вероятность того, что случайно выбранный студент закончит работу менее чем за полчаса равна \underline{\hspace{2cm}}.
\item Медиана распределения равна \underline{\hspace{2cm}}
\item Вероятность того, что студент, которому требуется по меньшей мере 15 минут для выполнения задания, справится с ним более, чем за 30 минут, равна \underline{\hspace{2cm}}
\item Функция распределения случайной величины $Y=1/X$ равна \underline{\hspace{2cm}}
\item Функция плотности случайной величины $Y=1/X$ равна \underline{\hspace{2cm}}
\end{enumerate}

\end{enumerate}

\subsection{Индивидуальный тур, решение}

\begin{enumerate}
\item Сократ, эта — Η, η, дзета — Ζ, ζ, вега — нет, шо — ϸ, τ — тау, θ — тета, ξ — кси.
Греческая буква шо, ϸ, была введена Александром Македонским и ныне вышла из употребления. По крайней мере, в греческом :) Заглавная примерно такая же, только её utf-код 03f7 не поддерживается шрифтом Linux Libertine.


\item да; события независимы в совокупности, если для любого поднабора событий $A_1$, \ldots, $A_k$ выполняется равенство $\P(A_1 \cap A_2 \cap \ldots \cap A_k) = \P(A_1) \cdot \ldots \cdot \P(A_k)$; нет
\item $1/4$, $2/3$, $15$
\item $0$, $0.8^5\cdot 0.2$, $1-0.8^6$
\item $0.8^{10}$, $C_{10}^3 0.2^3 0.8^7$, $2$
\item $1/2$, $3/16$, $3/8$
\begin{enumerate}
\item
\begin{flalign*}
F_X(x) &= \begin{cases}
0, \, x<0 \\
x^3, \, x \in [0;1] \\
1, \, x>1
\end{cases}&&
\end{flalign*}
\item $1/8$
\item $2^{-1/3}$
\item $56/63$
\item
\begin{flalign*}
F_Y(y) &= \begin{cases}
0, \, y<0 \\
1-1/y^3, \, y>0
\end{cases}&&
\end{flalign*}
\item
\begin{flalign*}
f_Y(y) &= \begin{cases}
0, \, y<0 \\
3y^{-4}, \, y>0
\end{cases}&&
\end{flalign*}

\end{enumerate}

\end{enumerate}

\subsection{Регата, исследователи, командный тур}

\begin{enumerate}
\item Восьминогий Кракен. У Кракена 8 ног-шупалец. Если отрубить одно щупальце, то в замен него с вероятностью $1/4$ вырастает новое; с вероятностью $1/4$ вырастает два новых; с вероятностью $1/2$, слава Океану, не вырастает ничего.

Против Кракена бьётся сам Капитан! Он наносит точные удары и безупречно умело уворачивается от ударов Кракена.

\begin{enumerate}
\item Какова вероятность того, что Капитан победит, отрубив ровно 10 щупалец?
\item Какова вероятность того, что бой Кракена и Капитана продлится вечно?
\item Сколько щупалец в среднем отрубит Капитан прежде чем победит?
\end{enumerate}

\item Разбавленный ром. Пират Злопамятный Джо очень любит неразбавленный ром. Из-за
того, что он много пьёт, у него проблемы с памятью, и он помнит не
больше, чем три последних пинты. Хозяин таверны с вероятностью 1/4 разбавляет
каждую подаваемую пинту рома. Если по ощущением Джо половина выпитых
пинт или больше была разбавлена, то он разносит таверну к чертям
собачьим.


\begin{enumerate}
\item Какова вероятность того, что хозяин таверны не успеет подать Джо третью пинту рома?
\item Сколько в среднем пинт выпьет Джо, прежде чем разнесёт таверну?
\end{enumerate}

\item $XY$ в степени $Z$. Чтобы поступить на службу Её Величества, пиратам предлагается следующая задача. Случайные величины $X$, $Y$ и $Z$ равномерны на отрезке $[0;1]$ и независимы.

\begin{enumerate}
\item Найдите функцию распределения случайной величины $-\ln X$
\item Найдите функцию распределения случайной величины $-(\ln X + \ln Y)$
\item Найдите функцию распределения случайной величины $-Z(\ln X + \ln Y)$
\item Какое распределение имеет случайная величина $(XY)^Z$?
\end{enumerate}

\item Тортики. Пираты очень любят тортики и праздновать день рождения! Если хотя бы у одного пирата на корабле день рождения, то все, включая капитана, празднуют и кушают тортики. Корабль в праздничный день дрейфует под действием ветра и не факт, что в нужном направлении.

\begin{enumerate}
\item Сколько пиратов нужно нанять капитану, чтобы ожидаемое количество праздничных дней было равно 100?
\item Сколько пиратов нужно нанять капитану, чтобы максимизировать ожидаемое количество рабочих пирато-дней (произведение числа пиратов на число рабочих дней)?
\end{enumerate}


\item Девятый вал. На побережье пиратского острова одна за одной набегают волны. Высота каждой волны — равномерная на $[0;1]$ случайная величина. Высоты волн независимы. Пираты называют волну «большой», если она больше предыдущей и больше следующей. Пираты называют волну «рекордной», если она больше всех предыдущих волн от начала наблюдения. Обозначим события $B_i= \{ i\text{-ая волна была большой} \}$ и $R_i=\{ i\text{-ая волна была рекордной} \}$.

\begin{enumerate}
\item Найдите $\P(R_{100})$, $\P(B_{100})$
\item Капитан насчитал 100 волн. Сколько в среднем из них были «рекордными»?
\item Найдите $\P(R_{99} | R_{100})$, $\P(R_{100}|B_{100})$
\end{enumerate}


\item Три сундука. Три пирата, Генри Рубинов, Френсис Пиастров и Эдвард Золотов играют одной командой в игру. В комнате в ряд, слева направо, стоят в случайном порядке три закрытых внешне неотличимых сундука: с рубинами, пиастрами и золотом. Общаться после начала игры они не могут, но могут заранее договориться о стратегии. Они заходят в комнату по очереди. Каждый из них может открыть два сундука по своему выбору. После каждого пирата комната возвращается уборщицей идеально точно в исходное состояние. Если Рубинов откроет коробку с рубинами, Писатров — с пиастрами, а Золотов — с золотом, то их команда выигрывает. Если хотя бы один из пиратов не найдёт свою цель, то их команда проигрывает.

\begin{enumerate}
\item Какова вероятность выигрыша, если все пираты пробуют открыть первый и второй сундуки?
\item Какова оптимальная стратегия?
\item Какова вероятность выигрыша при использовании оптимальной стратегии?
\end{enumerate}






\end{enumerate}

\subsection{Регата, исследователи, командный тур, решение}

\begin{enumerate}

\item Если отрублено 10 щупалец, значит либо был один удар породивший два новых щупальца, либо было два удара, породивших по одному новому, а все остальные удары не порождали новых щупалец.

Искомая вероятность равна: $8\cdot 0.5^9 \cdot 0.25^1 + C_8^2 0.5^8 0.25^2$.

Вероятность вечного боя равна нулю. Достаточно доказать, что с вероятностью один за конечное время побеждается одноногий Кракен. А эта вероятность удовлетворяет уравнению: $p=\frac{1}{4}p + \frac{1}{4}p^2 + \frac{1}{2} 1$. Единственный осмысленный корень у этого уравнения — $1$.

Замечаем, что на победу над $k$-шупальцевым Кракеном уходим в $k$ раз больше ударов в среднем чем на победу на $1$-щупальцевым. Отсюда:

\[
e_1=1 + 0.5\cdot 0 + 0.25\cdot e_1 + 0.25 \cdot 2e_1
\]

Решаем, получаем $e_1=4$ и $e_8=32$

\item Либо первая пинта разбавлена, либо первая неразбавлена, а вторая разбавлена, то есть
\[
0.25 + 0.75\cdot 0.25 =0.4375
\]

Рисуем граф:


Составляем систему (индекс — количество выпитых неразбавленных пинт):

\[
\begin{cases}
e_0=\frac{1}{4} + \frac{3}{16}2 + \frac{9}{16}(2+e_2) \\
e_2=1+\frac{3}{4}e_2 + \frac{1}{4}e_0
\end{cases}
\]

Находим $e_0=64/7\approx 9$

\item Начало из домашки! Для $t>0$:
\[
\P(-\ln X \leq t)=\P(\ln X > -t)=\P(X > e^{-t})=1-e^{-t}
\]

Итого,
\[
F_{-\ln X}(t)=\begin{cases}
0, \, t < 0 \\
1-e^{-t}, \, t \geq 0
\end{cases}
\]

Из геометрических соображений легко найти $\P(XY < a)$ для $a\in (0;1)$:
\[
\P(XY < a)=a + \int_a^1 \frac{a}{x} \, dx=a-a\ln a
\]

Переходим ко второму пункту, для $t>0$:
\[
\P(-(\ln X + \ln Y) < t)=\P(XY > e^{-t})= 1-e^{-t} -t e^{-t}
\]

Итого:
\[
F_{-\ln X - \ln Y}(t)=\begin{cases}
0, \, t < 0 \\
1-e^{-t} - te^{-t}, \, t \geq 0
\end{cases}
\]

После дифференциирования получаем функцию плотности для $S=-\ln X - \ln Y$:

\[
f_S(s)=\begin{cases}
0, \, s < 0 \\
se^{-s}, \, s \geq 0
\end{cases}
\]

Приближаемся к финальной вероятности:

\[
\P(ZS > t)= \int_t^{\infty} \int_{t/s}^1  se^{-s} \, dz\, ds=
\int_t^{\infty} (s-t)\cdot e^{-s} \, ds= \ldots = e^{-t}
\]

Сравниваем результат с первым пунктом и приходим к выводу, что величина $(XY)^Z$ имеет равномерное распределение на $[0;1]$.

\item Если нанято $n$ пиратов, то вероятность, того, что в конкретный день все работают равна $(364/365)^n$. Следовательно, ожидаемое количество праздничных дней равно $365(1-(364/365)^n)$.

Решаем уравнение

\[
1-(364/365)^n=100/365
\]

Получаем,
\[
n=\frac{\ln 265- \ln 365}{ \ln 364 - \ln 365}\approx 117
\]

Ожидаемое количество рабочих пирато-дней равно: $365n(364/365)^n$.

Получаем
\[
n^*=1/(\ln 365 - \ln 364)\approx 364
\]

\item
\begin{enumerate}
\item $\P(R_{100})=1/100$ (максимум из 100 величин должен плюхнуться на сотое место), $\P(B_{100})=1/3$ (максимум из трёх величин должен плюхнуться на второе место)
\item $\E(X)=1+\frac{1}{2} + \frac{1}{3}+\ldots + \frac{1}{100}\approx \ln 100 \approx 4.6$. Т.к. $X=X_1+X_2+\ldots + X_{100}$ и $\E(X_i)=1/i$.
\item $\P(R_{99} | R_{100})=1/99$, $\P(R_{100}|B_{100})=3/101$

Для проверки: $\P(R_{99} \cap R_{100})=98!/100!$ ($100!$ — всего перестановок, $98!$ — первые 98 можно переставлять свободно, а в конце должны идти второй наибольшое и наибольшее). $\P(R_{100} \cap B_{100})=1/101$ (максимум из 101 числа плюхнется на 100ое место).
\end{enumerate}

\item Если все пираты открывают первый и второй сундуки, то вероятность выигрыша равна нулю.

Оптимальная стратегия (одна из). Три пирата заранее договариваются, о названиях сундуков. Они называют эти три сундука (ещё до игры)  «рубиновым», «пиастровым» и «золотым». Генри Рубинов должен начать с открытия рубинового сундука, Френсис Пиастров — с пиастрового, Эдвард Золотов — с золотого. Далее каждый пират должен открыть тот сундук, на который указывает предмет, лежащий в первом открытом им сундуке. Например, если Генри Рубинов, открыв сначала рубиновый сундук обнаруживает там пиастры, он должен открывать пиастровый сундук.

Вероятность победы при такой стратегии легко находится перебором 6 возможных вариантов и равна\ldots Та-дам!!! $2/3$.




\end{enumerate}


\subsection{Контрольная номер 2, поток Арктура, 12.12.2015}

Продолжительность: 1 час 20 минут

\begin{enumerate}
\item Функция плотности случайного вектора $\xi=(\xi_1, \xi_2)^T$ имеет вид
\[
f(x,y)=\begin{cases}
0.5x + 1.5y, \text{ если } 0<x<1, \; 0<y<1 \\
0, \text{ иначе }
\end{cases}
\]
Найдите:
\begin{enumerate}
\item Математическое ожидание $\E(\xi_1 \cdot \xi_2)$
\item Условную плотность распределения $f_{\xi_1|\xi_2} (x|y)$
\item Условное математическое ожидание $\E(\xi_1| \xi_2=y)$
\item Константу $k$, такую, что функция $h(x,y)=kx\cdot f(x,y)$ будет являться совместной функцией плотности некоторой пары случайных величин
\end{enumerate}

\item На курсе учится очень много студентов. Вероятность того, что случайно выбранный студент по результатам рубежного контроля имеет хотя бы один незачет равна $0.2$. Пусть $\xi$ и $\eta$ — число студентов с незачетами и без незачетов в случайной группе из $10$ студентов. Найдите $\Cov(\xi,\eta)$, $\Corr(\xi,\eta)$, $\Cov(\xi-\eta,\xi)$. Являются ли случайные величины $\xi-\eta$ и $\xi$ независимыми?

\item Доходности акций компаний А и В – случайные величины $\xi$ и $\eta$. Известно, что $\E(\xi)=1$, $E(\eta)=1$, $\Var(\xi)=4$, $\Var(\eta)=9$, $\Corr(\xi,\eta)=-0.5$. Петя принимает решение потратить свой рубль на акции компании А, Вася — 50 копеек на акции компании А и 50 копеек на акции компании В, а Маша  принимает решение вложить свой рубль в портфель $R=\alpha\xi+(1-\alpha)\eta$, $(0 \leq \alpha \leq 1)$, обладающий минимальным риском. Найдите $\alpha$, ожидаемые доходности и риски портфелей Пети, Васи и Маши.

\item Будем считать, что рождение мальчика и девочки равновероятны.
\begin{enumerate}
\item Оцените с помощью неравенства Маркова вероятность того, что среди тысячи новорожденных младенцев, мальчиков будет более 75\%.
\item Оцените с помощью неравенства Чебышёва вероятность того, что доля мальчиков среди тысячи новорожденных младенцев будет отличаться от 0.5 более, чем на 0.25
\item С помощью теоремы Муавра-Лапласа вычислите вероятность из предыдущего пункта.
\end{enumerate}

\item Сейчас валютный курс племени «Мумба» составляет 100 оболов за один рубль. Изменение курса за один день — случайная величина $\delta_i$ с законом распределения:

\begin{center}
\begin{tabular}{lrrr}
\toprule
$\delta_i$ & $-1$ & $0$ & $2$ \\ \midrule
$\P(\cdot)$ & $0.25$ & $0.5$ & $0.25$ \\
\bottomrule
\end{tabular}
\end{center}

Найдите вероятность того, что через полгода (171 день) рубль будет стоить более 250 оболов, если ежедневные изменения курса происходят независимо друг от друга.

\item \textbf{Бонусная задача}

Число посетителей, зашедших в магазин в течении дня — пуассоновская случайная величина с параметром $\lambda$. Каждый из посетителей совершает покупку с вероятностью $p$, не зависимо от других посетителей. Найдите математическое ожидание числа человек, совершивших покупку.

\end{enumerate}


\subsection{Контрольная номер 2, поток Арктура, 12.12.2015, решение}

\begin{enumerate}
\item
\begin{enumerate}
\item $ \E({\xi_1 \cdot \xi_2}) = \int_{0}^1 \int_0^1 xy f(x,y) \, dx \, dy = \int_0^1 \int_0^1 \frac{1}{2}\cdot x^2y + \frac{3}{2}\cdot xy^2 \, dx \, dy = \int_{0}^1 \frac{y}{6} + \frac{3y^2}{4} \,dy = \frac{1}{3}$
\item $f_{\xi_1 | \xi_2} (x | y) = \frac{f_{\xi_1, \xi_2}(x, y)}{f_{\xi_2}(y)} = \frac{0.5x + 1.5y}{0.25 + 1.5y}, \text{ при } y \in (0,1)$
\item
\begin{multline*}
\E(\xi_1 | \xi_2 = y) = \int_0^1 x f_{\xi_1 | \xi_2} (x | y) dx = \\
= \int\limits_{0}^{1}  x \frac{0,5x + 1,5y}{0,25 + 1,5y} dx = \frac{1}{0,25 + 1,5y}  \left. \left( \dfrac{0,5x^3}{3} +  \dfrac{1,5yx^2}{2} \right) \right|_0^1  =  \frac{1/6 + 3/4y}{0,25 + 1,5y}
\end{multline*}
\item
Для того, чтобы функция являлась совместной плотностью для пары случайных величин, должно выполнятся следующее:
\[
\int_{\Omega} kx f(x,y) \, dx \, dy = 1
\]
Вычислим, чему равняется левая часть:
\[
1 = \int_{\Omega} kx f(x,y) \, dx \, dy = \int_{0}^1 \int_{0}^1 kx \left(\frac{x + 3y}{2}\right) dx \, dy = \int_{0}^1 \frac{k}{6} + \frac{3ky}{4} \, dy = \frac{k}{6} + \frac{3k}{8} \Rightarrow
\]
\[
k = \frac{24}{13}
\]
\end{enumerate}
\item Заметим, что $\xi + \eta = 10$, тогда $\Cov(\xi, \eta) = \Cov(\xi, 10-\xi) = -\Var(\xi)$.

Представим случайную величину $\xi$ в виде суммы случайных величин $\xi = \xi_1 + \ldots + \xi_{10}$, где
\[
\xi_i = \begin{cases}
1, & \text{если у студента есть хотя бы один незачёт}, p=0.2 \\
0, & \text{иначе}, p=0.8
\end{cases} \quad i = 1, \ldots, 10
\]

Поскольку результаты каждого из стуентов независимы, $\Var(\xi) = 10\Var(\xi_1)$
\[
\Cov(\xi, \eta) = -10(1^2 \cdot 0.2 - (1\cdot 0.2)^2) = -1.6
\]

Так как случайные величины $\xi$ и $\eta$ связаны соотношением $\xi = 10 - \eta$, $\Corr(\xi, \eta)=-1$.

Подставив в $\Cov(\xi - \eta, \xi)$ выражение $\eta = 10 - \xi$, получим:
\[
\Cov(\xi - \eta, \xi) = 2 \Cov(\xi, \xi) = 2 \cdot 0.16 = 0.32
\]
Случайные величины $\xi - \eta$ и $\xi$ не являются независиыми.
\item Найдем ожидаемую доходность и риск портфеля $R = \alpha \xi + (1-\alpha) \eta$ для любого $\alpha$, тогда при $\alpha = 1$ получим результаты Пети, при $\alpha = 0.5$ — результаты Васи.
\[
\E R = \alpha + (1-\alpha) = 1 \: \, \forall \, \: \alpha \in [0,1]
\]

Находим дисперсию:
\[
\Var(R) = \alpha^2 \cdot 4 + (1-\alpha)^2 \cdot 9 - 6\alpha (1-\alpha) = 19\alpha^2 -24\alpha + 9 \to \min_{\alpha} \Rightarrow
\]

Теперь, найдем оптимальное $\alpha$:
\[
\alpha = \frac{24}{38}
\]

Финальные цифры:
\[
\begin{cases}
\Var(R)^{P} = 4 \Rightarrow \sigma_{P} = 2 \\
\Var(R)^{V} = 1.75 \Rightarrow \sigma_{V} \approx 1.32 \\
\Var(R)^{M} = \frac{27}{19} \Rightarrow \sigma_{M} \approx 1.19 \\
\end{cases}
\]
\item
\begin{enumerate}
\item Пусть $S$ количество мальчиков, тогда используя \href{https://en.wikipedia.org/wiki/Markov%27s_inequality}{неравенство Маркова} получаем:
\[
\P(S \ge 750) \le \frac{\E(S)}{750} = \frac{2}{3}
\]
\item Пусть, теперь, $\overline{X}$ доля мальчиков, то есть, $\overline{X} = \sum_{i=1}^n X_i /n$, где
\[
X_i =
\begin{cases}
1, \text{ если }i\text{-ый ребёнок — мальчик }\\
0, \text{ иначе }
\end{cases}
\]
тогда используя \href{https://en.wikipedia.org/wiki/Markov%27s_inequality}{неравенство Чебышева} получаем:
\[
\P(|\overline{X} - 0.5| \ge 0.25) \le \frac{\Var(\overline{X})}{0.25^2} = \frac{1/4000}{0.25^2} = 0.004
\]
\item Вероятность из предыдущего пункта можно записать в таком виде:
\begin{multline*}
\P(|\overline{X} - 0.5| \ge 0.25) = \P(\overline{X} \ge 0.75) + \P(\overline{X} \le 0.25) = 2\P(\overline{X} \ge 0.75)=\\
= 2\P(\cN(0;1)\geq 0.25\sqrt{4000})=2\P(\cN(0;1)\geq 15.8) = 1.3 \cdot 10^{-56} \approx 0
\end{multline*}
\end{enumerate}
\item Пусть случайная величина $S$ —  это валютный курс через полгода. Заметим, что $S = 100 + \delta_1 + \ldots + \delta_{171}$.
Тогда по ЦПТ $S \sim \cN(142.75, 203.0625)$. Теперь можно искать нужную вероятность:
\[
\P(S > 250) = \P \left(\frac{S -  142.75}{\sqrt{203.0625}} > \frac{250-142.75}{\sqrt{203.0625}} \right) = \P(\cN(0, 1) > 7.6) \approx 0
\]
%\item $\lambda p$
\end{enumerate}




\subsection{Контрольная номер 2, поток Риччи, 12.12.2015}

Продолжительность: 1 час 20 минут


\begin{enumerate}
\item Функция плотности случайного вектора $\xi=(\xi_1, \xi_2)^T$ имеет вид
\[
f(x,y)=\begin{cases}
0.5x + 1.5y, \text{ если } 0<x<1, \; 0<y<1 \\
0, \text{ иначе }
\end{cases}
\]
Найдите:
\begin{enumerate}
\item Математическое ожидание $\E(\xi_1 \cdot \xi_2)$
\item Условную плотность распределения $f_{\xi_1|\xi_2} (x|y)$
\item Условное математическое ожидание $\E(\xi_1| \xi_2=y)$
\item Константу $k$, такую, что функция $h(x,y)=kx\cdot f(x,y)$ будет являться совместной функцией плотности некоторой пары случайных величин
\end{enumerate}

\item На курсе учится очень много студентов. Вероятность того, что случайно выбранный студент получит «отлично» за контрольную равна $0.2$, «хорошо» — $0.3$. Вероятности остальных результатов неизвестны. Пусть $\xi$ и $\eta$ — число отличников и хорошистов в случайной группе из $10$ студентов. Найдите $\Cov(\xi,\eta)$, $\Corr(\xi,\eta)$, $\Cov(\xi-\eta,\xi)$. Являются ли случайные величины $\xi-\eta$ и $\xi$ независимыми?

\item Доходности акций компаний А и В — случайные величины $\xi$ и $\eta$. Известно, что $\E(\xi)=1$, $E(\eta)=1$, $\Var(\xi)=4$, $\Var(\eta)=9$, $\Corr(\xi,\eta)=-0.5$. Петя принимает решение потратить свой рубль на акции компании А, Вася — 50 копеек на акции компании А и 50 копеек на акции компании В, а Маша  принимает решение вложить свой рубль в портфель $R=\alpha\xi+(1-\alpha)\eta$, $(0 \leq \alpha \leq 1)$, обладающий минимальным риском. Найдите $\alpha$, ожидаемые доходности и риски портфелей Пети, Васи и Маши.

\item Будем считать, что рождение мальчика и девочки равновероятны.
\begin{enumerate}
\item С помощью закона больших чисел определите в каком городе, большом или маленьком, больше случается таких дней, когда рождается более 75\% мальчиков.
\item Оцените с помощью неравенства Маркова вероятность того, что среди тысячи новорожденных младенцев мальчиков будет более 75\%.
\item Оцените с помощью неравенства Чебышёва вероятность того, что доля мальчиков среди тысячи новорожденных младенцев будет отличаться от 0.5 более, чем на 0.25
\item С помощью теоремы Муавра-Лапласа вычислите вероятность из предыдущего пункта.
\end{enumerate}

\item Сейчас валютный курс племени «Мумба» составляет 100 оболов за один рубль. Процентное изменение курса за один день — случайная величина $\delta_i$ с законом распределения:

\begin{center}
\begin{tabular}{lrr}
\toprule
$\delta_i$ & $-1\%$  & $1\%$ \\
$\P(\cdot)$ & $0.25$  & $0.75$ \\
\bottomrule
\end{tabular}
\end{center}

Найдите вероятность того, что через полгода (171 день) рубль будет стоить более 271 обола, если ежедневные изменения курса происходят независимо друг от друга.

\item Величины $X_1$, $X_2$, \ldots независимы и равновероятно принимают значения $-1$ и $3$.
\begin{enumerate}
\item Найдите $\plim_{n\to\infty} \frac{\sum_{i=1}^n(X_i-\bar X)^2}{n}$
\item С помощью дельта-метода найдите примерный закон распределения $\frac{\sum_{i=1}^{100}(X_i-\bar X)^2}{100}$
\end{enumerate}

\end{enumerate}

\subsection{Контрольная номер 2, поток Риччи, 12.12.2015, решение}

Решение: Аршак Минасян


\begin{enumerate}

\item
\begin{enumerate}
\item $ \E({\xi_1 \cdot \xi_2}) = \int_{0}^1 \int_0^1 xy f(x,y) \, dx \, dy = \int_0^1 \int_0^1 \frac{1}{2}\cdot x^2y + \frac{3}{2}\cdot xy^2 \, dx \, dy = \int_{0}^1 \frac{y}{6} + \frac{3y^2}{4} \,dy = \frac{1}{3}$
\item $f_{\xi_1 | \xi_2} (x | y) = \frac{f_{\xi_1, \xi_2}(x, y)}{f_{\xi_2}(y)} = \frac{0.5x + 1.5y}{0.25 + 1.5y}, \text{ при } y \in (0,1)$
\item
\begin{multline*}
\E(\xi_1 | \xi_2 = y) = \int_0^1 x f_{\xi_1 | \xi_2} (x | y) dx = \\
= \int\limits_{0}^{1}  x \frac{0,5x + 1,5y}{0,25 + 1,5y} dx = \frac{1}{0,25 + 1,5y}  \left. \left( \dfrac{0,5x^3}{3} +  \dfrac{1,5yx^2}{2} \right) \right|_0^1  =  \frac{1/6 + 3/4y}{0,25 + 1,5y}
\end{multline*}
\item
Для того, чтобы функция являлась совместной плотностью для пары случайных величин, должно выполнятся следующее:
\[
\int_{\Omega} kx f(x,y) \, dx \, dy = 1
\]
Вычислим, чему равняется левая часть:
\[
1 = \int_{\Omega} kx f(x,y) \, dx \, dy = \int_{0}^1 \int_{0}^1 kx \left(\frac{x + 3y}{2}\right) dx \, dy = \int_{0}^1 \frac{k}{6} + \frac{3ky}{4} \, dy = \frac{k}{6} + \frac{3k}{8} \Rightarrow
\]
\[
k = \frac{24}{13}
\]
\end{enumerate}
\item При расчёте ковариации применим разложение случайной величины в сумму простых случайных величин!

$\Cov(\xi, \eta) = \Cov(\xi_1 + \ldots + \xi_{10}, \eta_1 + \ldots + \eta_{10})=10\Cov(\xi_1, \eta_1)=10(0-0.2\cdot 0.3)=-0.6$

$\Var(\xi)=10\cdot 0.2\cdot 0.8=1.6 $

$\Var(\eta)=10\cdot 0.3\cdot 0.7=2.1 $

$\Corr(\xi,\eta)=-0.6/\sqrt{1.6\cdot 2.1} \approx -0.33 $

$\Cov(\xi - \eta, \xi) = \Var(\xi) - \Cov(\xi, \eta) \neq 0$

Следовательно, $\xi$ и $\eta$ зависимы.

\item Найдем ожидаемую доходность и риск портфеля $R = \alpha \xi + (1-\alpha) \eta$ для любого $\alpha$, тогда при $\alpha = 1$ получим результаты Пети, при $\alpha = 0.5$ — результаты Васи.
\[
\E R = \alpha + (1-\alpha) = 1 \: \, \forall \, \: \alpha \in [0,1]
\]

Находим дисперсию:
\[
\Var(R) = \alpha^2 \cdot 4 + (1-\alpha)^2 \cdot 9 - 6\alpha (1-\alpha) = 19\alpha^2 -24\alpha + 9 \to \min_{\alpha} \Rightarrow
\]

Теперь, найдем оптимальное $\alpha$:
\[
\alpha = \frac{24}{38}
\]

Финальные цифры:
\[
\begin{cases}
\Var(R)^{P} = 4 \Rightarrow \sigma_{P} = 2 \\
\Var(R)^{V} = 1.75 \Rightarrow \sigma_{V} \approx 1.32 \\
\Var(R)^{M} = \frac{27}{19} \Rightarrow \sigma_{M} \approx 1.19 \\
\end{cases}
\]

\item
\begin{enumerate}
\item По ЗБЧ имеем:
\[
\frac{\xi_1 + \dots + \xi_n}{n} \to \E(\xi_1) = \frac{1}{2},
\]
Поэтому, чем больше $n$ (количество жителей в городе), тем меньше таких дней, когда количество мальчиков больше $75\%.$ \\
\item Пусть $S$ количество мальчиков, тогда используя \href{https://en.wikipedia.org/wiki/Markov%27s_inequality}{неравенство Маркова} получаем:
\[
\P(S \ge 750) \le \frac{\E(S)}{750} = \frac{2}{3}
\]
\item Пусть, теперь, $\bar X$ доля мальчиков, то есть, $\overline{X} = \sum_{i=1}^n X_i /n$, где
\[
X_i =
\begin{cases}
1, \text{ если }i\text{-ый ребёнок — мальчик }\\
0, \text{ иначе }
\end{cases}
\]
тогда используя \href{https://en.wikipedia.org/wiki/Markov%27s_inequality}{неравенство Чебышева} получаем:
\[
\P(|\overline{X} - 0.5| \ge 0.25) \le \frac{\Var(\overline{X})}{0.25^2} = \frac{1/4000}{0.25^2} = 0.004
\]
\item Вероятность из предыдущего пункта можно записать в таком виде:
\begin{multline*}
\P(|\overline{X} - 0.5| \ge 0.25) = \P(\overline{X} \ge 0.75) + \P(\overline{X} \le 0.25) = 2\P(\overline{X} \ge 0.75)=\\
= 2\P(\cN(0;1)\geq 0.25\sqrt{4000})=2\P(\cN(0;1)\geq 15.8) = 1.3 \cdot 10^{-56} \approx 0
\end{multline*}
\end{enumerate}

\item

Ищем вероятность

\[
100\cdot X_1 \cdot \ldots \cdot X_{171} \geq 271
\]

Здесь $X_i$ принимают значения $0.99$ или $1.01$ с вероятностями $0.25$ и $0.75$

Берем логарифм:

\[
\sum \log X_i \geq 1
\]

Исходим из худшего случая, когда на калькуляторе нет логарифма, тогда неплохо знать, что $\log (1+\alpha) \sim \alpha$, поэтому можно считать, что $\log X_i$ принимает значения $-0.01$ и $0.01$.

Значит $\E(\log X_i)=1/200$, $\Var(\log X_i)=1/100 - 1/200^2 \approx 1/100$.

Поэтому сумма $S \sim \cN(171/200, 171/100)$ и

\[
\P(S \geq 1)=\P(\cN(0, 1) \geq 0.11) \approx 0.46
\]

\item
\begin{enumerate}
\item Используем ЗБЧ:
\[
\plim_{n \to \infty} \frac{\sum_{i=1}^n (X_i - \bar{X})^2}{n} = \plim \frac{\sum X_i^2}{n} - \plim \bar X \cdot \bar X = \E(X_1^2) - (\E(X_1))^2 = \Var(X_1) = 4
\]
\item Обозначим, $Y_i=X_i^2$, тогда наше выражение можно записать в виде:
\[
Q=\overline{Y} - (\overline{X})^2
\]

Причём, $\plim \bar Y = 5$, $\plim \bar X = 1$.

Согласно дельта-методу заменяем его на линейную аппроксимацию в окрестности предела:
\[
Q\approx 4 + (\bar Y - 5) + 2 (\bar X - 1)
\]

Стало быть, при больших $n$:

\[
\E(Q) \approx 4
\]

\begin{multline*}
\Var(Q) \approx \Var(\bar Y) + 4\Var(\bar X) + 4\Cov(\bar Y, \bar X) =\\
= \frac{1}{n} \left(\Var(Y_1) + 4 \Var(X_1) + 4\Cov(X_1, Y_1) \right) = \\
= 0.01 \cdot (16 + 4 \cdot 4 + 4 \cdot 8)= 0.64
\end{multline*}

Итого, $Q \approx \cN(4; 0.64)$

\end{enumerate}

\end{enumerate}



\subsection{Midterm, 21.12.2015}

\input{tests/2015_midterm}

\vspace{3ex}

Можно пользоваться простым калькулятором.  В каждом вопросе единственный верный ответ. Ни пуха, ни пера!

\vspace{3ex}

\AMCnumero{1} % сбрасываем номер вопроса на 1

\cleargroup{all}
\copygroup[5]{probability1}{all}
\copygroup{newpage}{all}
\copygroup[7]{probability}{all}
\copygroup[1]{commontext}{all}
\copygroup[4]{1316}{all}
\copygroup[3]{1719}{all}
\copygroup[1]{commontext3}{all}
\copygroup[4]{2023}{all}
\copygroup[1]{commontext4}{all}
\copygroup[2]{2425}{all}
\copygroup{ruler}{all}
\copygroup[5]{2630}{all}

\insertgroup{all}


\subsection{Контрольная работа 3. 1 апреля 2016}

\epigraph{Ищите и обрящете, толцыте и отверзется вам}{Лука 11:9}

\begin{enumerate}
\item В студенческом буфете осталось только три булочки одинаковой привлекательности и цены, но разной калорийности: 250, 400 и 550 ккал. Голодные Маша и Саша, не глядя на калорийность, покупают по булочке. Найдите математическое ожидание и дисперсию суммы поглощенных студентами калорий.
\item Дана реализация случайной выборки  независимых одинаково распределенных случайных величин: 11, 4, 6.
\begin{enumerate}
  \item Выпишите вариационный ряд;
  \item Постройте выборочную функцию распределения;
  \item Найдите выборочную медиану распределения;
  \item Вычислите выборочное среднее и несмещенную оценку дисперсии.
\end{enumerate}

\item Найдите математическое ожидание, дисперсию и коэффициент корреляции случайных величин $X$ и $Y$, совместное распределение которых имеет функцию плотности
\[
f(x, y) = \frac{5}{4\pi \sqrt{6}} \exp\left(
-\frac{25}{48}\left( (x-1)^2 -0.4(x-1)y + y^2 \right)
\right)
\]

\item Рост и размер обуви $(X, Y)$ взрослого мужчины хорошо описывается двумерным нормальным распределением с математическим ожиданием $(178, 42)$ и ковариационной матрицей $C = \begin{pmatrix}
49 & 5.6 \\
5.6 & 1 \\
\end{pmatrix}$.
\begin{enumerate}
  \item Какой процент мужчин обладает ростом выше 185 см?
  \item Являются ли рост и размер обуви случайно выбранного мужчины независимыми? Обоснуйте ответ.
  \item Среди мужчин с ростом 185 см, каков процент тех, кто имеет размер обуви, меньший сорок второго  $\P(Y < 42 \mid X=185)$?
\end{enumerate}


\item Дана случайная выборка $X_1$, \ldots, $X_n$ из равномерного распределения $U[0; 2\theta]$.
\begin{enumerate}
  \item С помощью первого момента найдите оценку параметра  $\theta$ методом моментов;
  \item Сформулируйте определения несмещенности, состоятельности и эффективности оценок;
  \item Проверьте, будет ли найденная в пункте (а) оценка несмещенной и состоятельной.
  \item С помощью статистики $X_{(n)}= \max\{ X_1,\ldots, X_n \}$ постройте несмещенную оценку параметра $\theta$  вида $cX_{(n)}$. Укажите значение $c$.
  \item Проверьте, будет ли данная оценка состоятельной;
  \item Какая из двух оценок является более эффективной? Обоснуйте ответ.
\end{enumerate}

\item Вовочка хочет проверить утверждение организаторов юбилейной лотереи «Метро-80 лет в ритме столицы», что почти треть всех билетов выигрышные. Для этого он попросил $n$ своих друзей купить по 10 лотерейных билетов.  Пусть  $X_i$ — число выигрышных билетов друга $i$ и $p$ — вероятность выигрыша одного билета.
\begin{enumerate}
  \item  Какое распределение имеет величина $X_i$?
  \item Запишите функцию правдоподобия $L(p)$  для выборки $X_1$, \ldots, $X_n$;
  \item Методом максимального правдоподобия найдите оценку $p$;
  \item Найдите информацию Фишера для одного наблюдения $i(p)$;
  \item Для произвольной несмещенной оценки $T(X_1, \ldots, X_n)$ запишите неравенство Рао-Крамера-Фреше;
  \item Будет оценка $\hat p_{ML}$ эффективной?
  \item Найдите оценку максимального правдоподобия математического ожидания и дисперсии выигранных произвольным другом билетов;
  \item Дана реализация случайной выборки 5 Вовочкиных друзей. Число выигрышных билетов  оказалось равно (3, 4, 0, 2, 6). Найдите значение точечной оценки вероятности выигрыша $p$. Как Вы думаете, похоже ли утверждение организаторов на правду?
\end{enumerate}

\item  Дана выборка $X_1$, $X_2$, \ldots, $X_n$ независимых одинаково распределенных величин из распределения с функцией плотности
\[
f(x)=\begin{cases}
(1+\theta)x^\theta, \text{ если } 0<y<1, \theta+1>0 \\
0, \text{ иначе}
\end{cases}.
\]

Методом максимального правдоподобия найдите оценку параметра $\theta$.

\item Пробег (в 1000 км) автомобиля «Лада Калина» до капитального ремонта двигателя является нормальной случайной величиной с неизвестным математическим ожиданием $\mu$ и известной дисперсией 49. По выборке из 20 автомобилей найдите значение доверительного интервала для математического ожидания пробега с уровнем доверия 0.95.

\end{enumerate}


\subsection{Контрольная работа 3, решения}

\begin{enumerate}
\item Пусть случайная величина $S$ – это сумма поглощённых калорий

\begin{tabular}{cccc}
\toprule
$S$ & $650$ & $800$ & $950$ \\ \midrule
$\P(\cdot)$ & $1/3$ & $1/3$ & $1/3$ \\ \bottomrule
\end{tabular}

Тогда
\[
\E(S) = \frac{1}{3}\cdot 650 +  \frac{1}{3}\cdot 800 +  \frac{1}{3}\cdot 950 = 800
\]
\[
\Var(S) = \frac{1}{3}(650-800)^2 + \frac{1}{3}(800-800)^2 + \frac{1}{3}(950-800)^2 = 15000
\]
\item Вариационный ряд: $4, 6, 11$; медиана: $6$; выборочное среднее: $7$; несмещённая оценка дисперсии: $13$
\item Фунуция плотности двумерного нормального распределения имеет вид:
\begin{multline*}
f(x,y) =  \frac{1}{2\pi}\cdot \frac{1}{\sigma_x \sigma_y \sqrt{1-\rho^2}} \cdot \\
\cdot \exp\left\{{-\frac{1}{2}\frac{1}{\sigma_x^2 \sigma_y^2(1-\rho^2)}\left[\sigma_x^2(x-\mu_x)^2-2\rho\sigma_x\sigma_y(x-\mu_x)(y-\mu_y)+\sigma_y^2(y-\mu_y)^2\right]}\right\}
\end{multline*}
Откуда: $\mu_X=1$, $\mu_Y=0$, $\sigma_X = 1$, $\sigma_Y = 1$, $\rho = 0.2$

\item
\begin{enumerate}
\item \begin{multline*}
X \sim \cN(178, 49)$, $\P(X>185) = 1  - \P(X<185) = \\
= 1- \P\left(\frac{X-178}{7} < \frac{185-178}{7}\right) = 1 - 0.8413 = 0.1587
\end{multline*}
\item Нет, так как $\Cov(X, Y) = 5.6 \neq 0$
\item $Y \mid X \sim \cN\left(\mu_Y + \rho\sigma_Y\cdot\frac{X-\mu_X}{\sigma_X}; \sigma_Y^2(1-\rho^2)\right)$

$Y \mid X=185 \sim \cN(42.8;0.36)$

$\P(Y<42 \mid X=185) = \P\left(\frac{Y-42.8}{0.6} < \frac{42-42.8}{0.6}\mid X=185\right) = 0.9082$
\end{enumerate}

\item
\begin{enumerate}
\item $\E(X) = \frac{0+2\theta}{2}\mid_{\hat{\theta}} = \overline{X}$, $\hat{\theta}_{MM} = \overline{X}$
\item $\forall \theta \in \Theta: \E(\hat{\theta})=\theta \Leftrightarrow \hat{\theta}$ – несмещённая.

$\forall \theta \in \Theta, \forall \epsilon > 0 : \P(\vert \widehat{\theta}_n - \theta \vert > \epsilon) \to 0 \Leftrightarrow  \widehat{\theta}_n$ – состоятельная.

$\forall \theta \in \Theta: I_n (\theta) = \Var(\hat{\theta}) \Leftrightarrow \hat{\theta} $ – эффективная.
\item $\E(\theta) = \E(\overline{X}) = \E(X_1) = \theta \Rightarrow \hat{\theta}$ – несмещённая оценка

$\Var(\hat{\theta_n}) = \Var(\overline{X}) = \frac{\Var(X_1)}{n} = \frac{4\theta^2}{12\cdot n} \to 0$; из условий $\E(\widehat{\theta}_n) = \theta$ и $\Var(\widehat{\theta}_n)  \to_{n \to \infty} 0$ следует, что $\widehat{\theta}_n \stackrel{\P}{\to}  \theta$ при $n \to \infty$, т.е. $\widehat{\theta}_n$ является состоятельной.

\item
\begin{multline*}
F_{X_{(n)}} = \P(max(X_1, \ldots, X_n) \leq x) = \P(X_1 \leq x) \cdot \ldots \cdot \P(X_n \leq x) = (\P(X_1 \leq x))^n = \\
= \begin{cases}
0 & \text{при } x<0 \\
\left(\frac{x}{2\theta}\right)^n & \text{при }  x \in [0, 2\theta] \\
1 & \text{при }  x > 2\theta
\end{cases}
\end{multline*}

\[
f_{X_{(n)}} (x)  = \begin{cases}
0 & \text{при } x<0 \\
\frac{nx^{n-1}}{2^n \theta^n} & \text{при }  x \in [0, 2\theta] \\
0 & \text{при }  x > 2\theta
\end{cases}
\]

\begin{multline*}
\E(X_{(n)}) = \int_{-\infty}^{+\infty} x \cdot f_{X_{(n)}} (x) dx = \int_{0}^{2\theta}	x \cdot \frac{nx^{n-1}}{2^n \theta^n} dx = \frac{n}{2^n \theta^n}  \cdot \frac{x^{n+1}}{n+1} \mid_{x=0}^{x=2\theta} \\
= \frac{n}{2^n \theta^n}  \cdot \frac{2^{n+1}\cdot \theta^{n+1}}{n+1} = \frac{n2\theta}{n+1}
\end{multline*}
Следовательно, $\E \left(\frac{n+1}{2n} \cdot X_{(n)}\right) = \theta$, а значит, $\tilde{\theta} = \frac{n+1}{2n} \cdot X_{(n)}$ – несмещённая оценка вида $c \cdot  X_{(n)}$
\item $\Var(\tilde{\theta}) = \frac{(n+1)^2}{4n^2} \Var(X_{(n)})$

\begin{multline*}
\E(X_{(n)}^2) = \int_{-\infty}^{+\infty} x^2 f_{X_{(n)}} (x) dx = \int_{0}^{2\theta} x^2 \frac{nx^{n-1}}{2^n \theta^n}  dx = \frac{n}{2^n \theta^n}  \int_{0}^{2\theta} x^{n+1} dx = \\
= \frac{n}{2^n \theta^n} \cdot \frac{x^{n+2}}{n+2} \mid_{x=0}^{x=2\theta} = \frac{n}{2^n \theta^n} \cdot  \frac{2^{n+2}\cdot \theta^{n+2}}{n+2} = \frac{n\cdot4\cdot\theta^2}{n+2}
\end{multline*}

\[
\Var(X_{(n)}) = \E(X_{(n)}^2)  - (\E(X_{(n)}))^2 = \frac{4n\theta^2}{n+2} - \frac{4 n^2 \cdot \theta^2}{(n+1)^2} = 4n\theta^2 \left(\frac{1}{n+2} - \frac{n}{(n+1)^2}\right)
\]

\[
\Var(\tilde{\theta}) = \frac{(n+1)^2}{4n^2} \Var(X_{(n)}) = \frac{(n+1)^2}{4n^2}  \cdot 4n\theta^2 \left(\frac{n^2+2n+1 - n^2-2n}{(n+2)(n+1)^2} \right) = \frac{\theta^2}{n(n+2)}
\]
Оценка $\tilde{\theta}_n$ является состоятельной, так как $\E(\tilde{\theta}_n) = \theta$ и $\Var(\tilde{\theta}_n) = \frac{\theta^2}{n(n+2)} \to_{n\to\infty} 0$
\item  Поскольку $\Var(\widehat{\theta}_n) = \frac{\theta^2}{3n}$, $\Var(\tilde{\theta}_n) = \frac{\theta^2}{n(n+2)}$ при достаточно большом $n$ $\Var(\tilde{\theta}_n) < \Var(\widehat{\theta}_n) $. Значит, при таких $n$ оценка $\tilde{\theta}_n$ будет более эффективной по сравнению с оценкой $\widehat{\theta}_n$.
\end{enumerate}

\item
\begin{enumerate}
\item $X_i \sim Bin (n=10, p)$
\item $L(p)  = \prod_{i=1}^{n} C_{10}^{x_i} p^{x_i} (1-p)^{10-x_i}$
\item $\ln L(p) = \sum_{i=1}^{n} \ln C_{10}^{x_i} + \sum_{i=1}^n x_i \ln p + \sum_{i=1}^{n} (10-x_i)\ln (1-p) \to \max_p$

$\frac{\partial \ln L}{\partial p} = \frac{\sum_{i=1}^n x_i}{p} - \frac{\sum_{i=1}^{n} (10-x_i)}{1-p} \mid_{p=\hat{p}} = 0 \Rightarrow \hat{p} = \frac{\overline{X}}{n} = \frac{\sum_{i=1}^{n} x_i}{10n}$

$\frac{\partial^2 \ln L}{\partial p^2} = -\frac{\sum_{i=1}^n x_i}{p^2} -  \frac{\sum_{i=1}^{n} (10-x_i)}{(1-p)^2}$

\item $I(p) = -\E \left(\frac{\partial^2 \ln L}{\partial p^2}  \right) = \E \left(\frac{\sum_{i=1}^n x_i}{p^2} + \frac{\sum_{i=1}^{n} (10-x_i)}{(1-p)^2}\right) = \frac{10np}{p^2} + \frac{10n - 10np}{(1-p)^2} = \frac{10n}{p(1-p)}$

$i(p) = \frac{I(p)}{n} = \frac{10}{p(1-p)}$

\item $\Var(T) \geq \frac{1}{ni(T)}$

\item $\Var(\hat{p}_{ML}) = \Var\left(\frac{\sum_{i=1}^{n} x_i}{10n}\right) = \frac{1}{(10n)^2} n \Var(X_i) = \frac{1}{100n}10p(1-p) = \frac{p(1-p)}{10n}$

$\frac{p(1-p)}{10n} = \frac{1}{\frac{10n}{p(1-p)}} \Rightarrow$ да

\item $\E(X_i) = 10p \Rightarrow \widehat{\E(X_i)} = 10 \hat{p}_{ML} = \overline{X}$

$\Var(X_i) = 10p(1-p) \Rightarrow \widehat{\Var(X_i)} = \overline{X}\left(1-\frac{\overline{X}}{10}\right)$

\item $\hat{p} = \frac{3+4+0+2+6}{10\cdot 5} = 0.3$
\end{enumerate}

\item $L(x, \theta) = \prod_{i=1}^{n} (1 + \theta) x_i^\theta = (1+\theta)^n \prod_{i=1}^n x_i^\theta \to \max_\theta$

$\ln L (x, \theta) = n\ln (1+\theta) + \theta\sum_{i=1}^{n} \ln x_i \to \max_\theta$

$\frac{\partial \ln L}{\partial \theta} = \frac{n}{1+\theta} + \sum_{i=1}^{n} \ln x_i \mid_{\theta=\hat{\theta}} = 0 \Rightarrow \hat{\theta}_{ML} = -\frac{1}{\sum_{i=1}^{n} \ln x_i} -1$

\item $\overline{X} - 1.96 \frac{7}{\sqrt{20}} < \mu <\overline{X} + 1.96 \frac{7}{\sqrt{20}} $
\end{enumerate}


\subsection{Контрольная работа 3. Брутальная часть. 1 апреля 2016}

Правила: 3 часа, всем можно пользоваться, интернетом тоже.

Все семь задач решать вовсе не обязательно, выбирайте любые пять! При самостоятельной работе можно всем пользоваться!!!! :)

\begin{enumerate}

\item Случайные величины $X_1$, \ldots, $X_n$ независимо и одинаково распределены с функцией плотности $f(x)=2ax\exp(-ax^2)$ при $x>0$.



По 100 наблюдениям известно, что $\sum X_i = 169.55$, $\sum X_i^2 = 351.48$.

\begin{enumerate}
\item Оцените параметр $a$ методом максимального правдоподобия.
\item Оцените дисперсию оценки $\hat a_{ML}$
\item Постройте 95\%-ый доверительный интервал для $a$ с помощью оценки максимального правдоподобия
\item Оцените параметр $a$ методом моментов
\item Оцените дисперсию оценки $\hat a_{MM}$
\item Постройте 95\%-ый доверительный интервал для $a$ с помощью оценки метода моментов
\end{enumerate}

\item Для того, чтобы люди давали правдивый ответ на деликатный вопрос (скажем, «Берёте ли Вы взятки?») при опросе используется рандомизация. Вопрос допускает всего два ответа «да» или «нет». Перед ответом респондент подбрасывает монетку, и только респондент видит результат подбрасывания. Если монетка выпадет «орлом», то респондент отвечает правду. Если «решкой», то респондент отвечает наоборот («да» вместо «нет» и «нет» вместо «да»).

Монетка выпадает орлом с вероятностью $0.4$. Из 500 опрошенных 300 ответили «да».

\begin{enumerate}
\item Какова вероятность того, что человек берёт взятки, если он ответил «да» в анкете?
\item Постройте оценку для доли людей берущих взятки
\item Постройте 95\%-ый доверительный интервал для доли людей берущих взятки
\end{enumerate}

\item Винни-Пух хочет измерить высоту Большого дуба, $d$. Для этого Винни-Пух три раза в случайное время дня измерил длину тени Большого Дуба:

\begin{knitrout}
\definecolor{shadecolor}{rgb}{0.969, 0.969, 0.969}\color{fgcolor}\begin{kframe}
\begin{verbatim}
## [1]  8.9 13.2 25.2
\end{verbatim}
\end{kframe}
\end{knitrout}

Предположим, что в дни измерений траектория движения Солнца проходила ровно через зенит :)

\begin{enumerate}
\item Найдите функцию плотности длины тени
\item Если возможно, постройте оценку метода моментов
\item Если возможно, постройте оценку метода максимального правдоподобия
\item Где живёт Винни-Пух и какого числа 2016 года он проводил измерения?
\end{enumerate}

\item Встроенный в \verb|R| набор данных \verb|morley| содержит результаты 100 опытов Майкельсона и Морли. В 1887 году они проводили измерения скорости света, чтобы понять, зависит ли она от направления.

\begin{enumerate}
\item Постройте 95\%-ый доверительный интервал для скорости света
\item Выпишите использованные формулы и алгоритм построения интервала
\item Чётко сформулируйте все гипотезы при которых данный алгоритм даёт корректный результат
\item Накрывает ли построенный доверительный интервал фактическую скорость света?
\end{enumerate}

Полезные команды: \verb|morley|, \verb|help("morley")|, \verb|mean|, \verb|sd|, \verb|qnorm|, \verb|pnorm|



\item  Исследователь Вениамин дрожащей от волнения рукой рисует прямоугольники размера $a\times b$. Поскольку Вениамин очень волнуется прямоугольники де-факто выходят со случайными сторонами $a+u_i$ и $b+v_i$. Случайные ошибки $u_i$ и $v_i$ независимы и одинаково распределены $N(0;1)$.




Вениамин нарисовал 400 прямоугольничков и посчитал очень аккуратно площадь каждого. Оказалась, что средняя площадь равна $1198.34$ см$^2$, а выборочное стандартное отклонение площади — $52.83$ см$^2$. Вениамин считает, что зная только площади прямоугольничков невозможно оценить оценить каждую из сторон.

Если возможно, то оцените параметры $a$ и $b$ подходящим методом. Если невозможно, то докажите.


\item На поле $D4$ шахматной доски стоит конь. Ли Седоль переставляет коня наугад, выбирая каждый возможный ход равновероятно.

Сколько в среднем пройдет ходов прежде чем Ли Седоль снова вернёт коня на $D4$?



\item В «Киллер» играли $n$ человек. После окончания игры, когда были убиты все, кто может быть убит, встретились два игрока (возможно убитых) и оказалось, что один убил 5 человек, а другой — 7 человек.

Оцените $n$ подходящим методом
\end{enumerate}


\subsection{Контрольная 4, 09.06.2016}

\begin{enumerate}
\item	Сформулируйте определения несмещённости, состоятельности и эффективности оценок.

\item На курсе учится 250 человек. Предположим, что число студентов, не явившихся на экзамен, хорошо описывается законом Пуассона.
\begin{enumerate}
\item	Методом максимального правдоподобия найдите оценку параметра распределения Пуассона.
\item	Проверьте выполнение свойств несмещенности, эффективности и состоятельности для данной оценки.
\item	Найдите оценку максимального правдоподобия для вероятности стопроцентной явки студентов на экзамен.
\item	Используя дельта-метод, постройте для этой вероятности асимптотический доверительный интервал.
\end{enumerate}

\item	Фармацевтическая компания выпустила новое лекарство от бессонницы, утверждая, что оно помогает 80\% людей, страдающих бессонницей. Чтобы проверить утверждение компании, случайным образом выбираются 20 человек, страдающих бессонницей. Обозначим за $Y$ количество человек из выборки, которым лекарство помогло. Основная гипотеза, $H_0$: $p=0.8$, альтернативная гипотеза $H_a$: $p=0.6$. Критическая область: $\{Y<12\}$.
\begin{enumerate}
\item	В терминах этой задачи сформулируйте, что является ошибкой первого рода.
Найдите уровень значимости, соответствующий заданной критической области.
\item	В терминах этой задачи сформулируйте, что является ошибкой второго рода.
Найдите вероятность ошибки второго рода.
\item	Найдите такое значение $c$, что вероятность ошибки первого рода $\alpha \approx 0.1$ при критической области вида $\{Y<c\}$. Найдите соответствующее значение вероятности ошибки второго рода.
\item	Каким должен быть размер выборки, чтобы выборочная доля страдающих бессонницей отличалась от истинной вероятности не более, чем на 0.01 с вероятностью не менее, чем 0.95?
\end{enumerate}

\item	Вася Сидоров утверждает, что ходит в кино в два раза чаще, чем на лекции по статистике, на лекции по статистике в два раза чаще, чем в спортзал. За последние полгода он 10 раз был в спортзале, 1 раз — на лекциях по статистике и 39 раз в кино.

При помощи критерия хи-квадрат Пирсона на уровне значимости $0.05$ проверьте, правдоподобно ли Васино утверждение.

\item У Евдокла есть случайная выборка из экспоненциального распределения с неизвестным параметром $\lambda$ в 50 наблюдений, $X_1$, $X_2$, \ldots, $X_{50}$. Оказалось, что $\bar X = 1.1$. Евдокл хочет проверить гипотезу о равенстве $\lambda = 1$ против альтернативной гипотезы о неравенстве $\lambda \neq 1$ на уровне значимости $0.1$.

Помогите Евдоклу и проверьте гипотезу с помощью критерия отношения правдоподобия.

Пачка логарифмов: $\ln 50 \approx 3.9$, $\ln 55 \approx 4.0$, $\ln 11 \approx 2.4$, $\ln 60 \approx 4.1$, $\ln 12 \approx 2.5$


\item Американский демографический журнал опубликовал исследование, в котором утверждается, что посетители крупных торговых центров за одно посещение тратят  в выходные дни больше, чем в будние. Наибольшие расходы приходятся на воскресенье в период с 4 до 6 часов вечера. Для двух независимых выборок посетителей средние расходы и выборочные стандартные отклонения расходов составили

\begin{tabular}{lrr}
\toprule
 & Выходные & Рабочие дни \\
\midrule
Число наблюдений & 21 & 19 \\
Средние расходы (\$) & 78 & 67 \\
Выборочное стандартное отклонение (\$) & 22 & 20 \\
\bottomrule
\end{tabular}


\begin{enumerate}
\item Проверьте гипотезу о равенстве дисперсий расходов
\item Предполагая, что дисперсии расходов одинаковы, проверьте гипотезу об отсутствии разницы в расходах в выходные и будние дни.
\item Сформулируйте все необходимые для проверки гипотез предыдущих пунктов предпосылки.
\end{enumerate}



\item Винни Пух знает, что пчёлы и мёд бывают правильные и неправильные. По результатам 100~попыток добыть мёд Винни Пух составил таблицу сопряженности признаков.


\begin{tabular}{lrr}
\toprule
 & Мёд правильный & Мёд неправильный \\
\midrule
Пчёлы правильные & 12	& 36 \\
Пчёлы неправильные & 32	& 20 \\
\bottomrule
\end{tabular}


На уровне значимости $0.05$ проверьте гипотезу о независимости характеристик пчёл и мёда.


\begin{figure}[b]
\centering
\includegraphics[width=9cm]{images/winnie_kr_4}
\end{figure}



\end{enumerate}


\subsection{Контрольная 4, решения}

\begin{enumerate}

	\item[4.] $H_0: p_{\text{c}} = \frac{1}{7}, p_{\text{л}} = \frac{2}{7}, p_{\text{к}} = \frac{4}{7}$

	$Q = \sum_{i=1}^{s=3} \frac{(\nu_i - np_i)^2}{np_i} \sim \chi^2_{s-k-1} = \chi^2_2$

	$Q_{obs} = \frac{\left(10-70\frac{1}{7}\right)^2}{70\frac{1}{7}} + \frac{\left(1-70\frac{2}{7}\right)^2}{70\frac{2}{7}} + \frac{\left(39-70\frac{4}{7}\right)^2}{70\frac{4}{7}} = 18.075$

	$Q_{crit} = 5.99$, $Q_{crit} < Q_{obs} \Rightarrow$ гипотеза отвергается

	\item[5.] $LR \sim \chi^2_1$, так как основная гипотеза содержит одно уравнение

	$L(x, \lambda) = \prod_{i=1}^{n=50} \lambda e^{-\lambda x} = \lambda^{50} e^{-\lambda \sum_{i=1}^{n=50} x_i}$

	$\ln L (x, \lambda) = 50\ln\lambda - \lambda \sum_{i=1}^{n=50} x_i \to \max_\lambda$

	$\frac{\partial \ln L}{\partial \lambda} = \frac{50}{\lambda} - \sum_{i=1}^{n=50} x_i \mid_{\lambda=\hat{\lambda}} = 0 \Rightarrow \hat{\lambda}_{ML} = \frac{1}{\overline{X}} = \frac{10}{11}$

	При верной $H_0:  \lambda=1$, тогда $\ln L (\lambda=1) = 50 \ln 1 - 1 \cdot 1.1 \cdot 50 = -55$

	При верной $H_1: \lambda=\lambda_{ML}$, тогда $\ln L \left(\lambda=\frac{10}{11}\right) = 50 \ln \frac{10}{11} - \frac{10}{11} \cdot 50 \cdot 1.1 = -54.77$

	$LR_{obs} = 2(\ln L (H_1) - \ln(H_0)) = 2(-54.77- (-55)) = 0.46$

	$LR_{crit} = 2.71$, $LR_{crit} > LR_{obs} \Rightarrow$ оснований отвергать $H_0$ нет

	\item[6.]  Будем проверять гипотезы на уровне значимости $0.05$
		\begin{enumerate}
			\item $\hat{\sigma}^2_{\text{в}} = 484$, $\hat{\sigma}^2_{\text{р}} = 400$

				$\frac{\hat{\sigma}^2_{\text{в}} }{\hat{\sigma}^2_{\text{р}}} \sim F_{21-1 , 19-1}$

				$F_{obs} = \frac{484}{400} = 1.21$, $F_{crit, left} = 0.4$, $F_{crit, right} = 2.6  \Rightarrow$ оснований отвергать $H_0$ нет

			\item $\hat{\sigma}_0^2 = \frac{22\cdot (21-1) + 20 \cdot (19-1)}{21 + 19 - 2} \approx 21$

			$t_{obs} = \frac{78-67}{\sqrt{21} \sqrt{\frac{1}{21}+ \frac{1}{21}}} \approx 7.8 $

			$t_{crit}  \sim t_{21+19-2} = t_{38}$, $t_{crit} = \pm 2.02 \Rightarrow$ гипотеза отвергается
		\end{enumerate}

	\item[7.] $\gamma = \sum_{i=1}^s \sum_{j=1}^m \frac{\left(n_{ij} - \frac{n_{i\cdot}n_{\cdot j}}{n}\right)^2}{\frac{n_{i\cdot}n_{\cdot j}}{n}} \sim \chi^2_{(s-1)(m-1)}$

	$\gamma_{obs} = \frac{\left(12-\frac{44\cdot48}{100}\right)^2}{\frac{44\cdot48}{100}} + \frac{\left(36-\frac{56\cdot48}{100}\right)^2}{\frac{56\cdot48}{100}} + \frac{\left(32-\frac{44\cdot52}{100}\right)^2}{\frac{44\cdot52}{100}} + \frac{\left(20-\frac{50\cdot52}{100}\right)^2}{\frac{50\cdot52}{100}} \approx 12$

	$\gamma_{criit} = 3.84 \Rightarrow$  гипотеза отвергается

\end{enumerate}


\subsection{Экзамен, 20.06.2016}

\input{tests/2015_final}

\AMCnumero{1} % сбрасываем номер вопроса на 1
\cleargroup{all}
\copygroup[6]{prob_one_sample}{all}
\copygroup[6]{prob_two_samples}{all}
\copygroup[6]{prob_sample_char}{all}
\copygroup[6]{prob_estimators}{all}
\copygroup[6]{prob_ml_mm}{all}
\insertgroup{all}

\section{2016-2017}

\input{chapters/year_2016_2017.tex}

\section{2017-2018}

\subsection{Теоретический минимум к кр 1}


\begin{enumerate}
	\item Классическое определение вероятности
	\item Определение условной вероятности
	\item Определение независимости случайных событий
	\item Формула полной вероятности
	\item Формула Байеса
	\item Функция распределения случайной величины. Определение и свойства.
	\item Функция плотности. Определение и свойства.
	\item Математическое ожидание. Определения для дискретного и абсолютно непрерывного случаев. Свойства.
	\item Дисперсия. Определение и свойства.
	\item Законы распределений. Определение, $\E(X)$, $\Var(X)$:
	\begin{enumerate}
	\item Биномиальное распределение
	\item Распределение Пуассона
	\item Геометрическое распределение
	\item Равномерное распределение
	\item Экспоненциальное распределение
	\end{enumerate}
\end{enumerate}


\subsection{Задачный минимум к кр 1}

\begin{enumerate}
\item  Пусть $\P(A) = 0.3, \P(B) = 0.4, \P(A\cap B) = 0.1 $. Найдите
	\begin{enumerate}
		\item  $\P(A|B)$
		\item  $\P(A\cup B)$
		\item  Являются ли события $A$ и $B$ независимыми?
	\end{enumerate}



\item  Пусть $\P(A) = 0.5, \P(B) = 0.5, \P(A\cap B) = 0.25 $. Найдите
\begin{enumerate}
	\item  $\P(A|B)$
	\item  $\P(A\cup B)$
	\item  Являются ли события $A$ и $B$ независимыми?
\end{enumerate}



\item  Карлсон выложил кубиками слово КОМБИНАТОРИКА. Малыш выбирает наугад четыре кубика и выкладывает их в случайном порядке.
Найдите вероятность того, что при этом получится слово КОРТ.


\item  Карлсон выложил кубиками слово КОМБИНАТОРИКА. Малыш выбирает наугад четыре кубика и выкладывает их в случайном порядке.
Найдите вероятность того, что при этом получится слово РОТА.

\item  В первой урне 7 белых и 3 черных шара, во второй урне 8 белых и 4 черных
шара, в третьей урне 2 белых и 13 черных шаров. Из этих урн наугад выбирается одна урна. Какова вероятность того, что шар, взятый наугад из выбранной урны, окажется белым?


\item  В первой урне 7 белых и 3 черных шара, во второй урне 8 белых и 4 черных
шара, в третьей урне 2 белых и 13 черных шаров. Из этих урн наугад выбирается одна урна. Какова вероятность того, что была выбрана первая урна, если шар, взятый наугад из выбранной урны, оказался белым?


\item  В операционном отделе банка работает 80\% опытных сотрудников и 20\%
неопытных. Вероятность совершения ошибки при очередной банковской операции
опытным сотрудником равна 0.01, а неопытным — 0.1. Найдите вероятность совершения ошибки при очередной банковской операции в этом отделе.


\item  В операционном отделе банка работает 80\% опытных сотрудников и 20\%
неопытных. Вероятность совершения ошибки при очередной банковской операции
опытным сотрудником равна 0.01, а неопытным — 0.1. Известно, что при очередной банковской операции была допущена ошибка. Найдите вероятность того, что ошибку допустил неопытный сотрудник.

\item  Пусть случайная величина $X$ имеет таблицу распределения:

\begin{tabular}{ ll l l}
	\toprule
	$X$ & -1  & 0  & 1 \\
	$\P_X$ & 0.25  & c  & 0.25 \\
  \bottomrule
\end{tabular}

Найдите
	\begin{enumerate}
	\item константу $c$
	\item $\P(\{X \geq 0\})$
	\item $\P(\{X < -3\}])$
	\item $\P(\{X \in [-\frac{1}{2}; \frac{1}{2}]\})$
	\item функцию распределения случайной величины $X$
	\item имеет ли случайная величина $X$ плотность распределения?
	\end{enumerate}


\item  Пусть случайная величина $X$ имеет таблицу распределения:

\begin{tabular}{ llll}
\toprule
$X$ & -1  & 0  & 1 \\
$\P_X$ & 0.25  & c  & 0.25 \\
\bottomrule
\end{tabular}

Найдите
\begin{enumerate}
	\item константу $c$
	\item $\E(X)$
	\item $\E(X^2)$
	\item $\Var(X)$
	\item $\E(|X|)$
\end{enumerate}

\item  Пусть случайная величина $X$ имеет таблицу распределения:

\begin{tabular}{ lll l}
\toprule
$X$ & -1  & 0  & 1 \\
$\P_X$ & 0.25  & c  & 0.5 \\
\bottomrule
\end{tabular}

Найдите
	\begin{enumerate}
	\item константу $c$
	\item $\P(\{X \geq 0\})$
	\item $\P(\{X < -3\}])$
	\item $\P(\{X \in [-\frac{1}{2}; \frac{1}{2}]\})$
	\item функцию распределения случайной величины $X$
	\item имеет ли случайная величина $X$ плотность распределения?
\end{enumerate}

\item  Пусть случайная величина $X$ имеет таблицу распределения:

\begin{tabular}{ l l l l}
  \toprule
$X$ & -1  & 0  & 1 \\
$\P_X$ & 0.25  & c  & 0.5 \\
\bottomrule
\end{tabular}

Найдите
\begin{enumerate}
	\item константу $c$
	\item $\E(X)$
	\item $\E(X^2)$
	\item $\Var(X)$
	\item $\E(|X|)$
\end{enumerate}

\item Пусть случайная величина $X$ имеет биномиальное распределение с
параметрами $n = 4$ и $\P = \frac{3}{4}$.
 Найдите
\begin{enumerate}
	\item $\P(\{X = 0\})$
	\item $\P(\{X > 0\})$
	\item $\P(\{X < 0\})$
	\item $\E(X)$
	\item $\Var(X)$
	\item  наиболее вероятное значение, которое принимает случайная величина $X$
\end{enumerate}

\item Пусть случайная величина $X$ имеет биномиальное распределение с
параметрами $n = 5$ и $\P = \frac{2}{5}$.
Найдите
\begin{enumerate}
	\item $\P(\{X = 0\})$
	\item $\P(\{X > 0\})$
	\item $\P(\{X < 0\})$
	\item $\E(X)$
	\item $\Var(X)$
	\item  наиболее вероятное значение, которое принимает случайная величина $X$
\end{enumerate}


\item  Пусть случайная величина X имеет распределение Пуассона с параметром $\lambda = 100$ . Найдите
\begin{enumerate}
	\item $\P(\{X = 0\})$
	\item $\P(\{X > 0\})$
	\item $\P(\{X < 0\})$
	\item $\E(X)$
	\item $\Var(X)$
	\item  наиболее вероятное значение, которое принимает случайная величина $X$
\end{enumerate}


\item  Пусть случайная величина X имеет распределение Пуассона с параметром $\lambda = 101$ . Найдите
\begin{enumerate}
	\item $\P(\{X = 0\})$
	\item $\P(\{X > 0\})$
	\item $\P(\{X < 0\})$
	\item $\E(X)$
	\item $\Var(X)$
	\item  наиболее вероятное значение, которое принимает случайная величина $X$
\end{enumerate}


\item В лифт 10-этажного дома на первом этаже вошли 5 человек. Вычислите
вероятность того, что на 6-м этаже выйдет хотя бы один человек.


\item В лифт 10-этажного дома на первом этаже вошли 5 человек. Вычислите
вероятность того, что на 6-м этаже не выйдет ни один человек.


\item При работе некоторого устройства время от времени возникают сбои.
Количество сбоев за сутки имеет распределение Пуассона. Среднее количество сбоев за сутки равно 3. Найти вероятность того, что в течение суток произойдет хотя бы один сбой.


\item При работе некоторого устройства время от времени возникают сбои.
Количество сбоев за сутки имеет распределение Пуассона. Среднее количество сбоев за сутки равно 3. Найти вероятность того, что за двое суток не произойдет ни одного сбоя.


\item Пусть случайная величина $X$ имеет плотность распределения

\[
f_X(x) =
	\begin{cases}
	c,\text{ при }  x \in [-1; 1] \\
	0,\text{ при } x \notin  [-1; 1] \\
	\end{cases}
\]

Найдите
\begin{enumerate}
	\item константу $c$
	\item $\P(\{X \leq 0\})$
	\item $\P(\{X \in [\frac{1}{2}; \frac{3}{2}]\})$
	\item $\P(\{X \in [2;3]\}$
	\item $F_X(x)$
\end{enumerate}


\item Пусть случайная величина $X$ имеет плотность распределения

\[
f_X(x) =
	\begin{cases}
	c,\text{ при }  x \in [-1; 1] \\
	0,\text{ при } x \notin  [-1; 1] \\
	\end{cases}
\]

Найдите
\begin{enumerate}
	\item константу $c$
	\item $\E(X)$
	\item $\E(X^2)$
	\item $\Var(X)$
	\item $\E(|X|)$
\end{enumerate}


\item Пусть случайная величина $X$ имеет плотность распределения

\[
f_X(x) =
	\begin{cases}
	cx,\text{ при }  x \in [0; 1] \\
	0,\text{ при } x \notin  [0; 1] \\
	\end{cases}
\]

Найдите
\begin{enumerate}
	\item константу $c$
	\item $\P(\{X \leq \frac{1}{2}\})$
	\item $\P(\{X \in [\frac{1}{2}; \frac{3}{2}]\})$
	\item $\P(\{X \in [2;3]\}$
	\item $F_X(x)$
\end{enumerate}


\item Пусть случайная величина $X$ имеет плотность распределения

\[
f_X(x) =
	\begin{cases}
	cx,\text{ при }  x \in [0; 1] \\
	0,\text{ при } x \notin  [0; 1] \\
	\end{cases}
\]

Найдите
\begin{enumerate}
	\item константу $c$
	\item $\E(X)$
	\item $\E(X^2)$
	\item $\Var(X)$
	\item $\E(\sqrt{X})$
\end{enumerate}
\end{enumerate}



\subsubsection*{Ответы}

\begin{enumerate}
	\item
			\begin{enumerate}
				\item 0.25
				\item 0.6
				\item нет
			\end{enumerate}
	\item
			\begin{enumerate}
				\item 0.5
				\item  0.75
				\item нет
			\end{enumerate}
	\item $\frac{4}{10 \cdot 11 \cdot 12 \cdot 13}$
	\item $\frac{4}{10 \cdot 11 \cdot 12 \cdot 13}$
	\item 0.5


	\item 0.42
	\item 0.028
	\item $\frac{5}{7}$
	\item
			\begin{enumerate}
				\item 0.5
				\item 0.75
				\item 0
				\item 0.5
			\end{enumerate}
	\item
			\begin{enumerate}
				\item 0.5
				\item  0
				\item  0.5
				\item  0.5
				\item  0.5
			\end{enumerate}
	\item
			\begin{enumerate}
				\item 0.25
				\item 0.75
				\item 0
				\item 0.5
			\end{enumerate}
	\item
			\begin{enumerate}
				\item 0.25
				\item 0.25
				\item 0.75
				\item 0.5
				\item 0.75
			\end{enumerate}
	\item
			\begin{enumerate}
				\item $\left( \frac{1}{4} \right) ^4$
				\item $1 - \left( \frac{1}{4} \right) ^4$
				\item 0
				\item 3
				\item 0.75
				\item 2, 3
			\end{enumerate}
	\item
			\begin{enumerate}
				\item $\left( \frac{3}{5} \right) ^5$
				\item $1 - \left( \frac{3}{5} \right) ^5$
				\item 0
				\item 2
				\item 1.2
				\item 2
			\end{enumerate}
	\item
			\begin{enumerate}
				\item $e^{-100}$
				\item $1 - e^{-100}$
				\item 0
				\item 100
				\item 100
			\end{enumerate}
	\item
			\begin{enumerate}
				\item $e^{-101}$
				\item $1 - e^{-101}$
				\item 0
				\item 101
				\item 101
			\end{enumerate}
	\item $1 - \frac{8^5}{9^5}$
	\item $\frac{8^5}{9^5}$
	\item $1 - e^{-3}$
	\item $e^{-3}$
	\item
			\begin{enumerate}
				\item 0.5
				\item 0.25
				\item 0.125
				\item 1
			\end{enumerate}
	\item
			\begin{enumerate}
				\item 0.5
				\item 0.5
				\item $\frac{1}{3}$
				\item $\frac{1}{12}$
				\item 1
			\end{enumerate}
	\item
			\begin{enumerate}
				\item 2
				\item 0.25
				\item $\frac{3}{4}$
				\item 1
			\end{enumerate}
	\item
			\begin{enumerate}
				\item 2
				\item 0.5
				\item 0.5
				\item 0
				\item 0.8
			\end{enumerate}
\end{enumerate}




\subsection{Контрольная работа 1, базовый поток, 24.10.2017}

\subsubsection*{Минимум}

\begin{enumerate}
\item Функция распределения случайной величины: определения и свойства.
\item Экспоненциальное распределение: определение, математическое ожидание и дисперсия.
\item В операционном отделе банка работает 80\% опытных сотрудников и 20\% неопытных. Вероятность совершения ошибки при очередной банковской операции опытным сотрудником равна $0.01$, а неопытным — $0.1$. Известно, что при очередной банковской операции была допущена ошибка. Найдите вероятность того, что ошибку допустил неопытный сотрудник.
\item При работе некоторого устройства время от времени возникают сбои. Количество сбоев за сутки имеет распределение Пуассона. Среднее количество сбоев за сутки равно 3. Найдите вероятность того, что за двое суток не произойдет ни одного сбоя.

\end{enumerate}

\subsubsection*{Задачи}

\begin{enumerate}

\item Правильный кубик подбрасывают один раз. Событие $A$ — выпало чётное число, событие $B$ — выпало число кратное трём, событие $C$ — выпало число, большее трёх.

\begin{enumerate}
\item Сформулируйте определение независимости двух событий;
\item Определите, какие из пар событий $A$, $B$ и $C$ будут независимыми.
\end{enumerate}


\item Теоретический минимум (ТМ) состоит из 10 вопросов, задачный (ЗМ) — из 24 задач.
Каждый вариант контрольной содержит два вопроса из ТМ и две задачи из ЗМ.
Чтобы получить за контрольную работу оценку 4 и выше, необходимо и достаточно правильно ответить на каждый вопрос ТМ и задачу ЗМ доставшегося варианта. Студент Вася принципиально выучил только $k$ вопросов ТМ и две трети ЗМ.
\begin{enumerate}
\item Сколько всего можно составить вариантов, отличающихся хотя бы одним заданием в ТМ или ЗМ части? Порядок заданий внутри варианта не важен.
\item Найдите вероятность того, что Вася правильно решит задачи ЗМ;
\item Дополнительно известно, что Васина вероятность правильно ответить на вопросы ТМ, составляет $1/15$. Сколько вопросов ТМ выучил Вася?
\end{enumerate}

\item Производитель молочных продуктов выпустил новый низкокалорийный йогурт Fit и утверждает, что он вкуснее его более калорийного аналога Fat.
Четырем независимым экспертам предлагают выбрать наиболее вкусный йогурт из трёх, предлагая им в одинаковых стаканчиках в случайном порядке два Fat и один Fit.
Предположим, что йогурты одинаково привлекательны.
Величина $\xi$ — число экспертов, отдавших предпочтение Fit.
\begin{enumerate}
\item Какова вероятность, что большинство экспертов выберут Fit?
\item Постройте функцию распределения величины $\xi$;
\item Каково наиболее вероятное число экспертов, отдавших предпочтение йогорту Fit?
\item Вычислите математическое ожидание и дисперсию $\xi$.
\end{enumerate}

\item Дядя Фёдор каждую субботу закупает в магазине продукты по списку, составленному котом Матроскином. Список не изменяется, и в него всегда входит 1 кг сметаны, цена которого является равномерно распределённой величиной $\alpha$, принимающей значения от 250 до 1000 рублей. Стоимость остальных продуктов из списка в тысячах рублей является случайной величиной $\xi$ с функцией распределения

\[
F(x)=\begin{cases}
1-\exp(-x^2 ), \text{ если } x \geq 0 \\
0, \text{ иначе.}\\
\end{cases}
\]

\begin{enumerate}
\item Какую сумму должен выделить кот Матроскин дяде Фёдору, чтобы её достоверно хватало на покупку сметаны?
\item Какую сумму должен выделить кот Матроскин дяде Фёдору, чтобы Дядя Фёдор с вероятностью 0.9 мог оплатить продукты без сметаны?
\item Найдите математическое ожидание стоимости продуктов без сметаны;
\item Найдите математическое ожидание стоимости всего списка.
\item Какова вероятность того, что общие расходы будут в точности равны их математическому ожиданию?
\end{enumerate}

Подсказка: $\int_0^{\infty} \exp(-x^2) \, dx = \sqrt{\pi} / 2$.

\item Эксперт с помощью детектора лжи пытается определить, говорит ли подозреваемый правду. Если подозреваемый говорит правду, то эксперт ошибочно выявляет ложь с вероятностью 0.1. Если подозреваемый обманывает, то эксперт выявляет ложь с вероятностью 0.95.

В деле об одиночном нападении подозревают десять человек, один из которых виновен и будет лгать, остальные невиновны и говорят правду.

\begin{enumerate}
\item Какова вероятность того, что детектор покажет, что конкретный подозреваемый лжёт?
\item Какова вероятность того, что подозреваемый невиновен, если детектор показал, что он лжёт?
\item Какова вероятность того, что эксперт точно выявит преступника?
\item Какова вероятность того, что эксперт ошибочно выявит  преступника, то есть покажет, что лжёт невиновный, а все остальные говорят правду?
\end{enumerate}



\end{enumerate}


\subsection{Контрольная работа 1, базовый поток, 24.10.2017, решения}

\begin{enumerate}
\item
\begin{enumerate}
\item События называются независимыми, если  $ \P(A \cap B) = \P(A) \cdot \P(B)$
\item Запасёмся всеми нужными вероятностями:

$\P(A) = \frac{1}{2}$

$\P(B) = \frac{1}{3}$

$\P(C) = \frac{1}{2}$

$\P(A \cap C) = \frac{1}{3} $ — выпадет чётое число больше трёх

$\P(A \cap B)  = \frac{1}{6}$ — выпадет чётное число, кратное трём

$\P(A \cap C) = \frac{1}{6}$ — выпадет число, большее трёх и кратное трём

Теперь можно проверять независимость:

$\P(A \cap C) \neq \P(A) \cdot \P(C) \Rightarrow$  не являются независимыми

$ \P(A \cap B) = \P(A) \cdot \P(B) \Rightarrow$ являются независимыми

$ \P(B \cap C) = \P(B) \cdot \P(C) \Rightarrow$ являются независимыми

\end{enumerate}
\item
\begin{enumerate}
\item Количество возможных вариантов ТМ: $ C_{10}^2 $,  количество возможных вариантов ЗМ: $ C_{24}^2 $. Количество их возможных сочетаний: $ C_{10}^2 \cdot C_{24}^2$ , где $ C_n^k = \frac{n!}{k!(n-k)!}$.
\item По классическому определению вероятностей, предполагая исходы равновероятными, искомая вероятность равна $ \frac{C_{16}^2}{C_{24}^2} $
\item По тому же принципу:
\[
\frac{C_k^2}{C_{10}^2} = \frac{1}{15} \Rightarrow \frac{\frac{k!}{2!(k-2)!}}{\frac{10!}{2! \cdot 8!}} = \frac{1}{15} \Rightarrow \frac{(k-1)k}{2}\frac{ 2}{9 \cdot 10} = \frac{1}{15}
\]
Получаем квадратное уравнение вида $ k^2 - k - 6 = 0 $ с корнями $-2$ и $3$. Так как $k$ не может быть отрицательным, ответ $3$.
\end{enumerate}
\item
\begin{enumerate}
\item Если эксперт отдаёт предпочтение Fit, то это можно интерпретировать как «успех» в схеме Бернулли. Так как $\xi$ - количество успехов, $ k \in [0;4]$, $p = \frac{1}{3} $, то
\[
\P(\xi = k) = C_n^k(p)^k(1-p)^{n-k}
\]

Большинство означает, что либо три, либо четыре эксперта выбрали Fit.
\[
\P(\xi = 3) = C_4^3\left(\frac{1}{3}\right)^3 \left(\frac{2}{3}\right)^{1} = \frac{8}{81}
\]
\[
\P(\xi = 4) = C_4^4\left(\frac{1}{3}\right)^4 \left(\frac{2}{3}\right)^{0} = \frac{1}{81}
\]
\[
\P( \xi > 2) =  \frac{9}{81}
\]
\item Аналогично:

\[ \P(\xi = 0) = C_4^0\left(\frac{1}{3}\right)^0 \left(\frac{2}{3}\right)^{4} = \frac{16}{81}\]

\[ \P(\xi = 1) = C_4^1\left(\frac{1}{3}\right)^1 \left(\frac{2}{3}\right)^{3} = \frac{32}{81}\]

\[ \P(\xi = 2) = C_4^2\left(\frac{1}{3}\right)^2 \left(\frac{2}{3}\right)^{2} = \frac{24}{81}\]

\begin{figure}[h!]
    \noindent\centering{
    \includegraphics[width=80mm]{images/kr1_2017_3.png}
    }
    \caption{Функция распределения}
    \label{cdf_kr2017}
\end{figure}

\item Все вероятности посчитаны, видим, что наибольшая достигается при $\xi=1$.
\item $\E(X) = np = \frac{4}{3} $, $ \Var(X) = npq = \frac{8}{9}$
\end{enumerate}
\item
\begin{enumerate}
\item Так как указано, что цена сметаны распределена равномерно на отерзке $[250, 1000]$, максимальное значение цены — $1000$, это и есть необходимая сумма.
\item Вспомним, что функция распределения $F(x) = \P(X \leq x)$, нужно найти такой $x$, что $ \P(X \leq x)=0.9$:
\[
0.9 = 1 - \exp({-x^{2}}) \Rightarrow \exp(-x^{2}) = 0.1 \Rightarrow -x^2 = \ln(0.1)  \Rightarrow x=  \sqrt{-\ln(0.1)}
\]
\item Взяв производную от функции распределения списка без сметаны, получим функцию плотности:
\[
f_X(x) =
\begin{cases}
2x\exp(-x^2) & x \ge 0 \\
0 & \text{иначе}
\end{cases}
\]
Найдём математическое ожидание:
\[
\int_{0}^{+\infty}2x^2\exp({-x^2}) dx = -x \exp({-x^2})\big|_0^{+\infty} + \int_{0}^{+\infty}\exp({-x^2}) dx = \frac{\sqrt{\pi}}{2}
\]
\item Математическое ожидание суммы случайных величин равно сумме математических ожиданий случайных влечин, если они существуют. Математическое ожидание от цены сметаны равно: $ \frac{1000 + 250}{2} = 625 $
Математическое ожидание списка без сметаны было найдено в предыдущем пункте, его осталось перевести в рубли. Получаем ответ: $ 625 + \frac{\sqrt{\pi}}{2} \cdot 1000 $.
\item Так как обе величины имеют абсолютно непрерывные распределения, вероятность попасть в конкретную точку равна нулю.
\end{enumerate}
\item
\begin{enumerate}
\item $\P(\text{детектор показл ложь и подозреваемый лжёт}) = 0.9 \cdot 0.1 + 0.1 \cdot 0.95 = 0.185$
\item $\P(\text{невниовен}|\text{детектор показал ложь}) = \frac{0.9\cdot0.1}{0.185} = \frac{90}{185}$
\item $\P(\text{эксперт точно выявит преступника}) = (0.9)^9 \cdot 0.95$
\item $\P(\text{эксперт ошибочно выявит преступника}) = 9 \cdot 0.1 \cdot 0.9^8\cdot 0.05$
\end{enumerate}


%\item
%$\P(Ложь|Лжёт) = 0.95 $

%$\P(Ложь|Не лжёт) = 0.1 $

%$\P({Лжёт}) = \frac{1}{10} $

%\P(\text{Не лжёт}) = \dfrac{9}{10} $

%\begin{enumerate}
%	\item
%	По формуле полной вероятности: \[  \P(Ложь) = 0.95 \cdot 0.1 + 0.1 \cdot 0.9 = 0.185 \]
%	\item
%	По формуле Байеса: \[ \P(\text{Не лжёт}|Ложь) = \dfrac{0.1 \cdot 0.9}{0.185} = 0.486\]
%    \item
%    Событие "эксперт точно выявит преступника" соответствует событию "эксперт выявит, что лгун лжет, и      что остальные говорят правду".  Таким образом:  \[ \P = 0.1 \cdot 0.95 + 0.9 \cdot 0.9  = 0.905\]
%    \item
%    Это означает, что эксперт выберет 8 человек из 9 невиновных и скажет, что они говорят правду
%    (количество вариантов выбрать так людей  $С_9^8$ ), а также выберет 1 виновного и скажет, что он
%    говорит правду (количество вариантов это сделать 1), и выберет одного невиновного и скажет, что он
%    лжет (количество вариантов выбрать так человека $С_9^1$ ). Просуммируем все с учетом вероятностей,
%    указанных в условии: \[ \P = с_9^8 \cdot 0.9 \cdot 0.9 \cdot C_9^1 \cdot 0.9 \cdot 0.1 \cdot 1 \cdot 0.1
%    \cdot 0.05 = 0. 26244\]

%\end{enumerate}
\end{enumerate}



\subsection{Контрольная работа 1, ИП, 24.10.2017}


Ровно 272 года назад императрица Елизавета повелела завезти во дворцы котов для ловли мышей.


\begin{enumerate}

\item В отсутствии кота Леопольда мыши Белый и Серый подкидывают по очереди игральный додекаэдр
%\footnote{Леопольд подсказывает по случаю праздника, что у додекаэдра 12 граней :)}
.
Сыр достаётся тому, кто первым выкинет число 6. Начинает подкидывать Белый.

\begin{enumerate}
  \item Какова вероятность того, что сыр достанется Белому?
  \item Сколько в среднем бросков продолжается игра?
  \item Какова дисперсия числа бросков?
\end{enumerate}

\item Микки Маус, Белый и Серый решили устроить труэль из любви к мышки Мии. Сначала стреляет Микки, затем Белый, затем Серый, затем снова Микки и так до тех пор, пока в живых не останется только один.

Прошлые данные говорят о том, что Микки попадает с вероятностью $1/3$, Белый — с вероятностью $2/3$, а Серый стреляет без промаха.

Найдите оптимальную стратегию каждого мыша.

\item Микки Маус, Белый и Серый пойманый злобным котом Леопольдом до начала труэли. И теперь Леопольд будет играть с ними в странную игру.

В комнате три закрытых внешне неотличимых коробки: с золотом, серебром и платиной. Общаться после начала игры мыши не могут, но могут заранее договориться о стратегии.

Правила игры таковы. Кот Леопольд будет заводить мышей в комнату по очереди. Каждый из мышей может открыть
две коробки по своему выбору. Перед следующим мышом коробки закрываются.

Если Микки откроет коробку с золотом, Белый
— с серебром, а Серый — с платиной, то они выигрывают. Если
хотя бы один из мышей не найдёт свой металл, то Леопольд их съест.
\begin{enumerate}
\item Какова оптимальная стратегия?
\item Какова вероятность выигрыша при использовании оптимальной стратегии?
\end{enumerate}

\item Накануне войны Жестокий Тиран Мышь очень большой страны издал указ. Отныне за каждого новорождённого мыша-мальчика семья получает денежную премию, но если в семье рождается вторая мышка-девочка, то всю семью убивают. Бедные жители страны запуганы и остро нуждаются в деньгах, поэтому в каждой семье мыши будут появляться до тех пор, пока не родится первая мышка-девочка.

\begin{enumerate}
  \item Каким будет среднее число детей в мышиной семье?
  \item Какой будет доля мышей-мальчиков в стране?
  \item Какой будет средняя доля мышей-мальчиков в случайной семье?
  \item Сколько в среднем мышей-мальчиков в случайно выбираемой семье?
\end{enumerate}

\item Вальяжный кот Василий положил на счёт в банке на Гаити один гурд. Сумма на счету растёт непрерывно с постоянной ставкой в течение очень длительного промежутка времени. В случайный момент этого промежутка кот Василий закрывает свой вклад.

Каков закон распределения первой цифры полученной Василием суммы?

\begin{comment}
\item Начинающий трейдер Афанасий совершает не более одной сделки в день.

Если в какой-то день у трейдера Афанасия есть акция, то за этот день он равновероятно продаёт или не продаёт её. Если в какой-то день у трейдера Афанасия нет акций, то он равновероятно покупает или не покупает одну акцию.

Найдите ожидаемую прибыль Афанасия, если известно, что реализовывал он свою стратегию 100 дней, в начале акции стоили по 50 рублей, в конце — по 80 рублей, максимум составил 120 рублей, а минимум — 30.


\item Страшный Мейн-кун разрубает палочку единичной длины на $10$ частей в случайных и независимых местах, равномерно распределённых по всех длине. Затем Страшный Мейн-кун выбирает случайно один из кусочков и возводит его длину в $24$ степень.

Какое в среднем число он получит?
\end{comment}



\subsection{Контрольная работа 1, ИП, 24.10.2017, решения}


\begin{enumerate}

\item[1.]

\begin{enumerate}
	\item Обозначим вероятность того, что сыр достанется Белому за $b$, если игра начинается с его броска. Получаем уравнение
\[
	b = \frac{1}{12} + \frac{11}{12} \frac{11}{12} b
\]

Пояснение: Как Белый может победить в исходной игре? Либо сразу выкинуть 6 с вероятностью $1/12$. Либо передать ход Серому ($11/12$), получить ход снова ($11/12$) и выиграть в продолжении игры. Продолжение игры по сути совпадает с исходной игрой.

\item Игра продолжается до тех пор, пока кто-то не выкинет «6». Для нахождения среднего количества бросков воспользуемся методом первого шага.

Обозначим среднее количество бросков нашей игры за $S$. Когда Белый бросает кубик, с вероятностью $\frac{1}{12}$ игра закончится за один бросок, а с вероятностью $\frac{11}{12}$ игра продолжится и ход перейдёт к Серому. Но та игра, которая начнётся, когда бросать будет Серый, ничем не отличается от предыдущей, поэтому среднее количество бросков в ней будет равно $S$. Однако мы попадём в эту игру, «потратив» один бросок. Таким образом мы получаем:

\[
S = \frac{1}{12} \cdot 1 + \frac{11}{12}(S +1)
\]

Получается, что $S = 12$, значит игра длится в среднем 12 бросков.
\end{enumerate}

\item[3.]

Для того, чтобы выжить, мышам нужно ещё до начала игры договориться о стратегии, которая позволит им с наибольшей вероятностью открыть нужные сундуки. Если хотя бы две мыши выберут одинаковый сундук, то их в любом случае съедят. Поэтому одной из оптимальных стратегий будет ещё до начала игры мышам договориться и назвать левый сундук золотым, сундук посередине серебряным, а правый — платиновым. Каждый мышонок должен открыть тот сундук, в честь которого назван необходимый ему металл. Если внутри он обнаруживает свой металл, то он выбирает этот сундук, если внутри находится не тот металл, мышонок открывает тот сундук, на который указывает лежащий внутри предмет.

Например, первым заходит Микки Маус. Он открывает золотой (левый) ящик. Если внутри лежит золото, то он выходит из комнаты. Если же внутри лежит, например, серебро, то Микки Маус открывает сундук посередине. Путём несложного перебора можно посчитать, что в 4 случаях из 6 мыши смогут найти нужный металл, поэтому вероятность выигрыша при данной стратегии равна $\frac{2}{3}$.

\item[5.]

Функция распределения дохода кота Василия, положившего один гурд на вклад, представляется в виде $m_t = 1\cdot e^{rt}$, где $r$ — процентная ставка, а $t$ — прошедшее время. Момент закрытия вклада Т равномерно распределён на отрезке от 0 до $a$, который очень велик, поэтому сумма, которую получит Василий, представима в виде $Z = e^{Y}$, где $Y \sim v[0; ra]$.

Вероятность того, что первая цифра будет равна 1, равна вероятности того, что доход Василия будет лежать в пределах от 1 до 2 гурдов, плюс вероятность того, что он лежит в пределах от 10 до 20 гурдов и т.д. Таким образом, можно представить эту вероятность, как:
\[
\P(N=1) = \P(e^Y \in [1;2) ) + \P(e^Y \in [10; 20) ) + \ldots
\]

Это выражение можно преобразовать таким образом:
\[
\P(N=1) = \P(Y \in [\ln 1; \ln2) ) + \P(Y \in [\ln 10; \ln 20) ) + \ldots
\]

Так как Y — равномерно распределённая величина, то $\P(Y \in [\ln 1; \ln2) ) = \frac{\ln 2 - \ln 1}{ra}$. Для последующих слагаемых вероятность рассчитывается таким же образом. Воспользовавшись свойством логарифма, можно заметить, что $\frac{\ln 20 - \ln 10}{ra} = \frac{\ln 2}{ra}$. Поэтому вероятность того, что на первом месте суммы вклада стоит единица, равна $n\cdot \frac{\ln 2}{ra}$, где $n$ -- количество слагаемых. Путём аналогичных рассуждений получаем, что вероятность того, что на первом месте стоит двойка, равна $n\cdot \frac{\ln 3- \ln 2}{ra}$. Из-за того, что $a$ велико, можно считать, что число слагаемых одинаково.

Т.к. на первом месте обязательно будет находиться какая-то цифра, то сумма вероятностей будет равна 1. Получаем:
\[
\dfrac{n}{ra}(\ln \frac{2}{1} + \ln \frac{3}{2} + \ldots + \ln \frac{10}{9}) = 1
\]

Таким образом $\frac{n}{ra} = \frac{1}{\ln 10}$. Получается, что вероятность того, что на первом месте стоит единица, равна:
\[
\P (N=1) = \dfrac{\ln 2}{\ln 10}
\]

Закон распределения первой цифры выводится сложением соответствующих вероятностей.

\end{enumerate}
\end{enumerate}



\subsection{Теоретический минимум к кр2}

\begin{enumerate}
\item Сформулируйте определение независимости событий, формулу полной вероятности.
\item Приведите определение условной вероятности случайного события, формулу Байеса.
\item Сформулируйте определение и свойства функции распределения случайной величины.
\item Сформулируйте определение и свойства функции плотности случайной величины.
\item Сформулируйте определение и свойства математического ожидания для абсолютно непрерывной случайной величины.
\item Сформулируйте определение и свойства математического ожидания для дискретной случайной величины.
\item Сформулируйте определение и свойства дисперсии случайной величины.
\item Сформулируйте определения следующих законов распределений: биномиального, Пуассона, шеометрического, равномерного, экспоненциального, нормального. Укажите математическое ожидание и дисперсию.
\item Сформулируйте определение функции совместного распределения двух случайных величин, независимости случайных величин. Укажите, как связаны совместное распределение и частные распределения компонент случайного вектора.
\item	Сформулируйте определение и свойства совместной функции плотности двух случайных величин, сформулируйте определение независимости случайных величин.
\item Сформулируйте определение и свойства ковариации случайных величин.
\item Сформулируйте определение и свойства корреляции случайных величин.
\item Сформулируйте определение и свойства условной функции плотности.
\item Сформулируйте определение  условного математического ожидания $\E(Y|X=x)$ для совместного дискретного и совместного абсолютно непрерывного распределений.
\item Сформулируйте определение математического ожидания и ковариационной матрицы случайного вектора и их свойства.
\item Сформулируйте неравенство Чебышёва и неравенство Маркова.
\item Сформулируйте закон больших чисел в слабой форме.
\item Сформулируйте центральную предельную теорему.
\item Сформулируйте теорему Муавра—Лапласа.
\item Сформулируйте определение сходимости по вероятности для последовательности случайных величин.

\end{enumerate}


\subsection{Задачный минимум кр 2}

\begin{enumerate}

\item Пусть задана таблица совместного распределения случайных величин $X$ и $Y$.

\begin{center}\begin{tabular}{lccc}
\toprule
 $X$ \textbackslash $Y$    & $-1$  & $0$   & $1$   \\ \midrule
$-1$                 & $0.2$ & $0.1$ & $0.2$ \\
 $1$                 & $0.1$ & $0.3$ & $0.1$ \\ \bottomrule
\end{tabular}\end{center}


Найдите
\begin{enumerate}
\item $\P(X = -1)$
\item $\P(Y = -1)$
\item $\P(X = -1 \cap Y = -1 )$
\item Являются ли случайные величины $X$ и $Y$ независимыми?
\item $F_{X,Y}(-1,0)$
\item Таблицу распределения случайной величины $X$
\item Функцию $F_{X}(x)$ распределения случайной величины $X$.
\item Постройте график функции $F_{X}(x)$ распределения случайной величины $X$.
\end{enumerate}

\item Пусть задана таблица совместного распределения случайных величин $X$ и $Y$.


\begin{center}\begin{tabular}{lccc}
\toprule
 $X$ \textbackslash $Y$    & $-1$  & $0$  & $1$   \\ \midrule
$-1$                 & $0.2$ & $0.1$ & $0.2$ \\
 $1$                 & $0.2$ & $0.1$ & $0.2$ \\ \bottomrule
\end{tabular}\end{center}

Найдите
\begin{enumerate}
\item $\P(X = 1)$,
\item $\P(Y = 1)$,
\item $\P(X = 1 \cap Y = 1)$
\item Являются ли случайные величины $X$ и $Y$ независимыми?
\item $F_{X,Y}(1,0)$
\item Таблицу распределения случайной величины $Y$
\item Функцию $F_{Y}(y)$ распределения случайной величины $Y$
\item Постройте график функции $F_{Y}(y)$ распределения случайной величины $Y$.
\end{enumerate}

\item Пусть задана таблица совместного распределения случайных величин $X$ и $Y$.

\begin{center}\begin{tabular}{lccc}
\toprule
 $X$ \textbackslash $Y$    & $-1$  & $0$   & $1$   \\ \midrule
$-1$                 & $0.2$ & $0.1$ & $0.2$ \\
 $1$                 & $0.1$ & $0.3$ & $0.1$ \\ \bottomrule
\end{tabular}\end{center}

Найдите
\begin{enumerate}
    \item $\E(X),$
    \item $\E(X^{2}),$
	\item $\Var(X),$
    \item $\E(Y),$
    \item $\E(Y^{2}),$
    \item $\Var(Y),$
    \item $\E(XY),$
	\item $\Cov(X,Y)$
    \item $\Corr(X,Y)$
    \item Являются ли случайные величины $X$ и $Y$ некоррелированными?
\end{enumerate}

\item Пусть задана таблица совместного распределения случайных величин $X$ и $Y$.

\begin{center}\begin{tabular}{lccc}
\toprule
 $X$ \textbackslash $Y$    & $-1$  &$ 0 $  & $1 $  \\ \midrule
$-1$                 & $0.2$ & $0.1$ & $0.2$ \\
 $1$                 & $0.2$ & $0.1$ & $0.2$ \\ \bottomrule
\end{tabular}\end{center}

Найдите
\begin{enumerate}
    \item $\E(X),$
    \item $\E(X^{2}),$
	\item $\Var(X),$
    \item $\E(Y),$
    \item $\E(Y^{2}),$
    \item $\Var(Y),$
    \item $\E(XY),$
	\item $\Cov(X,Y)$
    \item $\Corr(X,Y)$
    \item Являются ли случайные величины $X$ и $Y$ некоррелированными?
\end{enumerate}

\item
Пусть задана таблица совместного распределения случайных величин $X$ и $Y$.

\begin{center}\begin{tabular}{lccc}
\toprule
 $X$\textbackslash $Y$    & $-1$  & $0$   & $1$   \\ \midrule
$-1$                 & $0.2$ & $0.1$ & $0.2$ \\
 $1$                 & $0.1$ & $0.3$ & $0.1$ \\ \bottomrule
\end{tabular}\end{center}

Найдите
\begin{enumerate}
\item $\P(X = -1 | Y = 0)$
\item $\P(Y = 0 | X = -1)$
\item таблицу условного распределения случайной величины $Y$ при условии $X = -1$
\item условное математическое ожидание случайной величины $Y$ при $X = -1$
\item условную дисперсию случайной величины $Y$
при условии $X = -1$
\end{enumerate}

\item Пусть задана таблица совместного распределения случайных величин $X$ и $Y$.

\begin{center}\begin{tabular}{lccc}
\toprule
 $X$ \textbackslash $Y$    & $-1$  & $0$   & $1$   \\ \midrule
$-1$                 & $0.2$ & $0.1$ & $0.2$ \\
 $1$                 & $0.2$ & $0.1$ & $0.2$ \\ \bottomrule
\end{tabular}\end{center}

Найдите
\begin{enumerate}
\item $\P(X = 1 | Y = 0)$
\item $\P(Y = 0 | X = 1)$
\item таблицу условного распределения случайной величины $Y$ при условии $X = 1$
\item условное математическое ожидание случайной величины $Y$ при $X = 1$
\item условную дисперсию случайной величины $Y$
при условии $X = 1$
\end{enumerate}

\item Пусть $\E(X)=1$, $\E(Y)=2$, $\Var(X) = 3$, $\Var(Y) = 4$, $\Cov(X,Y) = -1$. Найдите
\begin{enumerate}
\item $\E(2X + Y - 4)$
\item $\Var(3Y + 3)$
\item $\Var(X - Y)$
\item $\Var(2X - 3Y +1)$
\item $\Cov(X+ 2Y + 1,3X - Y -1)$
\item $\Corr(X + Y, X - Y)$
\item Ковариационную матрицу случайного вектора $Z = (X\hspace*{0.4cm} Y)$ \end{enumerate}


\item Пусть $\E(X)=-1$, $\E(Y)=2$, $\Var(X) = 1$, $\Var(Y) = 2$, $\Cov(X,Y) = 1$. Найдите
\begin{enumerate}
\item $\E(2X + Y - 4)$
\item $\Var(2Y + 3)$
\item $\Var(X - Y)$
\item $\Var(2X - 3Y +1)$
\item $\Cov(3X+ Y + 1,X - 2Y -1)$
\item $\Corr(X + Y, X - Y)$
\item Ковариационную матрицу случайного вектора $Z = (X\hspace*{0.4cm}Y)$
\end{enumerate}

\item Пусть случайная величина $X$ имеет стандартное нормальное распределение.

Найдите
\begin{enumerate}
\item $\P(0 < X < 1)$
\item $\P(X > 2)$
\item $\P(0 < 1 - 2X \leq 1)$
\end{enumerate}

\item Пусть случайная величина $X$ имеет стандартное нормальное распределение.

Найдите
\begin{enumerate}
\item $\P(-1 < X < 1)$
\item $\P(X < -2)$
\item $\P(-2 < -X + 1 \leq 0)$
\end{enumerate}

\item Пусть случайная величина $X \sim \cN(1,4)$. Найдите $\P(1<X<4)$

\item Пусть случайная величина $X \sim \cN(2,4)$. Найдите $\P(-2<X<4)$

\item Случайные величины $X$ и $Y$ независимы и  имеют нормальное распределение, $\E(X) = 0 $, $\Var(X) = 1$, $\E(Y) = 2$, $\Var(Y) = 6$. Найдите $\P(1 < X + 2Y < 7)$.

\item Случайные величины $X$ и $Y$ независимы и  имеют нормальное распределение, $\E(X) = 0 $, $\Var(X) = 1$, $\E(Y) = 3$, $\Var(Y) = 7$. Найдите $\P(1 < 3X + Y < 7)$.

\item Игральная кость подбрасывается $420$ раз. При помощи центральной предельной теоремы приближенно найти вероятность того, что суммарное число очков будет находиться в пределах от $1400$ до $1505$?

\item При выстреле по мишени стрелок попадает в десятку с вероятностью $0.5$, в девятку – $0.3$, в восьмерку – $0.1$, в семерку – $0.05$, в шестерку – $0.05$.
Стрелок сделал $100$ выстрелов. При помощи центральной предельной теоремы приближенно найти вероятность того, что он набрал не менее 900 очков?

\item Предположим, что на станцию скорой помощи поступают вызовы, число которых распределено по закону Пуассона с параметром $\lambda = 73$, и в разные сутки их количество не зависит друг от друга. При помощи центральной предельной теоремы приближенно найти вероятность того, что в течение года (365 дней) общее число вызовов будет в пределах от $26500$ до $26800$.

\item Число посетителей магазина (в день) имеет распределение Пуассона с математическим ожиданием $289$. При помощи центральной предельной теоремы приближенно найти вероятность того, что за $100$ рабочих дней суммарное число посетителей составит от $28550$ до $29250$ человек.

\item Пусть плотность распределения случайного вектора $(X,Y)$ имеет вид
\begin{center} $f_{X,Y}(x,y) = \begin{cases} x+y, & \text{при } (x,y) \in [0;1] \times [0;1] \\ 0 , & \text{при } (x,y) \not\in [0;1] \times [0;1] \end{cases}$  \end{center}

Найдите
\begin{enumerate}
\item $\P(X \leq \frac{1}{2} \cap Y \leq \frac{1}{2})$,
\item $\P(X\leq Y)$,
\item $f_{X}(x)$,
\item $f_{Y}(y)$,
\item Являются ли случайные величины $X$ и $Y$ независимыми?
\end{enumerate}

\item Пусть плотность распределения случайного вектора $(X,Y)$ имеет вид
\begin{center} $f_{X,Y}(x,y) = \begin{cases} 4xy, & \text{при } (x,y) \in [0;1] \times [0;1] \\ 0 , & \text{при } (x,y) \not\in [0;1] \times [0;1] \end{cases}$  \end{center}

Найдите
\begin{enumerate}
\item $\P(X \leq \frac{1}{2} \cap Y \leq \frac{1}{2})$,
\item $\P(X\leq Y)$,
\item $f_{X}(x)$,
\item $f_{Y}(y)$,
\item Являются ли случайные величины $X$ и $Y$ независимыми?
\end{enumerate}

\item Пусть плотность распределения случайного вектора $(X,Y)$ имеет вид
\begin{center} $f_{X,Y}(x,y) = \begin{cases} x+y, & \text{при } (x,y) \in [0;1] \times [0;1] \\ 0 , & \text{при } (x,y) \not\in [0;1] \times [0;1] \end{cases}$  \end{center}

Найдите
\begin{enumerate}
\item $\E(X)$,
\item $\E(Y)$,
\item $\E(XY)$,
\item $\Cov(X,Y)$,
\item $\Corr(X,Y)$.
\end{enumerate}

\item Пусть плотность распределения случайного вектора $(X,Y)$ имеет вид
\begin{center} $f_{X,Y}(x,y) = \begin{cases} 4xy, & \text{при } (x,y) \in [0;1] \times [0;1] \\ 0 , & \text{при } (x,y) \not\in [0;1] \times [0;1] \end{cases}$  \end{center}

Найдите
\begin{enumerate}
\item $\E(X)$,
\item $\E(Y)$,
\item $\E(XY)$,
\item $\Cov(X,Y)$,
\item $\Corr(X,Y)$.
\end{enumerate}

\item Пусть плотность распределения случайного вектора $(X,Y)$ имеет вид
\begin{center} $f_{X,Y}(x,y) = \begin{cases} x+y, & \text{при } (x,y) \in [0;1] \times [0;1] \\ 0 , & \text{при } (x,y) \not\in [0;1] \times [0;1] \end{cases}$  \end{center}

Найдите
\begin{enumerate}
\item $f_{Y}(y)$,
\item $f_{X|Y}\left(x|\frac{1}{2}\right)$
\item $\E\left(X|Y = \frac{1}{2}\right)$
\item $\Var\left(X|Y = \frac{1}{2}\right)$
\end{enumerate}

\item Пусть плотность распределения случайного вектора $(X,Y)$ имеет вид
\begin{center} $f_{X,Y}(x,y) = \begin{cases} 4xy, & \text{при } (x,y) \in [0;1] \times [0;1] \\ 0 , & \text{при } (x,y) \not\in [0;1] \times [0;1] \end{cases}$  \end{center}

Найдите
\begin{enumerate}
\item $f_{Y}(y)$,
\item $f_{X|Y}\left(x|\frac{1}{2}\right)$
\item $\E\left(X|Y = \frac{1}{2}\right)$
\item $\Var\left(X|Y = \frac{1}{2}\right)$
\end{enumerate}

\end{enumerate}

\subsection*{Ответы}

\begin{enumerate}

\item
\begin{enumerate}
\item   $0.5 $
\item   $0.3$
\item   $0.2$
\item   нет
\item   $0.3$
\item
\begin{tabular}{lrr}
\toprule
$X$ & $-1$  & $1$   \\ \midrule
$\P(\cdot)$ & $0.5$ & $0.5$ \\ \bottomrule
\end{tabular}
\item  $F_{X}(x) = \begin{cases}
0, & \text{при } x < -1 \\
0.5 , & \text{при } x \in [-1;1) \\
1, & \text{при }  x \geq 1
\end{cases}$
\end{enumerate}
\item
\begin{enumerate}
\item   $0.5$
\item   $0.4$
\item   $0.2$
\item   да
\item   $0.6$
\item
\begin{tabular}{lrrr}
\toprule
$Y$ & $-1$  & $0$   & $1$   \\ \midrule
$\P(\cdot)$ & $0.4$ & $0.2$ & $0.4$ \\ \bottomrule
\end{tabular}
\item   $F_{Y}(y) = \begin{cases}
0, & \text{при } y < -1 \\
0.4 , & \text{при } y \in [-1;0) \\
0.6, & \text{при }  y \in [0;1)\\
1, & \text{при } y \geq 1
\end{cases}$
\end{enumerate}

\item
\begin{enumerate}
\item   $0$
\item   $1$
\item  $1$
\item   $0$
\item   $0.6$
\item   $0.6$
\item   $0$
\item   $0$
\item   $0$
\item   да, являются некоррелированными, но нельзя утверждать, что являются независимыми
\end{enumerate}

\item
\begin{enumerate}
\item   $0$
\item   $1$
\item   $1$
\item   $0$
\item   $0.8$
\item   $0.8$
\item   $0$
\item   $0$
\item   $0$
\item   да, являются некоррелированными, но нельзя утверждать, что являются независимыми
\end{enumerate}

\item
\begin{enumerate}
\item   $0.25$
\item   $0.2$
\item   \begin{tabular}{lrrr}
\toprule
$Y$ | $\{X = -1\}$ & $-1$  & $0$   & $1$   \\ \midrule
$\P(\cdot)$              & $0.4$ & $0.2$ & $0.4$ \\ \bottomrule
\end{tabular}
\item   $0$
\item   $0.8 $
\end{enumerate}
\item
\begin{enumerate}
\item   $0.5$
\item   $0.2$
\item   \begin{tabular}{lrrr}
\toprule
$Y$ | $\{X = 1\}$ & $-1$  & $0$   & $1$   \\ \midrule
$\P(\cdot)$             & $0.4$ & $0.2$ & $0.4$ \\ \bottomrule
\end{tabular}
\item   $0$
\item   $0.8$
\end{enumerate}

\item
\begin{enumerate}
\item   $0 $
\item   $36$
\item  $9 $
\item   $60 $
\item  $-4$
\item   $\frac{-1}{3\sqrt{5}}$
\item  $\begin{pmatrix}
 3 & -1 \\
-1 & 4
\end{pmatrix}$
\end{enumerate}

\item
\begin{enumerate}
\item $-4$
\item $8 $
\item $1 $
\item $10 $
\item $-6$
\item$ \frac{-1}{\sqrt{5}}$

\item $\begin{pmatrix}
 1 & 1 \\
 1 & 2
\end{pmatrix}$
\end{enumerate}
\item
\begin{enumerate}
\item $0.3413$
\item $0.0228$
\item $0.1915$
\end{enumerate}

\item
\begin{enumerate}
\item $0.6826$
\item $0.0228  $
\item $0.1574  $
\end{enumerate}

\item $0.4332 $
\item $0.8185  $
\item $0.4514 $
\item $0.5328  $
\item $\approx 0.8185 $
\item $\approx 0.9115$
\item $\approx 0.6422 $
\item $\approx 0.9606$

\item
\begin{enumerate}
\item $0.125   $
\item $0.5 $
\item $f_{X}(x) = \begin{cases} x+\frac{1}{2}, & \text{при } x \in [0;1] \\ 0 , & \text{при } x \not\in [0;1] \end{cases}$
\item $f_{Y}(y) = \begin{cases} y+\frac{1}{2}, & \text{при } y \in [0;1] \\ 0 , & \text{при } y \not\in [0;1] \end{cases}$
\item нет
\end{enumerate}

\item
\begin{enumerate}
\item $\frac{1}{16}$

\item $\frac{1}{2}$

\item$f_{X}(x) = \begin{cases} 2x, & \text{при } x \in [0;1] \\ 0 , & \text{при } x \not\in [0;1] \end{cases}$

\item$f_{Y}(y) = \begin{cases} 2y, & \text{при } y \in [0;1] \\ 0 , & \text{при } y \not\in [0;1] \end{cases}$

\item да
\end{enumerate}

\item
\begin{enumerate}
\item $\frac{7}{12}$

\item $\frac{7}{12}$

\item $\frac{1}{3}$

\item $-\frac{1}{144}$

\item $-\frac{1}{11}$
\end{enumerate}

\item
\begin{enumerate}
\item $\frac{2}{3}$

\item $\frac{2}{3}$

\item $\frac{4}{9}$

\item 0

\item 0
\end{enumerate}

\item
\begin{enumerate}
\item $f_{Y}(y) = \begin{cases} y+\frac{1}{2}, & \text{при } y \in [0;1] \\ 0 , & \text{при } y \not\in [0;1] \end{cases}$

\item $f_{X|Y}(x|\frac{1}{2}) = \begin{cases} x+\frac{1}{2}, & \text{при } x \in [0;1] \\ 0 , & \text{при } x \not\in [0;1] \end{cases}$

\item $\frac{7}{12}$

\item $\frac{11}{144}$
\end{enumerate}

\item
\begin{enumerate}
\item $f_{Y}(y) = \begin{cases} 2y, & \text{при } y \in [0;1] \\ 0 , & \text{при } y \not\in [0;1] \end{cases}$

\item $f_{X|Y}(x|\frac{1}{2}) = \begin{cases} 2x, & \text{при } x \in [0;1] \\ 0 , & \text{при } x \not\in [0;1] \end{cases}$

\item $\frac{2}{3}$

\item $\frac{1}{18}$
\end{enumerate}
\end{enumerate}

\subsection{Контрольная работа 2, базовый поток, 09.12.2017}



\subsubsection{Минимум}
% 2 + 4 + 14 + 16

\begin{enumerate}
\item Приведите определение условной вероятности случайного события, формулу Байеса.
\item Сформулируйте определение и свойства функции плотности случайной величины. 
\item Сформулируйте определение  условного математического ожидания $\E(Y|X=x)$ для совместного дискретного и совместного абсолютно непрерывного распределений.
\item Сформулируйте неравенство Чебышёва и неравенство Маркова.

\item Задана таблица совместного распределения случайных величин $X$ и $Y$.
\begin{center}
\begin{tabular}{lccc}
\toprule
                       & $Y=-1$  & $Y=0$   & $Y=1$   \\ 
 \midrule
$X=0$                 & $0.2$ & $0.1$ & $0.3$ \\
 $X=1$                 & $0.2$ & $0.1$ & $0.1$ \\ 
 \bottomrule
\end{tabular}
\end{center}


\begin{enumerate}
    \item Найдите $F_{X,Y}(0, 0)$;
    \item Найдите $\E(X)$, $\E(X^2)$, $\E(Y)$, $\E(Y^2)$;
    \item Найдите $\Var(X)$, $\Var(Y)$;
    \item Найдите $\Cov(X, Y)$, $\Corr(X, Y)$
\end{enumerate}    
\item Плотность распределения случайного вектора $(X,Y)$ имеет вид
\[
f_{X,Y}(x,y) = 
\begin{cases} 
\frac{4x+10y}{7}, & \text{при } (x,y) \in [0;1] \times [0;1] \\ 
0 , & \text{при } (x,y) \not\in [0;1] \times [0;1] \\
\end{cases}
\]

\begin{enumerate}
\item Найдите $\P(X \leq Y)$;
\item Найдите функцию плотности $f_X(x)$;
\item Найдите $\E(X)$, $\E(Y)$ и $\Cov(X, Y)$;
\item Являются ли случайные величины $X$ и $Y$ независимыми?
\end{enumerate} 


\end{enumerate}

\subsubsection{Задачи}

\begin{enumerate}[resume]

\item Статистика авиакомпании «А» за много лет свидетельствует о том, что 10\% людей, купивших билет на самолет, не являются на рейс. Авиакомпания продала 330 билетов на 300 мест.
\begin{enumerate}
\item Какова вероятность, что всем явившимся на рейс пассажирам хватит места?
\item Укажите наибольшее число билетов, которое можно продавать на 300 мест, чтобы случаи переполнения случались не чаще, чем на одном из десяти рейсов.
\end{enumerate}

\item Сегодня акция компании «Ух» стоит 1 рубль. Каждый день акция может с вероятностью 0.7 вырасти на 1\%, с вероятностью 0.2999 упасть на 1\% и с вероятностью 0.0001 обесцениться (упасть на 100\%).
\begin{enumerate}
\item Считая изменение цены акции независимыми, найдите математическое ожидание её стоимости через 20 торговых дней.
\item Найдите предел по вероятности среднего изменения цены акции в процентах на бесконечном промежутке времени (Ответ обоснуйте).
\item Найдите математическое ожидание цены акции на бесконечном промежутке времени.
\item Инвестор вложил все свои средства в акции компании «Ух». Найдите вероятность его разорения на бесконечном промежутке времени.
\end{enumerate}


\end{enumerate}




\subsection{Контрольная работа 2, базовый поток, решения}

\begin{enumerate}
\item[7.]
\begin{enumerate}
\item Всем хватит места, если число явившихся на рейс пассажиров ($X$) не превысит $300$,
то есть нужно найти $\P(X \leq 300)$. Найдём матожидание и дисперсию
случайной величины $X$:
\begin{align*}
\E(X) &= np = 330 \cdot 0.9 = 297 \\
\Var(X) &= np(1-p) = 330 \cdot 0.9 \cdot 0.1 = 29.7
\end{align*}
Теперь посчитаем нужную вероятность:
\[
\P(X \leq 300) = \P \left(\frac{X - 297}{\sqrt{29.7}} \leq \frac{300 - 297}{\sqrt{29.7}} \right) = \P(\cN(0,1) \leq 0.55) \approx 0.709
\]
\item Вероятность переполнения не должна превышать $0.1$:
\begin{align*}
&\P(X > 300) < 0.1 \\
&\P\left(\frac{X - 0.9 \cdot n}{\sqrt{0.9 \cdot 0.1 \cdot n}} > \frac{300 - 0.9 \cdot n}{\sqrt{0.9 \cdot 0.1 \cdot n}} \right) < 0.1 \\
&\frac{300 - 0.9 \cdot n}{\sqrt{0.9 \cdot 0.1 \cdot n}}  > 1.28 \\
&300 - 0.9n > 1.28 \cdot 0.3 \sqrt{n} \\
&n < 325.6
\end{align*}
\end{enumerate}
\item[8.]
\begin{enumerate}
\item Выпишем случайную величину $X_i$ — цену акции после $i$-ого дня:
\[
X_i =
\begin{cases}
1.01, & p = 0.7 \\
0.99, & p = 0.2999 \\
0, & p = 0.0001
\end{cases}
\]
Нужно посчитать ожидание цены акциии после 20 дней:
\[
\E(X_1 \cdot \ldots \cdot X_{20}) \stackrel{\text{незав-ть}}{=} \E(X_1) \cdot \ldots \cdot \E(X_{20}) = 1.004^{20} \approx 1.083
\]
\item По ЗБЧ:
\[
\plim_{n\to\infty} \frac{1}{n} \sum_{i=1}^n X_i = \E(X_i) = 1.004
\]
\item Аналогично пункту (а):
\[
\E(X_1 \cdot \ldots \cdot X_{n}) = (\E(X_1))^n = 1.004^n
\]
И понятно, что $1.004^n \to_{n\to\infty} +\infty$.
\item
\begin{multline*}
\P(\text{разорения}) = 1 - \P(X_1 > 0, \ldots, X_n >0) = 1 - \prod_{i=1}^n \P(X_i > 0) \\
= 1 - (1 - 0.0001)^n \to_{n\to\infty} 1
\end{multline*}
\end{enumerate}
\end{enumerate}







\subsection{Теоретический минимум к кр3}

\begin{enumerate}
  \item Дайте определение нормально распределённой случайной величины. Укажите диапазон возможных значений, функцию плотности, ожидание, дисперсию. Нарисуйте функцию плотности.
  \item Дайте определение хи-квадрат распределения. Укажите диапазон возможных значений, выражение через нормальные распределения, математическое ожидание. Нарисуйте функцию плотности при разных степенях свободы.
  \item Дайте определение распределения Стьюдента. Укажите диапазон возможных значений, выражение через нормальные распределения. Нарисуйте функцию плотности распределения Стьюдента при разных степенях свободы на фоне нормальной стандартной функции плотности.
  \item Дайте определение распределения Фишера. Укажите диапазон возможных значений, выражение через нормальные распределенеия. Нарисуйте возможную функцию плотности.
\end{enumerate}

Для следующего блока вопросов предполагается, что
имеется случайная выборка $X_1$, $X_2$, \ldots, $X_n$ из распределения
с функцией плотности $f(x, \theta)$, зависящей от от параметра $\theta$. Дайте определение каждого понятия из списка или сформулируйте соответствующую теорему:

\begin{enumerate}[resume]
  \item Выборочное среднее и выборочная дисперсия;
  \item Формула несмещённой оценки дисперсии;
  \item Выборочный начальный момент порядка $k$;
  \item Выборочный центральный момент порядка $k$;
  \item Выборочная функция распределения;
  \item Несмещённая оценка $\hat \theta$ параметра $\theta$;
  \item Состоятельная последовательность оценок $\hat \theta_n$;
  \item Эффективность оценки $\hat \theta$ среди множества оценок $\hat \Theta$;
  \item Неравенство Крамера–Рао для несмещённых оценок;
  \item Функция правдоподобия и логарифмическая функция правдоподобия;
  \item Информация Фишера о параметре $\theta$, содержащаяся в одном наблюдении;
  \item Оценка метода моментов параметра $\theta$ при использовании первого момента, если $\E(X_i)=g(\theta)$ и существует обратная функция $g^{-1}$;
  \item Оценка метода максимального правдоподобия параметра $\theta$;
\end{enumerate}

Для следующего блока вопросов предполагается, что величины $X_1$, $X_2$, \ldots, $X_n$ независимы и нормальны $\cN(\mu;\sigma^2)$.

\begin{enumerate}[resume]
  \item Укажите закон распределения выборочного среднего, величины $\frac{\bar X - \mu}{\sigma/\sqrt{n}}$, величины $\frac{\bar X - \mu}{\hat\sigma/\sqrt{n}}$, величины $\frac{\hat\sigma^2(n-1)}{\sigma^2}$;
  \item Укажите формулу доверительного интервала с уровнем доверия $(1-\alpha)$ для $\mu$ при известной дисперсии, для $\mu$ при неизвестной дисперсии, для $\sigma^2$;
\end{enumerate}

\subsection{Задачный минимум к кр3}

\begin{enumerate}

\item Рост в сантиметрах (случайная величина $X$) и вес в килограммах (случайная величина $Y$) взрослого мужчины является нормальным случайным вектором $Z = (X, Y)$ с математическим ожиданием $\E(Z) = (175, 74)$ и ковариационной матрицей

\[
\Var(Z) =
\begin{pmatrix}
 49 & 28 \\
28 & 36
\end{pmatrix}
\]

Лишний вес характеризуется случайной величиной $U = X - Y$. Считается, что человек страдает избыточным весом, если $U < 90$.

\begin{enumerate}
\item Определите вероятность того, что рост мужчины отклоняется от среднего более, чем на $10$ см.
\item Укажите распределение случайной величины $U$. Выпишите её плотность распределения.
\item Найдите вероятность того, что случайно выбранный мужчина страдает избыточным весом.
\end{enumerate}

\item Рост в сантиметрах, случайная величина $X$, и вес в килограммах, случайная величина $Y$, взрослого мужчины является нормальным случайным вектором $Z = (X, Y)$ с математическим ожиданием $\E(Z) = (175, 74)$ и ковариационной матрицей

\[
\Var(Z) =
\begin{pmatrix}
 49 & 28 \\
28 & 36
\end{pmatrix}
\]

\begin{enumerate}
\item Найдите средний вес мужчины при условии, что его рост составляет $170$ см.
\item Выпишите условную плотность распределения веса мужчины при условии, что его рост составляет $170$ см.
\item Найдите условную вероятность того, что человек будет иметь вес, больший $90$ кг, при условии, что его рост составляет $170$ см.
\end{enumerate}

\item Для реализации случайной выборки $x=(1,0,-1,1)$ найдите:

\begin{enumerate}
\item выборочное среднее,
\item неисправленную выборочную дисперсию,
\item исправленную выборочную дисперсию,
\item выборочный второй начальный момент,
\item выборочный третий центральный момент,
\end{enumerate}

\item Для реализации случайной выборки $x=(1,0,-1,1)$ найдите:

\begin{enumerate}
\item вариационный ряд,
\item первый член вариационного ряда,
\item последний член вариационного ряда,
\item график выборочной функции распределения.
\end{enumerate}

\item Пусть $X=(X_1, \ldots,X_n)$ — случайная выборка из дискретного распределения, заданного с помощью таблицы

\begin{center}
\begin{tabular}{cccc}
\toprule
 $x$ & $-3$  &$ 0 $  & $2 $  \\
 \midrule
 $\P(X_i = x)$ & $2/3 - \theta$ & $1/3$ & $\theta$ \\
 \bottomrule
\end{tabular}
\end{center}

Рассмотрите оценку $\hat{\theta} = \dfrac{\bar{X}+2}{5}$.

\begin{enumerate}
    \item Найдите $\E[\hat{\theta}]$.
    \item Является ли оценка $\hat{\theta}$ несмещенной оценкой неизвестного параметра $\theta$?
\end{enumerate}

\item Пусть $X=(X_1, \ldots ,X_n)$ — случайная выборка из распределения с плотностью распределения

\[
f(x,\theta) = \begin{cases}
\dfrac{6x(\theta - x)}{\theta^3} & \text{при } x \in [0;\theta], \\
0 & \text{при } x \not\in [0;\theta],
\end{cases}
\]


где $\theta > 0$ — неизвестный параметр распределения и $\hat{\theta} = \bar{X}$.

\begin{enumerate}
\item Является ли оценка $\hat{\theta} = \bar{X}$ несмещенной оценкой неизвестного параметра $\theta$?
\item Подберите константу $c$ так, чтобы оценка $\tilde{\theta} = c\bar{X}$ оказалась несмещенной оценкой неизвестного параметра $\theta$.
\end{enumerate}

\item Пусть $X = (X_1,X_2,X_3)$ — случайная выборка из распределения Бернулли с неизвестным параметром $p \in (0,1)$. Какие из следующих ниже оценкой являются несмещенными? Среди перечисленных ниже оценок найдите наиболее эффективную оценку:

\begin{itemize}
  \item $\hat{p}_1 = \dfrac{X_1+X_3}{2}$,
  \item $\hat{p}_2 = \frac{1}{4}X_1+\frac{1}{2}X_2+\frac{1}{4}X_3$,
  \item $\hat{p}_3 = \frac{1}{3}X_1+\frac{1}{3}X_2+\frac{1}{3}X_3$.
\end{itemize}

\item Пусть $X=(X_1, \ldots,X_n)$ — случайная выборка из распределения с плотностью

\[
f(x,\theta) =
\begin{cases}
\frac{1}{\theta} \ e^{-\frac{x}{\theta}} & \text{при } x \geq 0, \\
0 & \text{при } x < 0,
\end{cases}
\]
где $\theta > 0$ — неизвестный параметр.
Является ли оценка  $\hat{\theta}_n = \dfrac{X_1+...+X_n}{n+1}$ состоятельной?

\item Пусть $X=(X_1, \ldots ,X_n)$ — случайная выборка из распределения с плотностью распределения

\[
f(x,\theta) = \begin{cases}
\dfrac{6x(\theta-x)}{\theta^3} & \text{при } x \in [0;\theta], \\
0 & \text{при } x \not\in [0;\theta], \end{cases}
\]


где $\theta > 0$ — неизвестный параметр распределения. Является ли оценка \ $\hat{\theta}_n = \frac{2n+1}{n}\bar{X}_n$ состоятельной оценкой неизвестного параметра $\theta$?

\item Пусть $X=(X_1, \ldots ,X_n)$ — случайная выборка из распределения с плотностью распределения

\[
f(x,\theta) =
\begin{cases}
\dfrac{6x(\theta-x)}{\theta^3} & \text{при } x \in [0;\theta], \\
0 & \text{при } x \not\in [0;\theta],
\end{cases}
\]


где $\theta > 0$ — неизвестный параметр распределения. Используя центральный момент 2-го порядка, при помощи метода моментов найдите оценку для неизвестного параметра $\theta$.

\item Пусть $X=(X_1, \ldots,X_n)$ — случайная выборка. Случайные величины $X_1, \ldots, X_n$ имеют дискретное распределение, которое задано при помощи таблицы

\begin{center}
\begin{tabular}{cccc}
\toprule
 $x$ & $-3$  &$ 0 $  & $2 $  \\
 \midrule
 $\P(X_i = x)$ & $2/3 - \theta$ & $1/3$ & $\theta$ \\
 \bottomrule
\end{tabular}
\end{center}

Используя второй начальный момент, при помощи метода моментов найдите оценку неизвестного параметра $\theta$. Для реализации случайной выборки $x=(0,0,-3,0,2)$ найдите числовое значение найденной оценки параметра $\theta$.

\item Пусть $X=(X_1, \ldots,X_n)$ — случайная выборка из распределения с плотностью распределения

\[
f(x,\theta) =
\begin{cases}
\frac{2x}{\theta} \ e^{-\frac{x^2}{\theta}} & \text{при } x>0, \\
0 & \text{при } x \leq 0,
\end{cases}
\]

где $\theta > 0$. При помощи метода максимального правдоподобия найдите оценку неизвестного параметра $\theta$.

\item Пусть $X=(X_1, \ldots, X_n)$ – случайная выборка из распределения Бернулли с параметром $\P \in (0;1)$. При помощи метода максимального правдоподобия найдите оценку неизвестного параметра $\P$.

\item Пусть $X=(X_1, \ldots, X_n)$ — случайная выборка из распределения с плотностью

\[
f(x,\theta) =
\begin{cases}
\frac{1}{\theta} \ e^{-\frac{x}{\theta}} & \text{при } x \geq 0, \\
0 & \text{при } x < 0, \end{cases}
\]

где $\theta > 0$ — неизвестный параметр. Является ли оценка  $\hat{\theta} = \bar{X}$ эффективной?

\item Стоимость выборочного исследования генеральной совокупности, состоящей из трех страт, определяется по формуле $TC = c_1n_1 + c_2n_2 + c_3n_3$, где $c_i$ — цена одного наблюдения в $i$-ой страте, a $n_i$ — число наблюдений, которые приходятся на $i$-ую страту. Найдите $n_1$, $n_2$ и $n_3$, при которых дисперсия стратифицированного среднего достигает наименьшего значения, если бюджет исследования 8000 и имеется следующая информация:

\begin{center}
\begin{tabular}{cccc}
\toprule
 Страта & $1$ & $2$ & $3$  \\
 \midrule
 Среднее значение & $30$ & $40$ & $50$ \\
 Стандартная ошибка  & $5$ & $10$ & $20$ \\
 Вес & $25\%$ & $25\%$ & $50\%$ \\
 Цена наблюдения & $1$ & $5$ & $10$ \\
 \bottomrule
\end{tabular}
\end{center}

\end{enumerate}

Ответы:

\begin{enumerate}
\item
\begin{enumerate}
\item $\approx 0.15$
\item $U \sim \cN(101,29)$, $f(u) = \frac{1}{\sqrt{2\pi\cdot 29}}e^{-\frac{1}{2}\frac{(u-101)^2}{29}}$
\item $\approx 0.02$
\end{enumerate}
\item
\begin{enumerate}
\item $71.14$
\item $f(y|x=170) = \frac{1}{\sqrt{2\pi\cdot20}}e^{-\frac{1}{2}\frac{(y-71.14)^2}{20}}$
\item $\approx 0$
\end{enumerate}
\item
\begin{enumerate}
\item $0.25$
\item $0.6875$
\item $0.91(6)$
\item $0.75$
\item $-0.28125$
\end{enumerate}
\item
\begin{enumerate}
\item $-1, 0, 1, 1$
\item $-1$
\item $1$
\item $f(x) = \begin{cases}
0, & x < -1 \\
0.25, & -1 \leq x < 0 \\
0.5, & 0 \leq x < 1 \\
1, & x \geq 1
\end{cases}$
\end{enumerate}
\item
\begin{enumerate}
\item $\theta$
\item да
\end{enumerate}

\item
\begin{enumerate}
\item нет, оценка смещена
\item $c = 2$
\end{enumerate}
\item
\begin{enumerate}
\item все оценки несмещенные
\item $\hat{p}_3$ наиболее эффективная
\end{enumerate}
\item да
\item да
\item $\hat{\theta}_{MM} = \sqrt{\frac{\sum_{i=1}^n(X_i-\overline{X})^2\cdot20}{n}}$

\item $\hat{\theta}_{MM} = \frac{1}{5}\left(6 - \frac{1}{n}\sum_{i=1}^n X_i^2 \right)$, $\hat{\theta}_{MM} = 0.68$
\item $\hat{\theta}_{ML} = \frac{\sum_{i=1}^n x_i^2}{n}$
\item $\hat{p}_{ML} = \frac{\sum_{i=1}^n x_i}{n}$
\item да
\item $n_1 \approx 260$, $n_2 \approx 232$, $n_3 \approx 658$

\end{enumerate}




\end{document}
