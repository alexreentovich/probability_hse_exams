\documentclass[12pt]{article}

\usepackage{tikz} % картинки в tikz
\usepackage{microtype} % свешивание пунктуации

\usepackage{array} % для столбцов фиксированной ширины

\usepackage{indentfirst} % отступ в первом параграфе

\usepackage{sectsty} % для центрирования названий частей
\allsectionsfont{\centering}

\usepackage{amsmath} % куча стандартных математических плюшек

\usepackage{comment}
\usepackage{amsfonts}

\usepackage[top=2cm, left=1.2cm, right=1.2cm, bottom=2cm]{geometry} % размер текста на странице

\usepackage{lastpage} % чтобы узнать номер последней страницы

\usepackage{enumitem} % дополнительные плюшки для списков
%  например \begin{enumerate}[resume] позволяет продолжить нумерацию в новом списке
\usepackage{caption}

\usepackage{longtable}
\usepackage{multicol}
\usepackage{multirow}


\usepackage{fancyhdr} % весёлые колонтитулы
\pagestyle{fancy}
\lhead{Теория вероятностей}
\chead{}
\rhead{Минимум к контрольной \textnumero 4 по ТВ и МС}
\lfoot{}
\cfoot{}
\rfoot{\thepage/\pageref{LastPage}}
\renewcommand{\headrulewidth}{0.4pt}
\renewcommand{\footrulewidth}{0.4pt}



\usepackage{todonotes} % для вставки в документ заметок о том, что осталось сделать
% \todo{Здесь надо коэффициенты исправить}
% \missingfigure{Здесь будет Последний день Помпеи}
% \listoftodos --- печатает все поставленные \todo'шки


% более красивые таблицы
\usepackage{booktabs}
% заповеди из документации:
% 1. Не используйте вертикальные линии
% 2. Не используйте двойные линии
% 3. Единицы измерения - в шапку таблицы
% 4. Не сокращайте .1 вместо 0.1
% 5. Повторяющееся значение повторяйте, а не говорите "то же"



\usepackage{fontspec}
\usepackage{polyglossia}

\setmainlanguage{russian}
\setotherlanguages{english}

% download "Linux Libertine" fonts:
% http://www.linuxlibertine.org/index.php?id=91&L=1
\setmainfont{Linux Libertine O} % or Helvetica, Arial, Cambria
% why do we need \newfontfamily:
% http://tex.stackexchange.com/questions/91507/
\newfontfamily{\cyrillicfonttt}{Linux Libertine O}

\AddEnumerateCounter{\asbuk}{\russian@alph}{щ} % для списков с русскими буквами
\setlist[enumerate, 2]{label=\asbuk*),ref=\asbuk*}

%% эконометрические сокращения
\DeclareMathOperator{\Cov}{Cov}
\DeclareMathOperator{\Corr}{Corr}
\DeclareMathOperator{\Var}{Var}
\DeclareMathOperator{\E}{E}
\def \hb{\hat{\beta}}
\def \hs{\hat{\sigma}}
\def \htheta{\hat{\theta}}
\def \s{\sigma}
\def \hy{\hat{y}}
\def \hY{\hat{Y}}
\def \v1{\vec{1}}
\def \e{\varepsilon}
\def \he{\hat{\e}}
\def \z{z}
\def \hVar{\widehat{\Var}}
\def \hCorr{\widehat{\Corr}}
\def \hCov{\widehat{\Cov}}
\def \cN{\mathcal{N}}
\def \P{\mathbb{P}}


\begin{document}

\section{Теоретический минимум}


\begin{enumerate}
  \item Дайте определение ошибки первого и второго рода, критической области.
  \item Укажите формулу доверительного интервала с уровнем доверия $(1-\alpha)$ для вероятности успеха, построенного по случайной выборке большого размера из распределения Бернулли $Bin(1, p)$.  
\end{enumerate}

Для следующего блока вопросов предполагается, что величины $X_1$, $X_2$, \ldots, $X_n$ независимы и нормальны $\cN(\mu;\sigma^2)$.
Укажите формулу для статистики: 

\begin{enumerate}[resume]
  \item Статистика, проверяющая гипотезу о математическом ожидании при известной дисперсии $\sigma^2$, 
    и её распределение при справедливости основной гипотезы  $H_0$: $\mu = \mu_0$. 
  \item Статистика, проверяющая гипотезу о математическом ожидании при неизвестной дисперсии $\sigma^2$, 
    и её распределение при справедливости основной гипотезы  $H_0$: $\mu = \mu_0$. 
\end{enumerate}


Для следующего блока вопросов предполагается, что есть две независимые случайные выборки: 
выборка $X_1$, $X_2$, \ldots{ }размера $n_x$ из нормального распределения $\cN(\mu_x;\sigma^2_x)$ 
и выборка $Y_1$, $Y_2$, \ldots{ }размера $n_y$ из нормального распределения $\cN(\mu_y;\sigma^2_y)$. 

Укажите формулу для статистики или границ доверительного интервала:

\begin{enumerate}[resume]
  \item Доверительный интервал для разницы математических ожиданий, когда дисперсии известны;
  \item Доверительный интервал для разницы математических ожиданий, когда дисперсии не известны, но равны;
  \item Статистика, проверяющая гипотезу о разнице математических ожиданий при известных дисперсиях, 
    и её распределение при справедливости основной гипотезы $H_0$: $\mu_x - \mu_y = \Delta_0$;
  \item Статистика, проверяющая гипотезу о разнице математических ожиданий при неизвестных, но равных дисперсиях, 
    и её распределение при справедливости основной гипотезы $H_0$: $\mu_x - \mu_y = \Delta_0$;
  \item Статистика, проверяющая гипотезу о равенстве дисперсий, 
    и её распределение при справедливости основной гипотезы $H_0$: $\sigma^2_x = \sigma^2_y$.
\end{enumerate}


\section{Задачный минимум}



\end{document}
