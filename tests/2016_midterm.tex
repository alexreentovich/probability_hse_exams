\element{midterm_2016}{ % в фигурных скобках название группы вопросов
 \AMCcompleteMulti
  \begin{questionmult}{1} % тип вопроса (questionmult — множественный выбор) и в фигурных — номер вопроса
  Граф Сен-Жермен извлекает карты в случайном порядке из стандартной колоды в 52 карты без возвращения. Рассмотрим три события: $A$ — «первая карта — тройка»; $B$ — «вторая карта — семёрка»; $C$ — «третья карта — дама пик».
 %\begin{multicols}{3} % располагаем ответы в 3 колонки
   \begin{choices} % опция [o] не рандомизирует порядок ответов
      \correctchoice{События $A$ и $B$ зависимы, события $B$ и $C$ зависимы.}
      \wrongchoice{События $A$ и $B$ независимы, события $B$ и $C$ независимы.}
      \wrongchoice{События $A$ и $B$ независимы, события $B$ и $C$ зависимы.}
      \wrongchoice{События $A$ и $B$ зависимы, события $B$ и $C$ независимы.}
      \wrongchoice{События $A$ и $С$ независимы, события $B$ и $C$ зависимы.}
      \end{choices}
  %\end{multicols}
  \end{questionmult}
}



\element{midterm_2016}{ % в фигурных скобках название группы вопросов
 \AMCcompleteMulti
  \begin{questionmult}{2} % тип вопроса (questionmult — множественный выбор) и в фигурных — номер вопроса
Монетку подбрасывают три раза. Рассмотрим три события: $A$ — «хотя бы один раз выпала решка»; $B$ — «хотя бы один раз выпал орёл»; $C$ — «все три раза выпал орёл».
 %\begin{multicols}{3} % располагаем ответы в 3 колонки
   \begin{choices} % опция [o] не рандомизирует порядок ответов
     \correctchoice{События $A$ и $B$ совместны, события $A$ и $C$ несовместны.}
     \wrongchoice{События $A$ и $B$ несовместны, события $B$ и $C$ совместны.}
     \wrongchoice{События $A$ и $B$ несовместны, события $B$ и $C$ несовместны.}
     \wrongchoice{События $A$ и $B$ совместны, события $A$ и $C$ совместны.}
     \wrongchoice{События $A$ и $B$ несовместны, события $A$ и $C$ совместны.}
      \end{choices}
  %\end{multicols}
  \end{questionmult}
}


\element{midterm_2016_rejected}{ % в фигурных скобках название группы вопросов
 \AMCcompleteMulti
  \begin{questionmult}{3} % тип вопроса (questionmult — множественный выбор) и в фигурных — номер вопроса
  На шахматной доске в клетке A1 стоит белая ладья. На одну из оставшихся клеток случайным образом выставляется чёрная ладья. Вероятность того, что ладьи «бьют» друг друга равна
 \begin{multicols}{3} % располагаем ответы в 3 колонки
   \begin{choices} % опция [o] не рандомизирует порядок ответов
      \correctchoice{$14/63$}
      \wrongchoice{$1/2$}
      \wrongchoice{$16/64$}
      \wrongchoice{$14/64$}
      \wrongchoice{$16/63$}
      \wrongchoice{$15/64$}
      \end{choices}
  \end{multicols}
  \end{questionmult}
}


\element{midterm_2016}{ % в фигурных скобках название группы вопросов
 \AMCcompleteMulti
  \begin{questionmult}{3} % тип вопроса (questionmult — множественный выбор) и в фигурных — номер вопроса
  В школе три девятых класса: 9А, 9Б и 9В. В 9А классе — 50\% отличники, в 9Б — 30\%, в 9В — 40\%. Если сначала равновероятно выбрать один из трёх классов, а затем внутри класса равновероятно выбрать школьника, то вероятность выбрать отличника равна
 \begin{multicols}{3} % располагаем ответы в 3 колонки
   \begin{choices} % опция [o] не рандомизирует порядок ответов
      \correctchoice{$0.4$}
      \wrongchoice{$0.3$}
      \wrongchoice{$0.5$}
      \wrongchoice{$0.27$}
      \wrongchoice{$3/(3+4+5)$}
      \wrongchoice{$(3+4+5)/3$}
      \end{choices}
  \end{multicols}
  \end{questionmult}
}



\element{midterm_2016}{ % в фигурных скобках название группы вопросов
 \AMCcompleteMulti
  \begin{questionmult}{4} % тип вопроса (questionmult — множественный выбор) и в фигурных — номер вопроса
  Если $\P(A)=0.2$, $\P(B)=0.5$, $\P(A | B) = 0.3$, то
 \begin{multicols}{3} % располагаем ответы в 3 колонки
   \begin{choices} % опция [o] не рандомизирует порядок ответов
      \correctchoice{$\P(A \cap B) = 0.15$}
      \wrongchoice{$\P(A \cup B) = 0.8$}
      \wrongchoice{$\P(A \cap B) = 0.05$}
      \wrongchoice{$\P(A \cup B) = 0.7$}
      \wrongchoice{$\P(B \cup A) = 0.3$}
      \end{choices}
  \end{multicols}
  \end{questionmult}
}


\element{midterm_2016_rejected}{ % в фигурных скобках название группы вопросов
 \AMCcompleteMulti
  \begin{questionmult}{5} % тип вопроса (questionmult — множественный выбор) и в фигурных — номер вопроса
  Традиционно себя называют Стрельцами люди, родившиеся с 22 ноября по 21 декабря. Из-за прецессии земной оси линия Солнце–Земля указывает в созведие Стрельца в наше время с 17 декабря по 20 января. Предположим, что все даты рождения равновероятны. Вероятность того, что человек, называющий себя Стрельцом, родился в день, когда линия Солнце–Земля указывала в созвездие Стрельца, равна
 \begin{multicols}{3} % располагаем ответы в 3 колонки
   \begin{choices} % опция [o] не рандомизирует порядок ответов
      \correctchoice{$5/30$}
      \wrongchoice{$4/30$}
      \wrongchoice{$4/31$}
      \wrongchoice{$1/2$}
      \wrongchoice{$4/35$}
      \end{choices}
  \end{multicols}
  \end{questionmult}
}


\element{midterm_2016}{ % в фигурных скобках название группы вопросов
 \AMCcompleteMulti
  \begin{questionmult}{5} % тип вопроса (questionmult — множественный выбор) и в фигурных — номер вопроса
  Монетка выпадает орлом с вероятностью $0.2$. Вероятность того, что при 10 подбрасываниях монетка выпадет орлом хотя бы один раз, равна
 \begin{multicols}{3} % располагаем ответы в 3 колонки
   \begin{choices} % опция [o] не рандомизирует порядок ответов
      \correctchoice{$1 - 0.8^{10}$}
      \wrongchoice{$2/10$}
      \wrongchoice{$0.2^{10}$}
      \wrongchoice{$1/2$}
      \wrongchoice{$C_{10}^1 0.2^{1}0.8^9$}
      \wrongchoice{$C_{10}^1 0.8^{1}0.2^9$}
      \end{choices}
  \end{multicols}
  \end{questionmult}
}


\element{midterm_2016}{ % в фигурных скобках название группы вопросов
 \AMCcompleteMulti
  \begin{questionmult}{6} % тип вопроса (questionmult — множественный выбор) и в фигурных — номер вопроса
  Среди покупателей магазина мужчин и женщин поровну. Женщины тратят больше 1000 рублей с вероятностью 60\%, а мужчины — с вероятностью 30\%. Только что был пробит чек на сумму 1234 рубля. Вероятность того, что покупателем была женщина равна
 \begin{multicols}{3} % располагаем ответы в 3 колонки
   \begin{choices} % опция [o] не рандомизирует порядок ответов
      \correctchoice{$2/3$}
      \wrongchoice{$0.5$}
      \wrongchoice{$0.3$}
      \wrongchoice{$0.18$}
      \wrongchoice{$1/3$}
      \end{choices}
  \end{multicols}
  \end{questionmult}
}







\element{midterm_2016}{ % в фигурных скобках название группы вопросов
 \AMCcompleteMulti
  \begin{questionmult}{11} % тип вопроса (questionmult — множественный выбор) и в фигурных — номер вопроса
  Если $F_X(x)$ — функция распределения случайной величины, то
 \begin{multicols}{2} % располагаем ответы в 3 колонки
   \begin{choices} % опция [o] не рандомизирует порядок ответов
      \correctchoice{ $\P(X \in (a;b] = F_X(b) - F_X(a)$}
      \wrongchoice{$F_X(x)$ может принимать отрицательные значения}
      \wrongchoice{величина $X$ дискретна}
      \wrongchoice{величина $X$ непрерывна}
      \wrongchoice{$\lim\limits_{x \rightarrow -\infty} F_X(x) = 1 $}
      \wrongchoice{$F_X(x)$ может принимать значение 2016}
      \end{choices}
  \end{multicols}
  \end{questionmult}
}



\element{midterm_2016}{ % в фигурных скобках название группы вопросов
 \AMCcompleteMulti
  \begin{questionmult}{12} % тип вопроса (questionmult — множественный выбор) и в фигурных — номер вопроса
Функцией плотности случайной величины может являться функция
 \begin{multicols}{2} % располагаем ответы в 3 колонки
   \begin{choices} % опция [o] не рандомизирует порядок ответов
     \correctchoice{$ f(x) = \begin{cases}
            \frac{1}{x^2}, x \in [1,+ \infty) \\
            0,\text{ иначе}
        \end{cases}  $}

     \wrongchoice{$ f(x) = \begin{cases}
     				x - 1, x \in [0,1+\sqrt{3}] \\
     				0,\text{ иначе}
 				\end{cases}  $}

     \wrongchoice{$ f(x) = \begin{cases}
     			  x^2, x \in [0,2] \\
     				0,\text{ иначе}
 				\end{cases}  $}



     \wrongchoice{$ f(x) = \begin{cases}
     				-1, x \in [-1, 0] \\
     				0,\text{ иначе}
 				\end{cases}  $}

     \wrongchoice{$ f(x) = \frac{1}{\sqrt{2\pi}} e^{-x^2}  $}

      \end{choices}
  \end{multicols}
  \end{questionmult}
}




\element{midterm_2016}{ % в фигурных скобках название группы вопросов
 \AMCcompleteMulti
  \begin{questionmult}{13} % тип вопроса (questionmult — множественный выбор) и в фигурных — номер вопроса
  Известно, что $\E(X)=3$, $\E(Y)=2$, $\Var(X)=12$, $\Var(Y)=1$, $\Cov(X,Y)=2$. Ожидание $\E(XY)$ равно
 \begin{multicols}{3} % располагаем ответы в 3 колонки
   \begin{choices} % опция [o] не рандомизирует порядок ответов
      \correctchoice{8}
      \wrongchoice{6}
      \wrongchoice{0}
      \wrongchoice{2}
      \wrongchoice{5}
      \end{choices}
  \end{multicols}
  \end{questionmult}
}




\element{midterm_2016}{ % в фигурных скобках название группы вопросов
 \AMCcompleteMulti
  \begin{questionmult}{14} % тип вопроса (questionmult — множественный выбор) и в фигурных — номер вопроса
  Известно, что $\E(X)=3$, $\E(Y)=2$, $\Var(X)=12$, $\Var(Y)=1$, $\Cov(X,Y)=2$. Корреляция $\Corr(X,Y)$ равна
 \begin{multicols}{3} % располагаем ответы в 3 колонки
   \begin{choices} % опция [o] не рандомизирует порядок ответов
      \correctchoice{$\frac{1}{\sqrt{3}}$}
      \wrongchoice{$\frac{2}{\sqrt{13}}$}
      \wrongchoice{$\frac{1}{12}$}
      \wrongchoice{$\frac{1}{\sqrt{12}}$}
      \wrongchoice{$\frac{2}{12}$}
      \end{choices}
  \end{multicols}
  \end{questionmult}
}



\element{midterm_2016}{ % в фигурных скобках название группы вопросов
 \AMCcompleteMulti
  \begin{questionmult}{15} % тип вопроса (questionmult — множественный выбор) и в фигурных — номер вопроса
  Известно, что $\E(X)=3$, $\E(Y)=2$, $\Var(X)=12$, $\Var(Y)=1$, $\Cov(X,Y)=2$. Дисперсия $\Var(2X-Y+4)$ равна
 \begin{multicols}{3} % располагаем ответы в 3 колонки
   \begin{choices} % опция [o] не рандомизирует порядок ответов
      \correctchoice{41}
      \wrongchoice{49}
      \wrongchoice{53}
      \wrongchoice{57}
      \wrongchoice{45}
      \end{choices}
  \end{multicols}
  \end{questionmult}
}



\element{midterm_2016}{ % в фигурных скобках название группы вопросов
 \AMCcompleteMulti
  \begin{questionmult}{16} % тип вопроса (questionmult — множественный выбор) и в фигурных — номер вопроса
  Если случайные величины $X$ и $Y$ имеют совместное нормальное распределение с нулевыми математическими ожиданиями и единичной ковариационной матрицей, то
 \begin{multicols}{2} % располагаем ответы в 3 колонки
   \begin{choices} % опция [o] не рандомизирует порядок ответов
      \correctchoice{$X$ и $Y$ независимы}
      \wrongchoice{распределение $X$ может быть дискретным}
      \wrongchoice{существует такое $a>0$, что $\P(X=a)>0$}
      \wrongchoice{$\Corr(X,Y)>0$}
      \wrongchoice{$\Corr(X,Y)<0$}
      \wrongchoice{$\forall \alpha \in [0,1]: \Var(\alpha X + (1-\alpha)Y) = 0$}
      \end{choices}
  \end{multicols}
  \end{questionmult}
}


\element{midterm_2016}{ % в фигурных скобках название группы вопросов
 \AMCcompleteMulti
  \begin{questionmult}{17} % тип вопроса (questionmult — множественный выбор) и в фигурных — номер вопроса
  Если $\Corr(X, Y)= 0.5$ и $\Var(X)=\Var(Y)$, то $\Corr(X + Y, 2Y - 7)$ равна
 \begin{multicols}{2} % располагаем ответы в 3 колонки
   \begin{choices} % опция [o] не рандомизирует порядок ответов
      \correctchoice{$\sqrt{3}/2$}
      \wrongchoice{$\sqrt{2}/3$}
      \wrongchoice{$1$}
      \wrongchoice{$0$}
      \wrongchoice{$1/2$}
      \wrongchoice{$\sqrt{3}/3$}
      \end{choices}
  \end{multicols}
  \end{questionmult}
}
















\element{midterm_2016}{ % в фигурных скобках название группы вопросов
 \AMCcompleteMulti
  \begin{questionmult}{41} % тип вопроса (questionmult — множественный выбор) и в фигурных — номер вопроса
  Известно, что $\xi \sim U[0;\,1]$. Вероятность $\P(0.2<\xi<0.7)$ равна
 \begin{multicols}{3} % располагаем ответы в 3 колонки
   \begin{choices} % опция [o] не рандомизирует порядок ответов
      \correctchoice{$1/2$}
      \wrongchoice{$\int_{0.2}^{0.7}\frac{1}{\sqrt{2\pi}}\,e^{-t^2/2}\,dt$}
      \wrongchoice{$\int_{0}^{1}\frac{1}{\sqrt{2\pi}}\,e^{-t^2/2}\,dt$}
      \wrongchoice{$1/4$}
      \wrongchoice{$0.17$}
      \end{choices}
  \end{multicols}
  \end{questionmult}
}



\element{midterm_2016}{ % в фигурных скобках название группы вопросов
 \AMCcompleteMulti
  \begin{questionmult}{42} % тип вопроса (questionmult — множественный выбор) и в фигурных — номер вопроса
    Cлучайные величины $\xi_1, \, \ldots, \, \xi_n, \, \ldots$ независимы и имеют таблицы распределения
    \[
    \begin{tabular}{c|c|c}
      $\xi_i$                     & $-1$   & $1$   \\ \cline{1-3}
      $\P_{\xi_i}$        & $1/2$       & $1/2$   \\
    \end{tabular}
    \]
    Если $S_n = \xi_1 + \ldots + \xi_n$, то предел $\lim\limits_{n \rightarrow \infty}\P\Bigl(\frac{S_n - \E[S_n]}{\sqrt{\Var(S_n)}} > 1\Bigr)$ равен
 \begin{multicols}{3} % располагаем ответы в 3 колонки
   \begin{choices} % опция [o] не рандомизирует порядок ответов
     \correctchoice{$\int_{1}^{+\infty}\frac{1}{\sqrt{2\pi}}\,e^{-t^2/2}\,dt$}
     \wrongchoice{$\int_{-1}^{1}\frac{1}{\sqrt{2\pi}}\,e^{-t^2/2}\,dt$}
     \wrongchoice{$\int_{-\infty}^{1}\frac{1}{\sqrt{2\pi}}\,e^{-t^2/2}\,dt$}
     \wrongchoice{$\int_{1}^{+\infty}\frac{1}{2}\,e^{-t/2}\,dt$}
     \wrongchoice{$0.5$}
      \end{choices}
  \end{multicols}
  \end{questionmult}
}


\element{midterm_2016}{ % в фигурных скобках название группы вопросов
 \AMCcompleteMulti
  \begin{questionmult}{43} % тип вопроса (questionmult — множественный выбор) и в фигурных — номер вопроса
  Число посетителей сайта за один день является неотрицательной случайной величиной с математическим ожиданием 400 и дисперсией 400. Вероятность того, что за 100 дней общее число посетителей сайта превысит $40\,400$, приближённо равна
 \begin{multicols}{3} % располагаем ответы в 3 колонки
   \begin{choices} % опция [o] не рандомизирует порядок ответов
      \correctchoice{$0.0227$}
      \wrongchoice{$0.3413$}
      \wrongchoice{$0.1359$}
      \wrongchoice{$0.9772$}
      \wrongchoice{$0.0553$}
      \end{choices}
  \end{multicols}
  \end{questionmult}
}


\element{midterm_2016}{ % в фигурных скобках название группы вопросов
 \AMCcompleteMulti
  \begin{questionmult}{44} % тип вопроса (questionmult — множественный выбор) и в фигурных — номер вопроса
Размер выплаты страховой компанией является неотрицательной случайной величиной с математическим ожиданием $10\,000$ рублей. Согласно неравенству Маркова, вероятность того, что очередная выплата превысит $50\,000$ рублей, ограничена сверху числом
 \begin{multicols}{2} % располагаем ответы в 3 колонки
   \begin{choices} % опция [o] не рандомизирует порядок ответов
      \correctchoice{$0.2$}
      \wrongchoice{$0.3413$}
      \wrongchoice{$0.1359$}
      \wrongchoice{$0.4$}
      \wrongchoice{$0.5$}
      \wrongchoice{неравенство Маркова здесь неприменимо}
      \end{choices}
  \end{multicols}
  \end{questionmult}
}


\element{midterm_2016}{ % в фигурных скобках название группы вопросов
 \AMCcompleteMulti
  \begin{questionmult}{45} % тип вопроса (questionmult — множественный выбор) и в фигурных — номер вопроса
  Размер выплаты страховой компанией является неотрицательной случайной величиной с математическим ожиданием $50\,000$ рублей и стандартным отклонением $10\,000$ рублей. Согласно неравенству Чебышёва, вероятность того, что очередная выплата будет отличаться от своего математического ожидания не более чем на 20\,000 рублей, ограничена снизу числом
 \begin{multicols}{2} % располагаем ответы в 3 колонки
   \begin{choices} % опция [o] не рандомизирует порядок ответов
      \correctchoice{$3/4$}
      \wrongchoice{$1/2$}
      \wrongchoice{$2/5$}
      \wrongchoice{$1/4$}
      \wrongchoice{$3/5$}
      \wrongchoice{неравенство Чебышёва здесь неприменимо}
      \end{choices}
  \end{multicols}
  \end{questionmult}
}


\element{midterm_2016}{ % в фигурных скобках название группы вопросов
 \AMCcompleteMulti
  \begin{questionmult}{46} % тип вопроса (questionmult — множественный выбор) и в фигурных — номер вопроса
  Вероятность поражения мишени при одном выстреле равна $0.6$. Случайная величина $\xi_i$  равна $1$, если при $i$-ом выстреле было попадание, и равна $0$ в противном случае. Предел по вероятности последовательности $\frac{\xi_1^{2016} + \ldots + \xi_n^{2016}}{n}$ при $n \rightarrow \infty$ равен
 \begin{multicols}{3} % располагаем ответы в 3 колонки
   \begin{choices} % опция [o] не рандомизирует порядок ответов
      \correctchoice{$3/5$}
      \wrongchoice{$1/2$}
      \wrongchoice{$2/5$}
      \wrongchoice{$0.6^{2016}$}
      \wrongchoice{$3/4$}
      \end{choices}
  \end{multicols}
  \end{questionmult}
}





\element{midterm_2016}{ % в фигурных скобках название группы вопросов
 \AMCcompleteMulti
  \begin{questionmult}{51} % тип вопроса (questionmult — множественный выбор) и в фигурных — номер вопроса
  Правильный кубик подбрасывается 5 раз. Вероятность того, что ровно два раза выпадет шестерка равна
 \begin{multicols}{3} % располагаем ответы в 3 колонки
   \begin{choices} % опция [o] не рандомизирует порядок ответов
      \wrongchoice{$125/(2^4 3^5)$}
      \wrongchoice{$1/36$}
      \wrongchoice{$25/(2^5 3^5)$}
      \wrongchoice{$2/5$}
      \wrongchoice{$1/(2^5 3^5)$}
      \end{choices}
  \end{multicols}
  \end{questionmult}
}

\element{midterm_2016}{ % в фигурных скобках название группы вопросов
 \AMCcompleteMulti
  \begin{questionmult}{52} % тип вопроса (questionmult — множественный выбор) и в фигурных — номер вопроса
Правильный кубик подбрасывается 5 раз. Математическое ожидание и дисперсия числа выпавших шестерок равны соответственно
 \begin{multicols}{3} % располагаем ответы в 3 колонки
   \begin{choices} % опция [o] не рандомизирует порядок ответов
      \wrongchoice{$5/6$ и $5/36$}
      \wrongchoice{$0$ и $5/6$}
      \wrongchoice{$1$ и $5/6$}
      \wrongchoice{$0$ и $1$}
      \wrongchoice{$5/6$ и $1/5$}
      \wrongchoice{$5/6$ и $1/36$}
      \end{choices}
  \end{multicols}
  \end{questionmult}
}

\element{midterm_2016}{ % в фигурных скобках название группы вопросов
 \AMCcompleteMulti
  \begin{questionmult}{53} % тип вопроса (questionmult — множественный выбор) и в фигурных — номер вопроса
  Правильный кубик подбрасывается 5 раз. Наиболее вероятное число шестерок равняется
 \begin{multicols}{3} % располагаем ответы в 3 колонки
   \begin{choices} % опция [o] не рандомизирует порядок ответов
      \correctchoice{$0$ и $1$}
      \wrongchoice{только $0$}
      \wrongchoice{только $1$}
      \wrongchoice{$5/6$}
      \wrongchoice{$5$}
      \end{choices}
  \end{multicols}
  \end{questionmult}
}




\element{midterm_2016}{ % в фигурных скобках название группы вопросов
 \AMCcompleteMulti
  \begin{questionmult}{54} % тип вопроса (questionmult — множественный выбор) и в фигурных — номер вопроса
  Правильный кубик подбрасывается 5 раз. Математическое ожидание суммы выпавших очков равно
 \begin{multicols}{3} % располагаем ответы в 3 колонки
   \begin{choices} % опция [o] не рандомизирует порядок ответов
      \correctchoice{$17.5$}
      \wrongchoice{$21$}
      \wrongchoice{$18$}
      \wrongchoice{$18.5$}
      \wrongchoice{$3.5$}
      \end{choices}
  \end{multicols}
  \end{questionmult}
}

\element{midterm_2016}{ % в фигурных скобках название группы вопросов
 \AMCcompleteMulti
  \begin{questionmult}{55} % тип вопроса (questionmult — множественный выбор) и в фигурных — номер вопроса
  Случайный вектор $(\xi, \eta)^T$ имеет нормальное распределение
  $\cN \left(
  \begin{pmatrix}
    0 \\
    0
  \end{pmatrix};
  \begin{pmatrix}
    1 & 1/2 \\
    1/2 & 1
  \end{pmatrix}
\right)$ и функцию плотности $f_{\xi, \eta}(x, y) = \frac{1}{2\pi a} \exp\left(-\frac{1}{2a^2}(x^2-bxy+y^2) \right)$. При этом

 \begin{multicols}{3} % располагаем ответы в 3 колонки
   \begin{choices} % опция [o] не рандомизирует порядок ответов
      \correctchoice{$a=\sqrt{3}/2$, $b=1$}
      \wrongchoice{$a=1$, $b=1$}
      \wrongchoice{$a=\sqrt{3/4}$, $b=0$}
      \wrongchoice{$a=1/2$, $b=1$}
      \wrongchoice{$a=1$, $b=0$}
      \end{choices}
  \end{multicols}
  \end{questionmult}
}

\element{midterm_2016}{ % в фигурных скобках название группы вопросов
 \AMCcompleteMulti
  \begin{questionmult}{56} % тип вопроса (questionmult — множественный выбор) и в фигурных — номер вопроса
    Случайный вектор $(\xi, \eta)^T$ имеет нормальное распределение
    $\cN \left(
    \begin{pmatrix}
      0 \\
      0
    \end{pmatrix};
    \begin{pmatrix}
      1 & 1/2 \\
      1/2 & 1
    \end{pmatrix}
  \right)$. Если случайный вектор $z$ определён как $z=(\xi - 0.5\eta, \eta)^T$, то
 \begin{multicols}{2} % располагаем ответы в 3 колонки
   \begin{choices} % опция [o] не рандомизирует порядок ответов
      \correctchoice{$z$ является двумерным нормальным вектором}
      \wrongchoice{$z \sim \cN \left(
      \begin{pmatrix}
        0 \\
        0
      \end{pmatrix};
      \begin{pmatrix}
        1 & 0 \\
        0 & 1
      \end{pmatrix}
    \right)$}
      \wrongchoice{компоненты вектора $z$ зависимы}
      \wrongchoice{компоненты вектора $z$ коррелированы}
      \wrongchoice{$\xi - 0.5\eta \sim \cN(0;1)$}
      \wrongchoice{$(\xi - 0.5\eta)^2 + 2\eta^2 \sim \chi_2^2$}
      \end{choices}
  \end{multicols}
  \end{questionmult}
}




\element{midterm_2016}{ % в фигурных скобках название группы вопросов
 \AMCcompleteMulti
  \begin{questionmult}{57} % тип вопроса (questionmult — множественный выбор) и в фигурных — номер вопроса
    Случайный вектор $(\xi, \eta)^T$ имеет нормальное распределение
    $\cN \left(
    \begin{pmatrix}
      0 \\
      0
    \end{pmatrix};
    \begin{pmatrix}
      1 & 1/2 \\
      1/2 & 1
    \end{pmatrix}
  \right)$. Условное математическое ожидание и условная дисперсия равны
 \begin{multicols}{2} % располагаем ответы в 3 колонки
   \begin{choices} % опция [o] не рандомизирует порядок ответов
      \correctchoice{$\E(\xi | \eta=1)=1/2$, $\Var(\xi | \eta=1)=3/4$}
      \wrongchoice{$\E(\xi | \eta=1)=1/2$, $\Var(\xi | \eta=1)=1$}
      \wrongchoice{$\E(\xi | \eta=1)=0$, $\Var(\xi | \eta=1)=1$}
      \wrongchoice{$\E(\xi | \eta=1)=1$, $\Var(\xi | \eta=1)=1/2$}
      \wrongchoice{$\E(\xi | \eta=1)=1$, $\Var(\xi | \eta=1)=1$}
      \wrongchoice{$\E(\xi | \eta=1)=1/2$, $\Var(\xi | \eta=1)=1/4$}
      \end{choices}
  \end{multicols}
  \end{questionmult}
}









\element{midterm_2016table}{ % в фигурных скобках название группы вопросов
 \AMCcompleteMulti
  \begin{questionmult}{31} % тип вопроса (questionmult — множественный выбор) и в фигурных — номер вопроса
  Математическое ожидание случайной величины $X$ при условии $Y=0$ равно
 \begin{multicols}{3} % располагаем ответы в 3 колонки
   \begin{choices} % опция [o] не рандомизирует порядок ответов
      \correctchoice{$1$}
      \wrongchoice{$-1$}
      \wrongchoice{$0$}
      \wrongchoice{$1/6$}
      \wrongchoice{$1/3$}
      \end{choices}
  \end{multicols}
  \end{questionmult}
}



\element{midterm_2016table}{ % в фигурных скобках название группы вопросов
 \AMCcompleteMulti
  \begin{questionmult}{32} % тип вопроса (questionmult — множественный выбор) и в фигурных — номер вопроса
Вероятность того, что $X=0$ при условии $Y<1$ равна
 \begin{multicols}{3} % располагаем ответы в 3 колонки
   \begin{choices} % опция [o] не рандомизирует порядок ответов
     \correctchoice{$1/4$}

     \wrongchoice{$0$}

     \wrongchoice{$1/6$}

     \wrongchoice{$1/2$}

     \wrongchoice{$3/4$}

      \end{choices}
  \end{multicols}
  \end{questionmult}
}




\element{midterm_2016table}{ % в фигурных скобках название группы вопросов
 \AMCcompleteMulti
  \begin{questionmult}{33} % тип вопроса (questionmult — множественный выбор) и в фигурных — номер вопроса
  Дисперсия случайной величины $Y$ равна
 \begin{multicols}{3} % располагаем ответы в 3 колонки
   \begin{choices} % опция [o] не рандомизирует порядок ответов
      \correctchoice{$2/3$}
      \wrongchoice{$1/3$}
      \wrongchoice{$0$}
      \wrongchoice{$1$}
      \wrongchoice{$-1$}
      \end{choices}
  \end{multicols}
  \end{questionmult}
}




\element{midterm_2016table}{ % в фигурных скобках название группы вопросов
 \AMCcompleteMulti
  \begin{questionmult}{34} % тип вопроса (questionmult — множественный выбор) и в фигурных — номер вопроса
 Ковариация случайных величин $X$ и $Y$ равна:
 \begin{multicols}{3} % располагаем ответы в 3 колонки
   \begin{choices} % опция [o] не рандомизирует порядок ответов
      \correctchoice{$-1/3$}
      \wrongchoice{$-2/3$}
      \wrongchoice{$0$}
      \wrongchoice{$1/3$}
      \wrongchoice{$2/3$}
      \end{choices}
  \end{multicols}
  \end{questionmult}
}




\element{midterm_2016density}{ % в фигурных скобках название группы вопросов
 \AMCcompleteMulti
  \begin{questionmult}{35} % тип вопроса (questionmult — множественный выбор) и в фигурных — номер вопроса
 Вероятность того, что $X<0.5, Y<0.5$ равна:
 \begin{multicols}{3} % располагаем ответы в 3 колонки
   \begin{choices} % опция [o] не рандомизирует порядок ответов
      \correctchoice{$1/64$}
      \wrongchoice{$1/4$}
      \wrongchoice{$1/16$}
      \wrongchoice{$1/96$}
      \wrongchoice{$1/128$}
      \end{choices}
  \end{multicols}
  \end{questionmult}
}



\element{midterm_2016density}{ % в фигурных скобках название группы вопросов
 \AMCcompleteMulti
  \begin{questionmult}{36} % тип вопроса (questionmult — множественный выбор) и в фигурных — номер вопроса
  Условное распределение $X$ при условии $Y=1$ имеет вид
 \begin{multicols}{2} % располагаем ответы в 3 колонки
   \begin{choices} % опция [o] не рандомизирует порядок ответов
      \correctchoice{$ f(x) = \begin{cases}
     				3 x^2 , x \in [0,1] \\
     				0,\text{ иначе}
 				\end{cases}  $}
      \wrongchoice{$ f(x) = \begin{cases}
     				3 x , x \in [0,1] \\
     				0,\text{ иначе}
 				\end{cases}  $}
      \wrongchoice{$ f(x) = \begin{cases}
     				9 x^2 , x \in [0,1] \\
     				0,\text{ иначе}
 				\end{cases}  $}
      \wrongchoice{$ f(x) = \begin{cases}
     				9 x , x \in [0,1] \\
     				0,\text{ иначе}
 				\end{cases}  $}
      \wrongchoice{Не определено}
      \end{choices}
  \end{multicols}
  \end{questionmult}
}




\element{midterm_2016_density_before}{
% \newpage
\rule{\textwidth}{1pt} %2015ready
\textbf{В вопросах 31 и 32} совместное распределение пары величин $X$ и $Y$ задается функцией плотности
\[
f(x) = \begin{cases}
     				9 x^2 y^2, x \in [0,1], y \in [0,1] \\
     				0,\text{ иначе}
 				\end{cases}
\]
\vspace{0.2cm}
}






\element{midterm_2016_table_before}{
\rule{\textwidth}{1pt} %2015ready
\textbf{В вопросах 27–30} совместное распределение пары величин $X$ и $Y$ задано таблицей:

\begin{tabular}{c|ccc}
 & $Y=-1$ & $Y=0$ & $Y=1$ \\
\hline
$X=0$ & $0$ & $1/6$  &  $1/6$\\
$X=2$ & $1/3$ & $1/6$ &  $1/6$ \\
\end{tabular}


\vspace{0.2cm}
}
